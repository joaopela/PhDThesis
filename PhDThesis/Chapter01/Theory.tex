\chapter{Theory and motivations}
\label{CHAPTER:TheoryAndMotivations}

\glsresetall % Resetting all acronyms

% STATUS: DONE

The goal of particle physics is to study the most fundamental constituents of matter and understand how they interact with each other. The \gls{SM} of particle physics will briefly introduced and the Higgs mechanism explained and the search for Higgs boson decaying invisibly decays will be motivated. Throughout this chapter Einstein summation convention, Feynman slash notation and natural units are used, where $\hbar=c=1$. Additionally, greek letters are used to label the four vectors, and gauge group generators use roman letters.

%%%%%%%%%%%%%%%%%%%%%%%%%%%%%%%%%%%%%%%%%%%%%%%%%%%%%%%%%%%%%%%%%%%%%%%%%%%%%%%%%%%%%%%
%%% SECTION
%%%%%%%%%%%%%%%%%%%%%%%%%%%%%%%%%%%%%%%%%%%%%%%%%%%%%%%%%%%%%%%%%%%%%%%%%%%%%%%%%%%%%%%
\section{Standard Model of Particle Physics}

% STATUS: DONE

The \gls{SM} of particle physics is a \gls{QFT} including both relativistic and quantum mechanical effects. It describes the electromagnetic, weak nuclear and strong forces and their interaction with matter. This theory is one of the most successful theories ever made and was able to describe data from a wide range of experimental measurements. Before its discovery in 2012 \cite{ARTICLE:ATLAS_HiggsDiscovery,ARTICLE:CMS_HiggsDiscovery} the Higgs boson was the only missing particle that was predicted by this theory and not yet found. 

Although its success, the \gls{SM} does not explain some phenomena observed in nature, like the presence of large quantity of \textit{dark matter} in the universe, or the even more mysterious \textit{dark energy}. The discovery of the Higgs boson could allow to probe the production of dark matter directly, through its decay into these elusive particles. 

%%%%%%%%%%%%%%%%%%%%%%%%%%%%%%%%%%%%%%%%%%%%%%%%%%%%%%%%%%%%%%%%%%%%%%%%%%%%%%%%%%%%%%%
%%% SUBSECTION
%%%%%%%%%%%%%%%%%%%%%%%%%%%%%%%%%%%%%%%%%%%%%%%%%%%%%%%%%%%%%%%%%%%%%%%%%%%%%%%%%%%%%%%
\subsection{Fundamental matter particles}
\label{SUBSECTION:Theory_SM_ParticlesAndForces}

% STATUS: DONE 

All particles of matter considered fundamental observed until now are spin-$\frac{1}{2}$ fermions. The equations of motion for a spin-$\frac{1}{2}$ dirac fermion with a mass $m$ can be found in equation \ref{EQUATION:Theory_SM_ParticlesAndForces_Dirac}.

\begin{equation}
(i\gamma^{\mu}\partial_{\mu} - m)\psi = 0
\label{EQUATION:Theory_SM_ParticlesAndForces_Dirac}
\end{equation}

In this equation the matrices $\gamma^{\mu}$, $\mu\in{0,1,2,3}$, are defined by the anti-commutator relation $\gamma^{\mu}\gamma^{\nu}+\gamma^{\mu}\gamma^{\nu} = 2\eta^{\mu\nu}I_{4}$ where $\eta^{\mu\nu}$ is the flat space-time metric $(+,-,-,-)$ and $I_{4}$ is the $4\times4$ identity matrix. The solutions for the Dirac fermion equation of motion, $\psi$, are the massive particle and anti-particle states, with momentum $\mathbf{p}$ and energy $E$, which satisfy the relativistic expression, $E^{2} = \mathbf{p}\cdot\mathbf{p} + m^{2}$.

Fundamental fermions can be split in two categories depending if they interact (quarks) or not (leptons) with the strong nuclear force. Both this categories of particles can grouped into three generations, with similar properties between them but increasing mass. While lepton can be examined isolated, free quarks are not observed in nature, they are confined in composed structures of three (baryons) or two (mesons) quarks. Table \ref {TABLE:Theory_SM_ParticlesAndForces_MatterParticle} shows a summary of the know fundamental.

\begin{table}[!htb]
\centering
\begin{tabular}{|c||c|c|c||c|c|c|}
\hline
 & \multicolumn{3}{c||}{Leptons (J=1/2)} & \multicolumn{3}{c|}{Quarks (J=1/2)} \\
\hline
Generation                &     Symbol &            Mass & Q/e & Symbol &           Mass & Q/e \\
\hline\hline
\multirow{2}{*}{$1^{st}$} & e          &     $511\,\keV$ &   1 &      u &    $2.3\,\MeV$ &  2/3 \\
                          & $\nu_e$    &     $< 2\, \eV$ &   0 &      d &    $4.8\,\MeV$ & -1/3 \\
\hline
\hline
\multirow{2}{*}{$2^{nd}$} & $\mu$      &  $   106\,\MeV$ &   1 &      c &  $1.275\,\GeV$ &  2/3 \\
                          & $\nu_\mu$  &  $< 0.19\,\MeV$ &   0 &      s &     $95\,\MeV$ & -1/3 \\
\hline
\hline
\multirow{2}{*}{$3^{rd}$} & $\tau$     & $   1777\,\MeV$ &   1 &      t & $173.21\,\GeV$ &  2/3 \\
                          & $\nu_\tau$ & $< 18.2 \,\MeV$ &   0 &      b &   $4.66\,\GeV$ & -1/3 \\
\hline
\end{tabular}
\caption[List of leptons and their fundamental properties]{List of fermions grouped in generations and split in fermions and quarks and their fundamental properties \cite{ARTICLE:PDG2014}.}
\label{TABLE:Theory_SM_ParticlesAndForces_MatterParticle}
\end{table}

%%%%%%%%%%%%%%%%%%%%%%%%%%%%%%%%%%%%%%%%%%%%%%%%%%%%%%%%%%%%%%%%%%%%%%%%%%%%%%%%%%%%%%%
%%% SUBSECTION
%%%%%%%%%%%%%%%%%%%%%%%%%%%%%%%%%%%%%%%%%%%%%%%%%%%%%%%%%%%%%%%%%%%%%%%%%%%%%%%%%%%%%%%
\subsection{Fundamental forces}
\label{SUBSECTION:Theory_SM_FundamentalForces}

% STATUS: DONE 

Gauge bosons mediate the fundamental forces of nature. All the currently observed force mediators are spin-1 particles, which is consequence of symmetries in the relevant theory possesses. The \gls{QFT} that describes the electromagnetism is \gls{QED}, and the strong nuclear force is \gls{QCD}, both these theories describe massless mediator bosons, the photon and the gluons, they are appear as a direct consequence of the gauge invariance of those theories. A fundamental difference between these interactions is their range, while the electromagnetism is effectively infinite, the scale of the strong force is of around $10^{-15}\,\meter$.

The weak and electromagnetic forces mediator appear from the unification of the weak and electromagnetic interactions theories and the mixing of the associated gauge fields. The $W^{\pm}$ and $Z$ bosons are responsible to mediate the weak force and have a non-zero mass which has been measured experimentally~\cite{ARTICLE:TEVATRONcombinedWmass,ARTICLE:DELPHIMassZ,ARTICLE:PDG2014}. Table \ref{TABLE:Theory_SM_ParticlesAndForces_BosonProperties} contains a summary of the fundamental gauge bosons of the Standard Model. The description of gravity is currently not included in the Standard Model, but as its interaction strength is much smaller than the other three forces and should not have any impact in its predictions.

\begin{table}[!htb]
  \centering
  \begin{tabular}{|c|c|c|c|c|}
  \hline
  \multicolumn{4}{|c|}{Bosons} \\
  \hline
   Particle Name & Mass ($GeV/c^2$) &     Q/e & Spin \\
  \hline
  \hline
  Photon ($\gamma$) &               0 &       0 &    1 \\
  \hline
  $W^\pm$           &            80.4 & $\mp 1$ &    1 \\
  $Z^0$             &            91.2 &       0 &    1 \\
  \hline
  Gloun (g)         &               0 &       0 &    1 \\
  Higgs ($H^0$)     &         $> 114$ &       0 &    0 \\
  \hline
  \end{tabular}
  \caption[List of bosons and their fundamental properties]{List of bosons and their fundamental properties}
  \label{TheoreticalIntroduction_BosonProperties}
\end{table}


%%%%%%%%%%%%%%%%%%%%%%%%%%%%%%%%%%%%%%%%%%%%%%%%%%%%%%%%%%%%%%%%%%%%%%%%%%%%%%%%%%%%%%%
%%% SUBSECTION
%%%%%%%%%%%%%%%%%%%%%%%%%%%%%%%%%%%%%%%%%%%%%%%%%%%%%%%%%%%%%%%%%%%%%%%%%%%%%%%%%%%%%%%
\subsection{Electroweak Gauge Symmetry}
\label{SUBSECTION:Theory_SM_ElectroweakGaugeSymmetry}

% STATUS: DONE 

Symmetries in nature normally appear as a direct consequence of the fundamental law. It can be showed that if a physical system can be described within the Lagrangian formalism, all symmetries that can be found on the system Lagrangian have an associated conserved quantity~\cite{ARTICLE:InvariantVariationProblems}. These property can be applied when using dynamical quantum theories to constrain the Lagrangian of particle interactions, where the characteristics of the interaction itself allows the identification of transformation under which the Lagrangian should be symmetric.

The development of the \gls{SM} had one of its greatest successes in the unification of the electromagnetic and weak interactions~\cite{ARTICLE:Glashow,ARTICLE:Weinberg,ARTICLE:Salam}. The unification of these theories appears by associating them to a particular symmetry group. The characteristics of the electroweak interaction are described by a Lagrangian which is invariant under transformations of the group $\sutwol\times\uone$. The quantum numbers in these electroweak theory are the weak isospin $t_{1,2,3}$ and hypercharge $y$, which are related to the electric charge as expressed in equation \ref{EQUATION:Theory_SM_ElectroweakGaugeSymmetry_ElecCharge}.

\begin{equation}
Q = t_{3} + \frac{y}{2}
\label{EQUATION:Theory_SM_ElectroweakGaugeSymmetry_ElecCharge}
\end{equation}

These quantum numbers are associated with gauge fields. The weak isospin fields $W_{\mu}^{i}$, $i = 1,2,3$ and the hypercharge field $B_{\mu}$. The weak isospin fields act on doublets like the on in equation \ref{EQUATION:Theory_SM_ElectroweakGaugeSymmetry_Doublet}.
 
\begin{equation}
\psi_{L} =  \begin{pmatrix} u_{i} \\ d_{i} \end{pmatrix}_{L} ,   
\begin{pmatrix} \nu_{i} \\ {l_{i}} \end{pmatrix}_{L}
\label{EQUATION:Theory_SM_ElectroweakGaugeSymmetry_Doublet}
\end{equation}

In this equation the $u_{i}$ and $d_{i}$ are up-type and down-type quarks respectfully, the $l_{i}$ are charged leptons and the $\nu_{i}$ are the corresponding neutrinos. The index $i$ identifies the generation of fermions. The weak force only interacts with left handed fermions, which is indicatd by the subscript $L$, which makes it maximally parity violating. The fermion right handed projections $\psi_{R}$ are invariant under $SU(2)_{L}$ and transform as singlet states.
 
The physical electronweak boson fields, $W_{\mu}^{\pm}$, $A_{\mu}$ and photon field $Z_{\mu}^{0}$ result from the mixing between the electroweak gauge fields as it can be seen in equation \ref{EQUATION:Theory_SM_ElectroweakGaugeSymmetry_W} and \ref{EQUATION:Theory_SM_ElectroweakGaugeSymmetry_AZ}.


\begin{equation}
W_{\mu}^{\pm} = \frac{1}{\sqrt{2}}\left(W_{\mu}^{1} \mp W_{\mu}^{2}\right) ,\label{EQUATION:Theory_SM_ElectroweakGaugeSymmetry_W}\\
\begin{pmatrix} A_{\mu} \\ Z_{\mu}^{0} \end{pmatrix} = 
\begin{pmatrix} \cos{\theta_{W}} & \sin{\theta_{W}} \\ -\sin{\theta_{W}} &
\cos{\theta_{W}} \end{pmatrix} . 
\begin{pmatrix} B_{\mu} \\ W_{\mu}^{3} \end{pmatrix} ,
\label{EQUATION:Theory_SM_ElectroweakGaugeSymmetry_AZ}
\end{equation}

Where  $\theta_{W}$ is the weak mixing angle, which is connected to the coupling of the the weak neutral ($g$) and electromagnetic interactions ($g'$) through the relation $\theta_{W}=\tan^{-1}{\frac{g'}{g}}$. The Gargamelle bubble chamber experiment at CERN discovered the weak neutral currents in 1973~\cite{ARTICLE:GargamelleNeutrinoObservation}, while the $\PZ$ and $\PWpm$ were discovered by the UA1 and UA2 collaborations at CERN in 1983~\cite{ARTICLE:UA1WObservation,ARTICLE:UA2WObservation,ARTICLE:UA1ZObservation,ARTICLE:UA2ZObservation}.

This model construction leads to a Lagrangian that does not have any mass terms. Adding mass terms of the form $-M^{2}W_{\mu}W^{\mu}$ cannot be done since it would break gauge invariance. Adding fermion will mass terms of the form $-m\overline{\psi}\psi = -m(\overline{\psi_{R}}\psi_{L} + \overline{\psi_{L}}\psi_{R})$, where $\overline\psi$ is the adjoint of the field $\psi$, has field pairs of left and right handed which will transform differently under the $SU(2)_{L}$ and $U(1)_{Y}$ groups, and as a consequence will also break gauge invariance.

The photon mass has been experimentally measured to be compatible with zero within errors, but $\PW$ and $\PZ$ have masses of the order of $\approx 80\,\GeV$ and $\approx91\,\GeV$ respectively~\cite{ARTICLE:PDG2014}. Therefore the electroweak symmetry must be spontaneously broken to reconcile theory and observation. The Higgs mechanism is the electroweak symmetry breaking in the \gls{SM}.

%%%%%%%%%%%%%%%%%%%%%%%%%%%%%%%%%%%%%%%%%%%%%%%%%%%%%%%%%%%%%%%%%%%%%%%%%%%%%%%%%%%%%%%
%%% SUBSECTION
%%%%%%%%%%%%%%%%%%%%%%%%%%%%%%%%%%%%%%%%%%%%%%%%%%%%%%%%%%%%%%%%%%%%%%%%%%%%%%%%%%%%%%%
\subsection{The Higgs Mechanism in the Standard Model}
\label{SUBSECTION:Theory_SM_HiggsMechanism}

% STATUS: DONE 

In quantum field theory, a symmetry is spontaneously broken when the Lagrangian itself remains invariant but the vacuum state, where the Hamiltonian of the theory is at its minimum, does not~\cite{ARTICLE:AitchisonGaugeTheories}. For the electroweak theory, this symmetry breaking is obtained with the introduction of a complex scalar field has a non-zero vacuum expectation value (VEV)~\cite{ARTICLE:HiggsBrokenSymmetries1,ARTICLE:HiggsBrokenSymmetries2,ARTICLE:GlobalConservation,ARTICLE:HiggsSpontaneousSymmetryBreakdown,ARTICLE:SymmetryBreaking}. This field is an $SU(2)$ doublet as represented in equation~\ref{EQUATION:Theory_SM_HiggsMechanism_Doublet}.

\begin{equation}
\phi = \begin{pmatrix}\phi^{+} \\ \phi^{0} \end{pmatrix}
\label{EQUATION:Theory_SM_HiggsMechanism_Doublet}
\end{equation}

The electroweak Lagrangian can now be expressed in the simple form present in equation~\ref{EQUATION:Theory_SM_HiggsMechanism_EWKLagrangian}.

\begin{equation}
\mathcal{L}_{EW} = -\frac{1}{4}(\mathbf{F_{\mu\nu}}\cdot\mathbf{F^{\mu\nu}} +
G_{\mu\nu}G^{\mu\nu}),
\label{EQUATION:Theory_SM_HiggsMechanism_EWKLagrangian}
\end{equation}

In this Lagrangian $\mathbf{F_{\mu\nu}}$ is the weak isospin and $G_{\mu\nu}$ is the field strength tensor, which are related to the fields through equations~\ref{EQUATION:Theory_SM_HiggsMechanism_Field1} and~\ref{EQUATION:Theory_SM_HiggsMechanism_Field2}.

\begin{equation}
\mathbf{F_{\mu\nu}} = \partial_{\mu} \mathbf{W_{\nu}} - \partial_{\nu}\mathbf{W_{\mu}} -g\mathbf{W_{\mu}} \times \mathbf{W_{\nu}} \label{EQUATION:Theory_SM_HiggsMechanism_Field1}\\ 
G_{\mu\nu} = \partial_{\mu} B_{\nu} - \partial_{\nu}B_{\mu} \label{EQUATION:Theory_SM_HiggsMechanism_Field2}
\end{equation}

Where $\mathbf{W_{\mu}}= (W_{\mu}^{1},W_{\mu}^{2},W_{\mu}^{3})$. An additional term appears as a consequence of the introduction of the complex scalar field as expressed in equation~\ref{EQUATION:Theory_SM_HiggsMechanism_ExtraTerm1}.

\begin{equation}
\mathcal{L_{\phi}} = (D_{\mu}\phi)^{\dagger}(D^{\mu}\phi) - V(\phi) \label{EQUATION:Theory_SM_HiggsMechanism_ExtraTerm1} \\ 
\text{with } D_{\mu} = \partial_{\mu} - \frac{1}{2}(igT_{i}W_{\mu}^{i} - ig'B_{\mu}) \label{EQUATION:Theory_SM_HiggsMechanism_ExtraTerm2}
\end{equation}

where $T_{i}$ are the $SU(2)$ group generators, and $V(\phi)$ is the potential term which can be found in equation \ref{EQUATION:Theory_SM_HiggsMechanism_Potential}.

\begin{equation}
V(\phi) = \lambda(\phi^{\dagger}\phi)^{2} - \mu_{SM}\phi^{\dagger}\phi
\label{EQUATION:Theory_SM_HiggsMechanism_Potential}
\end{equation}

Where $\lambda$ and $\mu_{SM}$ are constants which take into account the self-interactions and the masses of the scalar fields. The vacuum states correspond to the minima of $V(\phi)$ and its expectation values of $\bra{0}\phi\ket{0}$, which are expressed in equation \ref{EQUATION:Theory_SM_HiggsMechanism_VEV}.

\begin{equation}
\bra{0}\phi\ket{0} = \frac{1}{\sqrt{2}} \begin{pmatrix}0 \\ v \end{pmatrix},\text{ with } v=\sqrt{\frac{\mu_{SM}^{2}}{\lambda}} 
\label{EQUATION:Theory_SM_HiggsMechanism_VEV}
\end{equation}

In order to obtain physical particles, the perturbations around the vacuum state are taken into account. If $\theta_{i}$ and $H_{SM}$ are small variations in four degrees if freedom of the field $\phi$ is can be expressed as equation \ref{EQUATION:Theory_SM_HiggsMechanism_Phi}.

\begin{equation}
\phi = \exp(-i\theta_{i}T^{i}/2v)\frac{1}{\sqrt{2}}\begin{pmatrix} 0 \\ v+H_{SM} \end{pmatrix}
\label{EQUATION:Theory_SM_HiggsMechanism_Phi}
\end{equation}

The phase fields $\theta_{i}$ can be set to zero by an appropriate gauge transformation which only leaves $H_{SM}$. This result can now be inserted into the Lagrangian, where $H_{SM}$ is the a scalar field with mass $\sqrt{2}\mu_{SM}$ which in turn means the $W_{\mu}^{\pm}$ and $Z_{\mu}^{0}$ fields acquire mass terms $m_{\PW}$ and $m_{\PZ}$ as expressed in equation \ref{EQUATION:Theory_SM_HiggsMechanism_WeakBosonMass}.

\begin{equation}
m_{\PW} = m_{\PZ}\cos{\theta_{W}}=\frac{gv}{2}.
\label{EQUATION:Theory_SM_HiggsMechanism_WeakBosonMass}
\end{equation}

Finally, equation \ref{EQUATION:Theory_SM_HiggsMechanism_FermionsMass} shows the form of mass terms for the fermions which are introduced via Yukawa interactions between the fermion and Higgs fields.

\begin{equation}
-\lambda_{f}( \overline{\psi_{L}}\phi\psi_{R} + \overline{\psi_{R}}\phi\psi_{L}),  
\label{EQUATION:Theory_SM_HiggsMechanism_FermionsMass}
\end{equation}

Where $\lambda_{f}$ is the coupling for each fermion. Heavier fermions will have a stronger coupling to the Higgs boson, the value of $\lambda_{f}$ will vary proportionally to the mass of the fermion $m_{f}$. The value of each coupling are not predicted in the \gls{SM} and have to be determined experimentally. The values of $\sin{\theta_{W}}$, $v$ and $g$ can be determined with experimental values of the $\PW$ and $\PZ$ masses and the fine structure constant, but the value of $\mu_{SM}$ cannot be predicted. The mass of a Higgs boson associated with the Higgs field is $m_{\PH}=\sqrt{2}\mu_{SM}$, so it cannot be predicted directly by the \gls{SM}, but indirect constraints can be imposed from theoretical considerations with the help of precision electroweak data~\cite{SITE:lepewwg}.

%%%%%%%%%%%%%%%%%%%%%%%%%%%%%%%%%%%%%%%%%%%%%%%%%%%%%%%%%%%%%%%%%%%%%%%%%%%%%%%%%%%%%%%
%%% SUBSECTION
%%%%%%%%%%%%%%%%%%%%%%%%%%%%%%%%%%%%%%%%%%%%%%%%%%%%%%%%%%%%%%%%%%%%%%%%%%%%%%%%%%%%%%%
\subsection{Searching for the SM Higgs boson}
\label{SUBSECTION:Theory_SM_SearchingSMHiggs}

As described in the previous section the Higgs boson mass is not directly predicted by the \gls{SM}, which implies searches for this particle need to be performed covering the widest possible mass range. In particle accelerators these searches are performed by looking for specific Higgs boson decays to either bosons of fermions. The coupling of the Higgs to a specific final state depends in both the mass of the Higgs and the masses of the particles of the final state. 

The production of the \gls{SM} Higgs boson the proton-proton collisions occurs primarily through gluon-gluon fusion, \gls{VBF} and associated production with either a vector boson or a pair of top quarks, the processes are illustrated in figure \ref{SUBSECTION:Theory_SM_SearchingSMHiggs}. The Higgs boson only couple with particles that have mass, hence it does not couple to gluons which are massless. The gluon fusion production process goes mainly through a top quark loop which is the heaviest quark and thus also has the highest coupling with the higgs. All these four processes are accessible at the \gls{LHC}.

\begin{figure}[htbp]
\subfloat[]{\includegraphics[width=0.45\textwidth]{Chapter01/Images/feynman_ggH.pdf}} \qquad
\subfloat[]{\includegraphics[width=0.45\textwidth]{Chapter01/Images/feynman_qqH.pdf}} \\
\subfloat[]{\includegraphics[width=0.45\textwidth]{Chapter01/Images/feynman_VH.pdf}}  \qquad
\subfloat[]{\includegraphics[width=0.45\textwidth]{Chapter01/Images/feynman_ttH.pdf}} \\
\caption[Feynman diagrams for the main production processes of the SM Higgs
boson.]{Feynman diagrams for the main production processes of the \gls{SM} Higgs
boson. Shown is (a) gluon fusion, (b) vector boson fusion and
associated production with (c) vector bosons and (d) top quarks.}
\label{FIGURE:SMFeynmanDiagrams}
\end{figure}

% Previous searches for the Higgs boson were carried out at LEP and the
% Tevatron. LEP collided electrons and positrons at centre of mass
% energies ($\sqrt{s}$) between $90$ and $209\,\GeV$. In such collisions the Higgs
% boson is predominantly produced in association with a $\PZ$ boson, a process
% referred to as ``Higgsstrahlung''. The decay channels used were predominantly decays
% to $\Pqb\Paqb$ and $\Pgtp\Pgtm$ pairs. The fact that the Higgs boson was not
% observed at LEP yields the exclusion of masses with $m_{\PH}<114.4\,\GeV$ at the
% 95\% \ac{CL} \cite{Barate:2003sz}. Searches were also performed by the CDF and D0 Collaborations at
% the Tevatron accelerator, which collided protons and antiprotons with
% $\sqrt{s}=1.96\,\TeV$. The searches were performed in a mass range of
% $90$--$200\,\GeV$ and focussed on decays into $\Pqb\Paqb$, $\PWp\PWm$,
% $\gamma\gamma$ and $\Pgtp\Pgtm$ pairs, with  $\Pqb\Paqb$ and $\PWp\PWm$ offering
% the highest sensitivity. The combined results from the Tevatron yielded
% exclusions of $m_{\PH}$ in the ranges $90$--$109\,\GeV$ and $149$--$182\,\GeV$
% \cite{Aaltonen:2013kxa}. The results of such direct searches
% can be combined with precision measurements of electroweak observables at LEP
% and by the SLAC Large Detector (SLD) to constrain the mass of the Higgs
% boson to $94^{+29}_{-24}\,\GeV$ \cite{lepewwg}, 
% where the uncertainty incorporates experimental effects only.
% 
% The LHC provides a higher centre of mass energy than the Tevatron, and so
% gives access to higher cross sections and a wider range of Higgs masses. Figure
% \ref{fig:SMHiggsXS} indicates the cross sections of the different production
% processes, for proton-proton collisions at $\sqrt{s}=8\,\TeV$ as used during the 
% 2012 operating period of the LHC. The gluon fusion production process dominates 
% over the others by at least one order of magnitude in cross section. 
% Despite this, the other production modes are useful as they have specific 
% topologies which can be exploited in the selection of signal-like events. 
% The \ac{VBF} process is characterised by the two outgoing quarks which
% hadronise to form jets with high momentum, typically found in the forward
% detector region. Production via the associated production process consists of a vector
% boson or a pair of top quarks in the final state, yielding multi-lepton and 
% multi-jet final states with reduced backgrounds from other \ac{SM} processes.

\begin{figure}[htbp]
 \includegraphics[width=0.7\textwidth]{Chapter01/Images/Higgs_XS_8TeV_lx.pdf}
\caption[Cross sections for Higgs production processes at $\sqrt{s}=8\,\TeV$ for
a range of Higgs boson masses.]{Cross sections for Higgs production processes at
$\sqrt{s}=8\,\TeV$ for a range of Higgs boson masses $m_{\PH}$~\cite{ARTICLE:HandbookofLHCHiggsCrossSectionsHiggsProperties}. Across the
mass range the gluon-fusion mode dominates, followed by the vector boson fusion
and associated production modes. The widths of the lines represent the
theoretical uncertainties on the cross section calculation.}
\label{fig:SMHiggsXS}
\end{figure}

% The branching fractions to the different decay channels are shown in
% figure~\ref{fig:SMHiggsBRs} for a range of possible Higgs masses. At low Higgs
% boson masses, a variety of different decay channels are accessible. The decay
% into two photons occurs via a fermion and $\PW$ loop, and decays to $\PWp\PWm$, $\PZ\PZ$,
% $\Pqb\Paqb$ and $\Pgtp\Pgtm$ pairs are also possible. At higher Higgs masses
% above $130\,\GeV$, the decays into $\PWp\PWm$, $\PZ\PZ$ are more kinematically
% favourable and hence dominate. In the discovery of a Higgs boson, identication of
% any of these decay channels is important, but the most sensitive channels for
% discovery in the low mass region are $\Pphoton\Pphoton$ and $\PZ\PZ$ due to
% their clean signatures. However, upon discovery of a Higgs boson, decays into 
% fermions are particularly important
% to establish the Yukawa couplings. In fermionic decays, the decay into
% $\Pqb\Paqb$ dominates with a branching fraction of $\sim 57\%$ in this mass
% range. However, this final state is difficult to disentangle from the large QCD
% jet background at the LHC, meaning that the $\Pgtp\Pgtm$ final state has higher sensitivity.

\begin{figure}[htbp]
 \includegraphics[width=0.6\textwidth]{Chapter01/Images/Higgs_BR.pdf}
\caption[Higgs boson branching ratios in the SM for a range of Higgs boson
masses.]{Higgs boson branching ratios in the SM for a range of Higgs boson
masses $m_{\PH}$ \cite{ARTICLE:HandbookofLHCHiggsCrossSectionsHiggsProperties}. At high masses, above their
kinematic thresholds, the $\PW\PW$,
$\PZ\PZ$ and $\ttbar$ (shown in red) decay modes dominate. 
At lower masses a wide range of different final states is possible. 
The widths of the lines represent the
theoretical uncertainties on the branching ratio calculation.}
\label{fig:SMHiggsBRs}
\end{figure}

% 
% On 4 July 2012 the ATLAS and CMS Collaborations announced the discovery of a new
% boson with a mass around $125\,\GeV$~\cite{CMSobservation125,ATLASobservation125}. 
% Both experiments analysed approximately $5\,\invfb$ of data collected at 
% $\sqrt{s}=7\,\TeV$ and $5$--$6\,\invfb$ at $\sqrt{s}=8\,\TeV$. The discovery was
% made predominantly using the $\PZ\PZ$ and $\Pphoton\Pphoton$ decay modes with a
% combined excess of events in data yielding a $5\sigma$ deviation from the
% background-only expectation.
% 
% Run 1 of the LHC was completed in early 2013, increasing the dataset to
% $\approx20\,\invfb$ at $8\,\TeV$. The increased dataset allows access to the less
% sensitive decay modes. The most recent combination of CMS Higgs results yields a
% best fit mass of
% $125.02^{+0.26}_{-0.27}(\text{stat})^{+0.14}_{-0.15}(\text{syst})\,\GeV$ and a
% signal strength relative to the \ac{SM} expectation of
% $1.00\pm0.09(\text{stat})^{+0.08}_{-0.07}(\text{theo})\pm0.01(\text{syst})$,
% combining results from the $\Pphoton\Pphoton$, $\PWp\PWm$, $\PZ\PZ$,
% $\Pqb\Paqb$, $\Pgtp\Pgtm$ and $\APmuon\Pmuon$ final states, as well as 
% searches for the $\Pqt\Pqt\PH$ production mode and searches for an invisible
% Higgs \cite{CMScomb}. Within the combination is a study of the compatibility of
% the couplings in each decay mode with the \ac{SM}. 
% These channels all show consistency with the \ac{SM}
% predictions for a $125\,\GeV$ Higgs boson. Other analyses from ATLAS have
% studied the production rates and couplings in various channels~\cite{Aad:2014eva,Aad:2014lwa,Aad:2015vsa}, 
% and results from both ATLAS and CMS study the spin-parity quantum
% numbers~\cite{Chatrchyan:2013mxa,Chatrchyan:2013iaa,Aad:2013xqa} and limits on
% the decay width~\cite{Khachatryan:2014iha} and invisible branching
% fraction~\cite{Aad:2014iia,Chatrchyan:2014tja}. No significant deviations from \ac{SM}
% predictions have been found to date in any of these results.
% 
% Future studies will further test the compatibility of the observed boson with
% the \ac{SM} Higgs, including discovery and precision measurements in the
% channels which are less sensitive and require more data. 
% The most recent CMS results for the search for the Higgs boson in the $\HToTauTau$
% decay channel, corresponding to the complete Run 1 dataset and as included in
% the most recent combination, are reported in chapter~\ref{chap:htt-sm}.
