\chapter{Parked Data Analysis}
\label{CHAPTER:ParkedDataAnalysis}

\glsresetall % Resetting all acronyms

This chapter describes the analysis performed over the \gls{CMS} Run I parked data collected over 2012 and 2013. This data was collected and stored without reconstruction and only became fully available a few months after data taking was finished. The advantage of this dataset is the possibility to use lower threshold triggers which can collect more signal but also more backgrounds. To take full advantage of this data the analysis had to be redesigned and extended with new control regions.

%TODO: references? previous chapter?

\section{Event quality filters}

During data recording issues may happen with the detector or data acquisition which may render some of the events unusable. The groups responsible for each part of the detector and physics object check the data after it was taken and if they find such problems occurred. This groups produce software event filters analysts to be able to remove this problematic events. This issues cover from know detector problems to miss firing of calibration sequences or even failure to reconstruct physics objects.

The Jet-MET Particle Object Group (POG) recommends the usage of the following filters which are used in this analysis.
% \cite{CMS:JetMETPOG:MissingETOptionalFilters}

\begin{itemize}
  \item CSCTightHaloFilter
  \item HBHENoiseFilter
  \item EcalDeadCellTriggerPrimitiveFilter
  \item trackingFailureFilter
  \item eeBadScFilter
  \item ECAL Laser filter
  \item HCAL Laser filter
\end{itemize}

In turn the JetMET group recommend the usage of the following Tracking POG Filter:
%\cite{CMS:TrackingPOG:TrackingPOGFilters}


\begin{itemize}
  \item logErrorTooManyClusters
  \item manystripclus53X
  \item toomanystripclus53X
\end{itemize}

\section{Electrons}

\subsubsection{Veto electrons}

The "veto electrons" are defined by the base requirements of the cut based electron ID veto working point of the EGamma POG with some additional cuts on top. 
  
Requirements of the cut based electron ID veto working point:
 
\begin{table}[htp]
  
\begin{tabular}{|l|c|c|}
\hline
Variable & Barel & Endcap \\
\hline\hline
$| \Delta\eta(track,supercluster) |$ & $<0.007$ & $<0.1$  \\
$| \Delta\phi(track,supercluster) |$ & $<0.8  $ & $<0.7$  \\
$ \sigma(i\eta,i\eta)$               & $<0.01 $ & $<0.03$ \\
$H/E$                                & $<0.15 $ &       - \\
$|d_{0}(vertex)|$                    & $<0.04 $ & $<0.04$ \\
$|d_{Z}(vertex)|$                    & $<0.2  $ & $<0.2 $ \\
$\frac{PF_{isolation}}{p_{\perp}}$ for $ \Delta R_{cone}=0.3$  & $<0.15 $ & $<0.15$ \\
\hline
\end{tabular}
\caption{Details of the \gls{CMS} Electron-Gamma \gls{POG} recommendations for a \textit{veto electron}. Here barrel is defined as $ |\eta_{supercluster}|<=1.479 $ and endcap is $ 1.479 < |\eta_{supercluster}| < 2.5 $.} 
\end{table}

 
 
Additional requirements for this analysis
\begin{itemize}
  \item $ p_{\perp} > 10\,\GeV$
  \item $ |\eta| < 2.4 $
  \item $ Effective-Area-Corrected-Isolation < 0.15 $ (is stated in the note as additional requirement but does not look like it is)
  \item $d_{xy}<0.04\,\centi\meter$ (is stated in the note as additional requirement but does not look like it is)
  \item $d_{z} < 0.2\,\centi\meter$ (is stated in the note as additional requirement but does not look like it is)
\end{itemize}
 

\subsubsection{Tight electrons}
 
The "tight electrons" are defined by using the base requirements of the cut based electron ID tight working point (similar to 2011 very tight WP70) of the EGamma POG with some additional cuts on top.
  
Requirements of the cut based electron ID tight working point (similar to 2011 very tight WP70):

\begin{table}[htp]
  
\begin{tabular}{|l|c|c|}
\hline
Variable & Barel & Endcap \\
\hline\hline
$| \Delta\eta(track,supercluster) |$                           & $<0.004$ & $<0.005$ \\
$| \Delta\phi(track,supercluster) |$                           & $<0.3  $ & $<0.2  $ \\
$ \sigma(i\eta,i\eta)$                                         & $<0.01 $ & $<0.03 $ \\
$H/E$                                                          & $<0.12 $ & $<0.10 $ \\
$|d_{0}(vertex)|$                                              & $<0.02 $ & $<0.02 $ \\
$|d_{Z}(vertex)|$                                              & $<0.1  $ & $<0.1  $ \\
$|\frac{1}{E}-\frac{1}{p}| $                                   & $<0.05 $ & $<0.05 $ \\
$\frac{PF_{isolation}}{p_{\perp}}$ for $ \Delta R_{cone}=0.3$ and $\pt > 20 (p_{\perp} <= 20) $ & $<0.10 $ & $<0.10(0.07)$ \\
Conversion rejection: vertex fit probability                   & 1e-6 & 1e-6 \\
Conversion rejection: missing hits                             & $<=0$ & $<=0$ \\
\hline
\end{tabular}
\caption{Details of the \gls{CMS} Electron-Gamma \gls{POG} recommendations for a \textit{tight electron}. Here barrel is defined as $ |\eta_{supercluster}|<=1.479 $ and endcap is $ 1.479 < |\eta_{supercluster}| < 2.5 $.} 
\end{table}

Additional requirements for this analysis:
\begin{itemize}
  \item $ p_{\perp} > 20\,\GeV$
  \item $ |\eta| < 2.4 $
  \item $ Effective-Area-Corrected-Isolation < 0.10 $
  \item $d_{xy}<0.02\,\centi\meter$
  \item $d_{z} < 0.1\,\centi\meter$
\end{itemize}


\section{The Cross Check Analysis}

% The search for an invisible decay of a vector boson produced Higgs boson was first made public with CMS Physics Analysis Summary (PAS) HIG-13-013 which was further improved and combined with other Higgs boson production channel in the CMS paper HIG-13-30. Additional support material can be found at the CMS Analysis Notes (AN) AN-2012/403 \cite{CMS_AN_2013-403} and AN-2013/205.
% 
% During the 2012 data taking run two main streams of data were recorded. The main stream with an event rate of the order of 300 Hz to be promptly reconstructed and made available for analysis in a few days after being recorded, this dataset is referred to as the prompt data. The secondary stream with lower trigger thresholds with an event rate of the order of 1kHz which would only be reconstructed when the computing resources would be available outside of the data taking period, dataset is referred to as parked data. Our previous results
% were produced using the prompt data only and this work now extends on previous work by using the now available parked data. Since this dataset has been recorded with lower trigger thresholds the analysis was re-optimised to take advantage of this new available phase space. The details of the newly developed analysis can be found in CMS AN-14-243\cite{CMS_AN_2014-243}.
% 
% It is normally a requirement for many CMS publications to have a cross check analysis implemented independently from the main result in order to be able to ensure accuracy of the final results due to possible errors with the software implementation. For this purpose the previous prompt data VBF Higgs to Invisible results and publication were produced by two different and independent code frameworks and before publication a good level of synchronization were obtained. Due to lack of man power and time it was decide for the 2012 parked data analysis to only proceed with a single framework. At a later stage of the analysis it was thought that at least some level of cross check would be a good measure to limit the possibility of implementation errors and to allow extra confidence on the final results.
% 
% This cross check analysis starts from the same ntuples produced by the main analysis which were produced over all the relevant datasets and are recorded with data formats also used by other analysis at Imperial College London, e.g. both the SM and MSSM Higgs to $\tau\bar{\tau}$, the Higgs to $\tau\bar{\tau}b\bar{b}$ and prompt Higgs to invisible analyses. No cuts are applied at ntuple production except the official CMS selection for good usable data using the appropriate golden JSON file.
% 
% From those initial ntuple an independent code framework was developed in order to replicate all relevant numbers and plots from the main analysis.

% For this analysis we use two categories of electrons "veto electrons" and "tight electrons". Both this categories of particles are based on standard EGamma POG cut based object definition which can be found at (TODO:CITATION). Additionally, we require some additional cuts to each category (TODO:WHY).
