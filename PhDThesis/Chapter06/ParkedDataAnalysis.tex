\chapter{Parked Data Analysis}
\label{CHAPTER:ParkedDataAnalysis}

\glsresetall % Resetting all acronyms

This chapter describes the analysis performed over the \gls{CMS} Run I parked data collected over 2012 and 2013. This data was collected and stored without reconstruction and only became fully available a few months after data taking was finished. The advantage of this dataset is the possibility to use lower threshold triggers which can collect more signal but also more backgrounds. To take full advantage of this data the analysis had to be redesigned and extended with new control regions.

%TODO: references? previous chapter?

%%%%%%%%%%%%%%%%%%%%%%%%%%%%%%%%%%%%%%%%%%%%%%%%%%%%%%%%%%%%%%%%%%%%%%%%%%%%%%%%%%%%
%%% SECTION
%%%%%%%%%%%%%%%%%%%%%%%%%%%%%%%%%%%%%%%%%%%%%%%%%%%%%%%%%%%%%%%%%%%%%%%%%%%%%%%%%%%%
\section{Data and MC samples}

%%%%%%%%%%%%%%%%%%%%%%%%%%%%%%%%%%%%%%%%%%%%%%%%%%%%%%%%%%%%%%%%%%%%%%%%%%%%%%%%%%%%
%%% SUBSECTION
%%%%%%%%%%%%%%%%%%%%%%%%%%%%%%%%%%%%%%%%%%%%%%%%%%%%%%%%%%%%%%%%%%%%%%%%%%%%%%%%%%%%
\subsection{Data}

In this analysis we used the full certified data with collisions at $\sqrt{s}=8\,\TeV$ from 2012-13 data acquisition (Run I), using golden JSON file \verb|Cert_190456-208686_8TeV_22Jan2013ReReco_Collisions12_JSON.txt| was used in this analysis. It amounts to an integrated luminosity of $19.2 \pm 0.5 \,\femto\barn^{-1}$. A summary of the dataset names and their integrated luminosity can be found in table \ref{TABLE:ParkedData_Data_RunI_IntegratedLuminosity}.

\begin{table}[!htb]
\centering
\begin{tabular}{|l|c|}
\hline
Dataset & $\int{Luminosity}$ $[pb^{-1}]$ \\
\hline \hline
/MET/Run2012A-22Jan2013-v1/AOD & 889 \\
/VBF1Parked/Run2012B-22Jan2013-v1/AOD & 3871 \\
/VBF1Parked/Run2012C-22Jan2013-v1/AOD & 7152 \\
/VBF1Parked/Run2012D-22Jan2013-v1/AOD & 7317 \\
\hline
Total analysed & 19229 \\
\hline \hline
Total certified luminosity & 19789 \\
\hline
\end{tabular}
\caption{Relevant parked datasets from Run I and their total analysed integrated luminosity. Total analysed and certified also showed.}
\label{TABLE:ParkedData_Data_RunI_IntegratedLuminosity}
\end{table}


The difference between certified and analysed datasets is due to out analysis trigger not being active for the first few runs of of Run 2012B. 

%%%%%%%%%%%%%%%%%%%%%%%%%%%%%%%%%%%%%%%%%%%%%%%%%%%%%%%%%%%%%%%%%%%%%%%%%%%%%%%%%%%%
%%% SECTION
%%%%%%%%%%%%%%%%%%%%%%%%%%%%%%%%%%%%%%%%%%%%%%%%%%%%%%%%%%%%%%%%%%%%%%%%%%%%%%%%%%%%
\section{Event quality filters}

During data recording issues may happen with the detector or data acquisition which may render some of the events unusable. The groups responsible for each part of the detector and physics object check the data after it was taken and if they find such problems occurred. This groups produce software event filters analysts to be able to remove this problematic events. This issues cover from know detector problems to miss firing of calibration sequences or even failure to reconstruct physics objects.

The Jet-MET Particle Object Group (POG) recommends the usage of the following filters which are used in this analysis.
% \cite{CMS:JetMETPOG:MissingETOptionalFilters}

\begin{itemize}
  \item CSCTightHaloFilter
  \item HBHENoiseFilter
  \item EcalDeadCellTriggerPrimitiveFilter
  \item trackingFailureFilter
  \item eeBadScFilter
  \item ECAL Laser filter
  \item HCAL Laser filter
\end{itemize}

In turn the JetMET group recommend the usage of the following Tracking POG Filter:
%\cite{CMS:TrackingPOG:TrackingPOGFilters}


\begin{itemize}
  \item logErrorTooManyClusters
  \item manystripclus53X
  \item toomanystripclus53X
\end{itemize}

%%%%%%%%%%%%%%%%%%%%%%%%%%%%%%%%%%%%%%%%%%%%%%%%%%%%%%%%%%%%%%%%%%%%%%%%%%%%%%%%%%%%
%%% SECTION
%%%%%%%%%%%%%%%%%%%%%%%%%%%%%%%%%%%%%%%%%%%%%%%%%%%%%%%%%%%%%%%%%%%%%%%%%%%%%%%%%%%%
\section{Electrons}

%%%%%%%%%%%%%%%%%%%%%%%%%%%%%%%%%%%%%%%%%%%%%%%%%%%%%%%%%%%%%%%%%%%%%%%%%%%%%%%%%%%%
%%% SUBSECTION
%%%%%%%%%%%%%%%%%%%%%%%%%%%%%%%%%%%%%%%%%%%%%%%%%%%%%%%%%%%%%%%%%%%%%%%%%%%%%%%%%%%%
\subsection{Veto electrons}

The "veto electrons" are defined by the base requirements of the cut based electron ID veto working point of the EGamma POG with some additional cuts on top. 
  
Requirements of the cut based electron ID veto working point:
 
\begin{table}[htp]
  
\begin{tabular}{|l|c|c|}
\hline
Variable & Barel & Endcap \\
\hline\hline
$| \Delta\eta(track,supercluster) |$ & $<0.007$ & $<0.1$  \\
$| \Delta\phi(track,supercluster) |$ & $<0.8  $ & $<0.7$  \\
$ \sigma(i\eta,i\eta)$               & $<0.01 $ & $<0.03$ \\
$H/E$                                & $<0.15 $ &       - \\
$|d_{0}(vertex)|$                    & $<0.04 $ & $<0.04$ \\
$|d_{Z}(vertex)|$                    & $<0.2  $ & $<0.2 $ \\
$\frac{PF_{isolation}}{p_{\perp}}$ for $ \Delta R_{cone}=0.3$  & $<0.15 $ & $<0.15$ \\
\hline
\end{tabular}
\caption{Details of the \gls{CMS} Electron-Gamma \gls{POG} recommendations for a \textit{veto electron}. Here barrel is defined as $ |\eta_{supercluster}|<=1.479 $ and endcap is $ 1.479 < |\eta_{supercluster}| < 2.5 $.} 
\end{table}

 
 
Additional requirements for this analysis
\begin{itemize}
  \item $ p_{\perp} > 10\,\GeV$
  \item $ |\eta| < 2.4 $
\end{itemize}
 
%%%%%%%%%%%%%%%%%%%%%%%%%%%%%%%%%%%%%%%%%%%%%%%%%%%%%%%%%%%%%%%%%%%%%%%%%%%%%%%%%%%%
%%% SUBSECTION
%%%%%%%%%%%%%%%%%%%%%%%%%%%%%%%%%%%%%%%%%%%%%%%%%%%%%%%%%%%%%%%%%%%%%%%%%%%%%%%%%%%%
\subsection{Tight electrons}

% More information at: 
%  * https://twiki.cern.ch/twiki/bin/viewauth/CMS/EgammaIDRecipes
%  * https://twiki.cern.ch/twiki/bin/view/CMS/EgammaCutBasedIdentification
%  * https://twiki.cern.ch/twiki/bin/view/CMS/EgammaEARhoCorrection

The "tight electrons" are defined by using the base requirements of the cut based electron ID tight working point (similar to 2011 very tight WP70) of the EGamma POG with some additional cuts on top.
  
Requirements of the cut based electron ID tight working point (similar to 2011 very tight WP70) and can be found in table \ref{TABLE:ParkedDataAnalysis_TightElectronID}

\input{Chapter06/Tables/table_TightElectronID}

Additional requirements for this analysis:
\begin{itemize}
  \item $ p_{\perp} > 20\,\GeV$
  \item $ |\eta| < 2.4 $
\end{itemize}

%%%%%%%%%%%%%%%%%%%%%%%%%%%%%%%%%%%%%%%%%%%%%%%%%%%%%%%%%%%%%%%%%%%%%%%%%%%%%%%%%%%%
%%% SECTION
%%%%%%%%%%%%%%%%%%%%%%%%%%%%%%%%%%%%%%%%%%%%%%%%%%%%%%%%%%%%%%%%%%%%%%%%%%%%%%%%%%%%
\section{Muons}

%%%%%%%%%%%%%%%%%%%%%%%%%%%%%%%%%%%%%%%%%%%%%%%%%%%%%%%%%%%%%%%%%%%%%%%%%%%%%%%%%%%%
%%% SUBSECTION
%%%%%%%%%%%%%%%%%%%%%%%%%%%%%%%%%%%%%%%%%%%%%%%%%%%%%%%%%%%%%%%%%%%%%%%%%%%%%%%%%%%%
\subsection{Loose Muons}

% From: https://twiki.cern.ch/twiki/bin/view/CMSPublic/SWGuideMuonId
%
% bool muon::isLooseMuon (const reco::Muon & recoMu);
% Particle-Flow muon id     & recoMu.isPFMuon()                               & Can be complemented by muon quality cuts similar to those used in the Tight Muon selection.
% Is Global OR Tracker Muon & recoMu.isGlobalMuon() || recoMu.isTrackerMuon() & Avoid using muons which are only Standalone Muons. (~0.01% of PF muons) 
% REFERENCE: MUO-10-004

Requirements:
\begin{itemize}
  \item $\pt>10\,\GeV$
  \item $|\eta|<2.1$
  \item Relative Combined Isolation $<0.2$
\end{itemize}

POG Loose ID:
\begin{itemize}
  \item Particle-Flow muon ID
  \item Global or Tracker Muon
\end{itemize}


%%%%%%%%%%%%%%%%%%%%%%%%%%%%%%%%%%%%%%%%%%%%%%%%%%%%%%%%%%%%%%%%%%%%%%%%%%%%%%%%%%%%
%%% SUBSECTION
%%%%%%%%%%%%%%%%%%%%%%%%%%%%%%%%%%%%%%%%%%%%%%%%%%%%%%%%%%%%%%%%%%%%%%%%%%%%%%%%%%%%
\subsection{Tight Muons}

Requirements:
\begin{itemize}
  \item $\pt>20\,\GeV$
  \item $|\eta|<2.1$
  \item Relative Combined Isolation $<0.12$
  \item $d_{xy}<0.045$
  \item $d_z < 0.2$
\end{itemize}

POG Tight ID:
\begin{itemize}
  \item Global Muon
  \item Particle-Flow muon id 
  \item $\chi^2/ndof < 10$
  \item At least one muon chamber hit included in the global-muon track fit
  \item Muon segments in at least two muon stations
  \item $d_{xy} < 2$ mm w.r.t. the primary vertex
  \item $d_z < 5$ mm
  \item Number of pixel hits $> 0$ 
  \item Number of tracker layers with hits $>5$
\end{itemize}


% bool muon::isTightMuon(const reco::Muon & recoMu, const reco::Vertex & vtx);
% The candidate is reconstructed as a Global Muon & recoMu.isGlobalMuon()    
% Particle-Flow muon id                           & recoMu.isPFMuon()                            & the exclusive effect of this requirement is very small, i.e. PFMuon is keeping almost all Tight Muons without this cut
% χ2/ndof of the global-muon track fit < 10       & recoMu.globalTrack()->normalizedChi2() < 10. &   To suppress hadronic punch-through and muons from decays in flight (see CMS AN 2008/098). This cut might need to be re-tuned due to the change to fully segment based global fit in 50X releases and later. It will need a retuning when muon APEs will be activated in the global track fit.
% At least one muon chamber hit included in the global-muon track fit     recoMu.globalTrack()->hitPattern().numberOfValidMuonHits() > 0  To suppress hadronic punch-through and muons from decays in flight.
% Muon segments in at least two muon stations
% This implies that the muon is also an arbitrated tracker muon, see SWGuideTrackerMuons  recoMu.numberOfMatchedStations() > 1    To suppress punch-through and accidental track-to-segment matches.
% Also makes selection consistent with the logic of the muon trigger, which requires segments in at least two muon stations to obtain a meaningful estimate of the muon pT.
% Its tracker track has transverse impact parameter dxy < 2 mm w.r.t. the primary vertex  fabs(recoMu.muonBestTrack()->dxy(vertex->position())) < 0.2
% Or dB() < 0.2 on pat::Muon [1]  To suppress cosmic muons and further suppress muons from decays in flight (see CMS AN 2008/098).
% The 2 mm cut preserves efficiency for muons from decays of b and c hadrons. It is a loose cut and can be tightened further with minimal loss of efficiency for prompt muons if background from cosmic muons is an issue. Another way to obtain a better cosmic-ray suppression is to complement the dxy cut with a cut on the opening angle α or use a dedicated cosmic-id algorithm (see Section 7.1 of MUO-10-004). innerTrack() is also supported for dxy cut, as the performance of the two is very close.
% The longitudinal distance of the tracker track wrt. the primary vertex is dz < 5 mm     fabs(recoMu.muonBestTrack()->dz(vertex->position())) < 0.5      Loose cut to further suppress cosmic muons, muons from decays in flight and tracks from PU. innerTrack() is also supported for dz cut, as the performance of the two is very close.
% Number of pixel hits > 0        recoMu.innerTrack()->hitPattern().numberOfValidPixelHits() > 0  To further suppress muons from decays in flight.
% Cut on number of tracker layers with hits >5    recoMu.innerTrack()->hitPattern().trackerLayersWithMeasurement() > 5    To guarantee a good pT measurement, for which some minimal number of measurement points in the tracker is needed.
% Also suppresses muons from decays in flight. 
%
% REFERENCE: MUO-10-002 and MUO-10-004


\section{Taus}

Requirements
\begin{itemize}
  \item $\pt > 20 \,\GeV$
  \item $|\eta|<2.3$
  \item $d_z<0.2 \,\centi\meter$
\end{itemize}

Descriminants:
\begin{itemize}
  \item decayModeFinding
  \item byTightCombinedIsolationDeltaBetaCorr3Hits
  \item againstMuonTight
  \item againstElectronTight
\end{itemize}

%%%%%%%%%%%%%%%%%%%%%%%%%%%%%%%%%%%%%%%%%%%%%%%%%%%%%%%%%%%%%%%%%%%%%%%%%%%%%%%%%%%%
%%% SECTION
%%%%%%%%%%%%%%%%%%%%%%%%%%%%%%%%%%%%%%%%%%%%%%%%%%%%%%%%%%%%%%%%%%%%%%%%%%%%%%%%%%%%
\section{The Cross Check Analysis}

% The search for an invisible decay of a vector boson produced Higgs boson was first made public with CMS Physics Analysis Summary (PAS) HIG-13-013 which was further improved and combined with other Higgs boson production channel in the CMS paper HIG-13-30. Additional support material can be found at the CMS Analysis Notes (AN) AN-2012/403 \cite{CMS_AN_2013-403} and AN-2013/205.
% 
% During the 2012 data taking run two main streams of data were recorded. The main stream with an event rate of the order of 300 Hz to be promptly reconstructed and made available for analysis in a few days after being recorded, this dataset is referred to as the prompt data. The secondary stream with lower trigger thresholds with an event rate of the order of 1kHz which would only be reconstructed when the computing resources would be available outside of the data taking period, dataset is referred to as parked data. Our previous results
% were produced using the prompt data only and this work now extends on previous work by using the now available parked data. Since this dataset has been recorded with lower trigger thresholds the analysis was re-optimised to take advantage of this new available phase space. The details of the newly developed analysis can be found in CMS AN-14-243\cite{CMS_AN_2014-243}.
% 
% It is normally a requirement for many CMS publications to have a cross check analysis implemented independently from the main result in order to be able to ensure accuracy of the final results due to possible errors with the software implementation. For this purpose the previous prompt data VBF Higgs to Invisible results and publication were produced by two different and independent code frameworks and before publication a good level of synchronization were obtained. Due to lack of man power and time it was decide for the 2012 parked data analysis to only proceed with a single framework. At a later stage of the analysis it was thought that at least some level of cross check would be a good measure to limit the possibility of implementation errors and to allow extra confidence on the final results.
% 
% This cross check analysis starts from the same ntuples produced by the main analysis which were produced over all the relevant datasets and are recorded with data formats also used by other analysis at Imperial College London, e.g. both the SM and MSSM Higgs to $\tau\bar{\tau}$, the Higgs to $\tau\bar{\tau}b\bar{b}$ and prompt Higgs to invisible analyses. No cuts are applied at ntuple production except the official CMS selection for good usable data using the appropriate golden JSON file.
% 
% From those initial ntuple an independent code framework was developed in order to replicate all relevant numbers and plots from the main analysis.

% For this analysis we use two categories of electrons "veto electrons" and "tight electrons". Both this categories of particles are based on standard EGamma POG cut based object definition which can be found at (TODO:CITATION). Additionally, we require some additional cuts to each category (TODO:WHY).
