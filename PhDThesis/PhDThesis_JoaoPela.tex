\documentclass[hyperpdf]{hepthesis}

%\documentclass[a4paper,10pt]{book}
\usepackage[utf8]{inputenc}
%\usepackage{fullpage}
%\usepackage[margin=2cm]{geometry} % Defining geometry of the usable text space

%% Using Babel allows other languages to be used and mixed-in easily
%\usepackage[ngerman,english]{babel}
% \usepackage[english]{babel}
% \selectlanguage{english}

%% Citation system tweaks
\usepackage{cite}

% \usepackage{amsmath}
% \usepackage{amsthm}
% \usepackage{amsfonts}
% \usepackage{amssymb}
% \usepackage{graphicx}
\usepackage{color}                 % For color (highlights,...) 
% \usepackage{caption}               % ?????
% \usepackage{soul}                  % For text highlights
% \usepackage{threeparttable}        % Better tables
\usepackage[xindy,toc]{glossaries} %(recommended) for the indexing phase, as opposed to makeindex (the default)
% \usepackage{hyperref}              % Need after glossaries (allows links)

\title{Search for Higgs Decay to Dark Matter and Trigger Studies} 
\author{João Carlos Arnauth Pela}
% \email[Contact email: ]{joaopela@gmail.com}
% \affiliation{Imperial College London}
% \keywords{Particle Physics}
\date{08 January 2014}

\newacronym{BSM}{BSM}{Beyond the Standard Model}
\newacronym{CMS}{CMS}{Compact Muon Solenoid}
\newacronym{DAQ}{DAQ}{Data Acquisition}
\newacronym{DQM}{DQM}{Data Quality Monitoring}
\newacronym{FCT}{FCT}{Fundação para a Ciência e a Tecnologia}
\newacronym{MET}{MET}{Missing Transverse Energy}
\newacronym{L1T}{L1T}{Level 1 Trigger}
\newacronym{LHC}{LHC}{Large Hadron Collider}
\newacronym{SM} {SM} {Standard Model}
\makeglossaries

%% Doc-specific PDF metadata
 \makeatletter
 \@ifpackageloaded{hyperref}{
 \hypersetup{
   pdftitle    = {Search for Higgs Decay to Dark Matter and Trigger Studies},
   pdfsubject  = {João Pela's PhD thesis},
   pdfkeywords = {LHC, CMS, Higgs, Dark Matter, Trigger},
   pdfauthor   = {\textcopyright\ João Pela}
 }}{}
 \makeatother

\begin{document}

\begin{frontmatter}

\titlepage[of the High Energy Physics Group]{%
  A dissertation submitted to the Imperial College London\\ for the degree of Doctor of Philosophy}
  
\begin{abstract}
Here the abstract of the thesis
\end{abstract}

\begin{declaration}
Is this needed?
\end{declaration}

\begin{acknowledgements}
I would like to thank \gls{FCT} for their support of my PhD work through their grant SFRH/BD/77979/2011. 

\colorbox{red}{TODO: Include FCT symbols}

\end{acknowledgements}

\begin{preface}
Thesis structure and so on...
\end{preface}

\dedication{To all who study...}

%% ToC
\tableofcontents

  
\end{frontmatter}

\begin{mainmatter}

  \chapter{Theory}
%NOTE: Maybe would be good to make a more descriptive title

\colorbox{red}{TODO: Something}

\section{Standard Model of Particle Physics}

\colorbox{red}{TODO: Very brief summary of the Standard Model.}

% The \gls{sm} of particle physics is the currently accepted model for describing the physics of elementary particles.

\section{Higgs Mechanism}

Summary of the Higgs Mechanism. Should include
\begin{itemize}
 \item Motivations 
 \item Explanation of the mechanism itself
 \item Consequences 
 \item Possible decays
\end{itemize}

\section{Higgs Invisible decays}

\colorbox{red}{TODO: Explain what are SM Higgs invisible decays. Go over the possibility of BSM invisible decays.}

  \chapter{Experimental Apparatus}
\label{CHAPTER:ExperimentalApparatus}




% To cite
The LHC Machine article \cite{ARTICLE:LHC Machine}.
%

\section{The Large Hadron Collider}
\label{SECTION:ExperimentalApparatus_LHC}

\colorbox{red}{
\begin{minipage}{\linewidth}
TODO: 

\begin{itemize}
  \item CERN basics location etc
  \item Basics of machine and operation
  \item New to include Instantaneous luminosity equation
\end{itemize}

\end{minipage}
}

The Large Hadron Collider \cite{ARTICLE:LHCMachine} is a 27 km synchrotron machine located in Geneva Switzerland.

Luminosity Equation

\begin{equation}
L=\frac{N_{b}^{2}n_{b}f_{\text{rev}}\gamma}{4\pi\epsilon_{n}\beta^{*}}F,
\end{equation}

\section{The Compact Muon Solenoid Experiment}
\label{SECTION:ExperimentalApparatus_CMS}





\subsection{Tracker}
\label{SUBSECTION:ExperimentalApparatus_CMS_Tracker}

\subsection{Electromagnetic Calorimeter}
\label{SUBSECTION:ExperimentalApparatus_CMS_ECAL}

\subsection{Hadronic Calorimeter}
\label{SUBSECTION:ExperimentalApparatus_CMS_HCAL}

\subsection{Solenoid Magnet}
\label{SUBSECTION:ExperimentalApparatus_CMS_Magnet}

\subsection{Muon System}
\label{SUBSECTION:ExperimentalApparatus_CMS_Mouns}

\subsection{Data Acquisition System}
\label{SUBSECTION:ExperimentalApparatus_CMS_DAQ}

\subsection{Trigger System}
\label{SUBSECTION:ExperimentalApparatus_CMS_Trigger}

The upgrade tdr \cite{CMSL1UpgradeTDR}.

\subsection{Computing}
\label{SUBSECTION:ExperimentalApparatus_CMS_Computing}

% \gls{dqm} and \gls{dqm} 

\subsection{Run II Updgrades}
\label{SUBSECTION:ExperimentalApparatus_CMS_RUNII}


  \chapter{Physics Objects and Monte Carlo simulation}

\section{Physics objects definition}

\subsection{Electron}

\subsection{Muon}

\subsection{Tau}

\subsection{Jets}

\subsection{Missing Transverse Energy}

\section{Monte Carlo simulation}



% \clearpage
% \section{Introduction}

The search for an invisible decay of a vector boson produced Higgs boson was first made public with CMS Physics Analysis Summary (PAS) HIG-13-013 which was further improved and combined with other Higgs boson production channel in the CMS paper HIG-13-30. Additional support material can be found at the CMS Analysis Notes (AN) AN-2012/403\cite{CMS_AN_2013-403} and AN-2013/205.

During the 2012 data taking run two main streams of data were recorded. The main stream with an event rate of the order of 300 Hz to be promptly reconstructed and made available for analysis in a few days after being recorded, this dataset is referred to as the prompt data. The secondary stream with lower trigger thresholds with an event rate of the order of 1kHz which would only be reconstructed when the computing resources would be available outside of the data taking period, dataset is referred to as parked data. Our previous results
were produced using the prompt data only and this work now extends on previous work by using the now available parked data. Since this dataset has been recorded with lower trigger thresholds the analysis was re-optimised to take advantage of this new available phase space. The details of the newly developed analysis can be found in CMS AN-14-243\cite{CMS_AN_2014-243}.

It is normally a requirement for many CMS publications to have a cross check analysis implemented independently from the main result in order to be able to ensure accuracy of the final results due to possible errors with the software implementation. For this purpose the previous prompt data VBF Higgs to Invisible results and publication were produced by two different and independent code frameworks and before publication a good level of synchronization were obtained. Due to lack of man power and time it was decide for the 2012 parked data analysis to only proceed with a single framework. At a later stage of the analysis it was thought that at least some level of cross check would be a good measure to limit the possibility of implementation errors and to allow extra confidence on the final results.

This cross check analysis starts from the same ntuples produced by the main analysis which were produced over all the relevant datasets and are recorded with data formats also used by other analysis at Imperial College London, e.g. both the SM and MSSM Higgs to $\tau\bar{\tau}$, the Higgs to $\tau\bar{\tau}b\bar{b}$ and prompt Higgs to invisible analyses. No cuts are applied at ntuple production except the official CMS selection for good usable data using the appropriate golden JSON file.

From those initial ntuple an independent code framework was developed in order to replicate all relevant numbers and plots from the main analysis.

% \section{Data samples}

For this analysis we used the proton-proton collision parked data collected by the CMS experiment during 2012-13. An analysis purposely constructed trigger was used to collect this data.

\begin{table}[htp]
\centering

\begin{tabular}{|l|c|}
\hline
Dataset & $\int{Luminosity}$ $[pb^{-1}]$ \\
\hline \hline
/MET/Run2012A-22Jan2013-v1/AOD        & 889 \\
/VBF1Parked/Run2012B-22Jan2013-v1/AOD & 3871 \\
/VBF1Parked/Run2012C-22Jan2013-v1/AOD & 7152 \\
/VBF1Parked/Run2012D-22Jan2013-v1/AOD & 7317 \\
\hline
Total analysed & 19229 \\
\hline \hline
Total certified luminosity & 19789 \\
\hline
\end{tabular}

\caption{Used datasets and their respective integrated luminosities after applying certified for physics filtering. Each dataset corresponds to a recording period/era of the 2012-13 data acquisition run.}
\end{table}

% Review text abouve

% To include
% * Explain breifly the validadion process (JSON)
% * Table with data included


% \section{Event Filters and object definitions}

%STATUS: Finished
%TODO: 
% * Could not find any reference to this recommendations!
\subsection{Vertex}

The interaction point is normally assumed to be the reconstructed  primary vertex, defined as the vertex with highest sum of associated tracks \pt squared, or if that cannot be determined the beam spot position is assumed. Knowing precisely the interaction point will allow to determine object quantities relative to it which allow for better object identification and pile-up control. At least one good vertex implicitly required by tracking failure event quality filter. And Additionally we require explicitly that a good vertex is reconstructed with the following characteristics. 

\begin{itemize}
  \item NOT(isFake): We require a real reconstructed vertex from tracks, not the beam spot.
  \item Number of degrees of freedom: $n_{dof}>4$
  \item Longitudinal distance: $|z|<=24$
  \item Radial distance to beam line: $d_{xy}<2$
\end{itemize}

%STATUS: Finished
%TODO: 
% * Add citation for the Jet-MET POG for this filter usage (include data and page version)
% * QUESTION: Should I include description of what each filter does?
\subsection{Event quality filters}

During data recording issues may happen with the detector or acquisition which may render some of the events unusable. The groups responsible for each part of the detector and physics object check the data after and was taken and if they find such problems software event filters are made available for analysis to be able to remove this problematic events. This event filters will address issues like detector know problems, miss firing of calibration sequences or failure to reconstruct physics objects. The Jet-MET Particle Object Group (POG) recommends the usage of the following filters which are used in this analysis\cite{CMS:JetMETPOG:MissingETOptionalFilters}.

\begin{itemize}
  \item CSCTightHaloFilter
  \item HBHENoiseFilter
  \item EcalDeadCellTriggerPrimitiveFilter
  \item trackingFailureFilter
  \item eeBadScFilter
  \item ECAL Laser filter (via event list)
  \item HCAL Laser filter (via event list)
\end{itemize}

In turn the JetMET group recommend the usage of the following Tracking POG Filter\cite{CMS:TrackingPOG:TrackingPOGFilters}:

\begin{itemize}
  \item  logErrorTooManyClusters
  \item  manystripclus53X
  \item  toomanystripclus53X
\end{itemize}

The event rejection efficiencies from the all filters except ECAL and HCAL laser event filters can be found at table \ref{table_EventQualityFilterEff}. This values are measured the vertex requirements. 

\begin{table}[!htp]
\centering

\begin{tabular}{|l||c|c|c|c|}
\hline
Filter                  & Prompt A & Parked B & Parked C & Parked D \\
\hline \hline
ECAL Laser Filter       & 0.928521 & 0.008659 & 0.000000 & 0.000000 \\
HCAL Laser Filter       & 0.007258 & 0.000000 & 0.000270 & 0.000000 \\
\hline
ECAL+ HCAL Laser Filter & 0.935704 & 0.008659 & 0.000270 & 0.000000 \\
\hline
\end{tabular}

\caption{Event rejection efficiency for the ECAL and HCAL Laser Filters. This events have to be removed due to the untimely firing of the calibration laser for this systems.}
\label{table_CaloLaserFilterEff}
\end{table}


The values for the ECAL and HCAL laser event filters are present in table \ref{table_CaloLaserFilterEff}. This value are calculated after vertex requirement and the filters in table \ref{table_EventQualityFilterEff}. 

\begin{table}[htp]
\centering

\begin{tabular}{|l||c|c|c|c|}
\hline
Filter                             & Prompt A  & Parked B & Parked C & Parked D \\
\hline \hline
HBHENoiseFilter                    & 22.900905 & 0.190670 & 0.187739 & 0.170753 \\
EcalDeadCellTriggerPrimitiveFilter &  0.375381 & 0.009300 & 0.010206 & 0.012526 \\
eeBadScFilter                      &  0.007852 & 0.000001 & 0.000000 & 0.000009 \\
trackingFailureFilter              &  3.073876 & 0.000328 & 0.007464 & 0.000290 \\
manystripclus53X                   &  0.001829 & 0.001319 & 0.002335 & 0.001327 \\
toomanystripclus53X                &  0.000484 & 0.001149 & 0.002006 & 0.001173 \\
logErrorTooManyClusters            &  0.000027 & 0.000009 & 0.000021 & 0.000016 \\
CSCTightHaloFilter                 & 10.263068 & 0.398497 & 0.402936 & 0.508025 \\
\hline
Total                              & 28.501208 & 0.598417 & 0.601999 & 0.689380 \\
\hline
\end{tabular}

\caption{Percentage of events in data failing each of the event quality filters after requesting on good vertex. For the parked analysis we use prompt A and parked B, C and D datasets.}
\label{table_EventQualityFilterEff}
\end{table}

It is observed that the HBHENoiseFilter vetos around 22\% of the event for prompt run A but this behaviour is not observed in any of the parked datasets. This specific filter removed events with noise in the HCAL. An example of noise would be a ``hot tower'' reading very high energy values, some times for several events. Since this energy will not be balance in each event this will greatly increase \MET. On the prompt dataset there are \MET only triggers which were fired in such events which then get removed by this filter, those triggers are not present in parked dataset. This can clearly be seen if we apply first our trigger selection (Jets+\MET) and then recalculate the event rejection efficiencies, this values can be found in table \ref{table_EventQualityFilterEff_PostTrigger}.

\begin{table}[htp]
\centering

\begin{tabular}{|l||c|c|c|c|}
\hline
Filter                             & Prompt A & Parked B & Parked C & Parked D \\
\hline \hline
HBHENoiseFilter                    & 1.264571 & 0.186895 & 0.178422 & 0.154080 \\
EcalDeadCellTriggerPrimitiveFilter & 0.572468 & 0.010081 & 0.010338 & 0.010725 \\
eeBadScFilter                      & 0.000989 & 0.000001 & 0.000000 & 0.000000 \\
trackingFailureFilter              & 0.062289 & 0.000401 & 0.009729 & 0.000231 \\
manystripclus53X                   & 0.002966 & 0.001264 & 0.002304 & 0.001255 \\
toomanystripclus53X                & 0.002966 & 0.001093 & 0.001955 & 0.001103 \\
logErrorTooManyClusters            & 0.000000 & 0.000000 & 0.000000 & 0.000000 \\
CSCTightHaloFilter                 & 0.400431 & 0.397531 & 0.403156 & 0.506917 \\
\hline                             
Total                              & 2.201877 & 0.594332 & 0.592977 & 0.671026 \\
\hline
\end{tabular}

\caption{Percentage of events in data failing each of the event quality filters after requesting on good vertex and trigger conditions. For the parked analysis we use prompt A and parked B, C and D datasets.}
\label{table_EventQualityFilterEff_PostTrigger}
\end{table}

Both tables \ref{table_EventQualityFilterEff} and \ref{table_CaloLaserFilterEff} can be directly compared with the values produced by the main analysis and presented in AN-14-243\cite{CMS_AN_2014-243}. No differences are observed.

%STATUS: In progress
%TODO:  
\subsection{Jets}

In this analysis we use particle flow  jets clustered with the $anti-k_{T}$ algorithm with a cone size of 0.5. 

The correction L1FastJet, L2Relative and L3Absolute are applied to both data and monte carlo (MC) and additionally we apply L2L3Residual to data.

The to correct Jet Energy Scale we apply global tag FT53\_V21A\_AN6::All for data and START53\_V27::All for monte carlo.

The jets with this characteristics and corrections are available in the ntuples produced by the main analysis as the "standard" jets collection. We further require that our selected jets pass PFJet ID and pileup ID. And we clean the jet collection by removing all jets that closer then $\Delta R < 0.5$ to any veto electron or loose muon (relevant for control regions).

%Out of this jets we select the ones passing loose PFJet ID and pileup ID. We also clean the jet collection by removing all jets that closer then %$\Delta R < 0.5$% to any veto electron or loose muon (relevant for control regions).
 
%% FOR MY PHD THESIS 
%% Details on 
%  
% As a selection condition for out jets we use the PFJet ID described at [[CMS/JetID.Recommendations_for_8_TeV_data_a][JetID page]].
% 
% The jet requirements are:
%    * Neutral Hadron Fraction %$ <0.99 $%
%    * Neutral EM Fraction %$ <0.99 $%
%    * Number of Constituents  %$ >1 $%
% Additionally for %$|\eta| < 2.4$% we require:
%    * Charged Hadron Fraction %$  >0 $%
%    * Charged Multiplicity %$ >0 $%
%    * Charged EM Fraction %$ 0.99 $%
% 
% ---+++ Pileup ID 
% 
% Using the prescription at the [[CMS.PileupJetID][PileupJetID page]] we reject pileup jets cutting at loose working point of a BDT with Dec2012 weights and package recojets/jetproducers with version METPU_5_3_X_v4. The computation of the MVA value per jet is made when the ntuples are produced and this values are stored within each jet. We cut on the BDT score categorizing in %$p_{\perp}$% and %$|\eta|$% as follows:
% 
% %BEGINLATEX%
% \begin{tabular}{|c|c|c|}
% \hline
% Jet $p_{\perp}$ & Jet $|\eta|$ & $BDT_{score}$ \\
% \hline \hline
% $20 < p_{\perp} \leq 30$ & $|\eta| < 2.5$ & $BDT_{score} > -0.80$ \\
% $20 < p_{\perp} \leq 30$ & $2.50 \leq |\eta| < 2.75$ & $BDT_{score} > -0.85$ \\
% $20 < p_{\perp} \leq 30$ & $2.75 \leq |\eta| < 3.00$ & $BDT_{score} > -0.84$ \\
% $20 < p_{\perp} \leq 30$ & $3.00 \leq |\eta| < 5.00$ & $BDT_{score} > -0.85$ \\
% $30 < p_{\perp}$ & $|\eta| < 2.5$ & $BDT_{score} > -0.80$ \\
% $30 < p_{\perp}$ & $2.50 \leq |\eta| < 2.75$ & $BDT_{score} > -0.74$ \\
% $30 < p_{\perp}$ & $2.75 \leq |\eta| < 3.00$ & $BDT_{score} > -0.68$ \\
% $30 < p_{\perp}$ & $3.00 \leq |\eta| < 5.00$ & $BDT_{score} > -0.77$ \\
% \hline
% \end{tabular}
% %ENDLATEX%

%STATUS: In progress
%TODO:  
\subsection{Missing transverse energy \texorpdfstring{(\MET)}{}}


%STATUS: In progress
%TODO: 
% Convert base cut based ID to a table  
\subsection{Electrons}

% Make citation for CMS.EgammaCutBasedIdentification
For this analysis we use two categories of electrons "veto electrons" and "tight electrons". Both this categories of particles are based on standard EGamma POG cut based object definition which can be found at (TODO:CITATION). Additionally, we require some additional cuts to each category (TODO:WHY).

\subsubsection{Veto electrons}
 
The "veto electrons" are defined by the base requirements of the cut based electron ID veto working point of the EGamma POG with some additional cuts on top. 
 
Requirements of the cut based electron ID veto working point:

Barrel Cuts ( $ |\eta_{supercluster}|<=1.479 $ )
\begin{itemize}
  \item $ | \Delta\eta(track,supercluster) | < 0.007 $
  \item $ | \Delta\phi(track,supercluster) | < 0.8 $
  \item $ \sigma(i\eta,i\eta) < 0.01 $
  \item $ H/E < 0.15 $
  \item $ |d_{0}(vertex)| < 0.04 $
  \item $ |d_{Z}(vertex)| < 0.2 $
  \item $ \frac{PF_{isolation}}{p_{\perp}} < 0.15 $ for $ \Delta R_{cone}=0.3 $
\end{itemize}

Endcap Cuts ( $ 1.479 < |\eta_{supercluster}| < 2.5 $ )
\begin{itemize}
  \item $ | \Delta\eta(track,supercluster) | < 0.1 $
  \item $ | \Delta\phi(track,supercluster) | < 0.7 $
  \item $ \sigma(i\eta,i\eta) < 0.03 $
  \item $ |d_{0}(vertex)| < 0.04 $
  \item $ |d_{Z}(vertex)| < 0.2 $
  \item $ \frac{PF_{isolation}}{p_{\perp}} < 0.15 $ for $ \Delta R_{cone}=0.3 $
\end{itemize}

Additional requirements for this analysis
\begin{itemize}
  \item $ p_{\perp} > 10 $ GeV
  \item $ |\eta| < 2.4 $
  \item $ Effective-Area-Corrected-Isolation < 0.15 $ (is stated in the note as additional requirement but does not look like it is)
  \item $d_{xy}<0.04$ cm (is stated in the note as additional requirement but does not look like it is)
  \item $d_{z} < 0.2 $ cm (is stated in the note as additional requirement but does not look like it is)
\end{itemize}

\subsubsection{Tight electrons}

The "tight electrons" are defined by using the base requirements of the cut based electron ID tight working point (similar to 2011 very tight WP70) of the EGamma POG with some additional cuts on top.
 
Requirements of the cut based electron ID tight working point (similar to 2011 very tight WP70):

Barrel Cuts ( $ |\eta_{supercluster}|<=1.479 $ )
\begin{itemize}
  \item $ | \Delta\eta(track,supercluster) | < 0.004 $
  \item $ | \Delta\phi(track,supercluster) | < 0.3  $
  \item $ \sigma(i\eta,i\eta) < 0.01 $
  \item $ H/E < 0.12 $
  \item $ |d_{0}(vertex)| < 0.02 $
  \item $ |d_{Z}(vertex)| < 0.1 $
  \item $ |\frac{1}{E}-\frac{1}{p}| < 0.05 $
  \item $ \frac{PF_{isolation}}{p_{\perp}} < 0.10 $ for $ \Delta R_{cone}=0.3 $
  \item Conversion rejection: vertex fit probability: 1e-6
  \item Conversion rejection: missing hits $<= 0$
\end{itemize}

Endcap Cuts ( $ 1.479 < |\eta_{supercluster}| < 2.5 $ )
\begin{itemize}
  \item $ | \Delta\eta(track,supercluster) | < 0.005 $
  \item $ | \Delta\phi(track,supercluster) | < 0.2  $
  \item $ \sigma(i\eta,i\eta) < 0.03 $
  \item $ H/E < 0.10 $
  \item $ |d_{0}(vertex)| < 0.02 $
  \item $ |d_{Z}(vertex)| < 0.1 $
  \item $ |\frac{1}{E}-\frac{1}{p}| < 0.05 $
  \item $ \frac{PF_{isolation}}{p_{\perp}} < 0.10(0.07) $ for $ p_{\perp} > 20 (p_{\perp} <= 20) $ and $ \Delta R_{cone}=0.3 $
  \item Conversion rejection: vertex fit probability: 1e-6
  \item Conversion rejection: missing hits $<= 0$
\end{itemize}

Additional requirements for this analysis:
\begin{itemize}
  \item $ p_{\perp} > 20 $ GeV
  \item $ |\eta| < 2.4 $
  \item $ Effective-Area-Corrected-Isolation < 0.10 $
  \item $d_{xy}<0.02 $ cm
  \item $d_{z} < 0.1 $ cm
\end{itemize}


%STATUS: In progress
%TODO: 
\subsection{Muons}




%STATUS: In progress
%TODO: 
\subsection{Taus}


% \section{Signal selection}

\subsection{Pre-selection}

\begin{table}[!htp]
\centering

\begin{tabular}{|l||c|c|c|c|}
\hline
Filter                  & Prompt A & Parked B & Parked C & Parked D \\
\hline \hline
ECAL Laser Filter       & 0.928521 & 0.008659 & 0.000000 & 0.000000 \\
HCAL Laser Filter       & 0.007258 & 0.000000 & 0.000270 & 0.000000 \\
\hline
ECAL+ HCAL Laser Filter & 0.935704 & 0.008659 & 0.000270 & 0.000000 \\
\hline
\end{tabular}

\caption{Event rejection efficiency for the ECAL and HCAL Laser Filters. This events have to be removed due to the untimely firing of the calibration laser for this systems.}
\label{table_CaloLaserFilterEff}
\end{table}


\begin{table}[htp]
\centering

\begin{tabular}{|l|c|c|c|c||c|}
\hline
 & \rotatebox{90}{Prompt Run A} & \rotatebox{90}{Parked Run B} & \rotatebox{90}{Parked Run C} & \rotatebox{90}{Parked Run D} & \rotatebox{90}{Total Data} \\
\hline \hline
Vertex Filter & 3606391 & 132346320 & 228049748 & 308041846 & 672044305 \\
Event Quality Filters & 2658960 & 131554431 & 226680352 & 305918529 & 666812272 \\
ECAL Laser Filter & 2634271 & 131543040 & 226680352 & 305918529 & 666776192 \\
HCAL Laser Filter & 2634080 & 131543040 & 226679741 & 305918529 & 666775390 \\
L1T ETM Filter & 2461217 & 88174347 & 160560859 & 227801622 & 478998045 \\
HLT Filter & 97522 & 75100422 & 137527238 & 152041761 & 364766943 \\
$N(Electrons_{veto})=0$ & 96600 & 74947192 & 137241812 & 151725585 & 364011189 \\
$N(Muon_{loose})=0$ & 94864 & 74913002 & 137179173 & 151652654 & 363839693 \\
Dijet cut & 28164 & 23666926 & 43292391 & 42218637 & 109206118 \\
MET cut & 6252 & 57929 & 102384 & 120600 & 287165 \\
$MET_{Significance}$ cut & 3828 & 24179 & 42683 & 41620 & 112310 \\
$Min(\Delta\phi(MET,jets))$ cut & 405 & 1824 & 3452 & 3374 & 9055 \\
\hline
\end{tabular}
\caption{Event Yield for the Pre-Selection Region.}
\end{table}


\begin{table}[!htp]
\centering

\begin{tabular}{|l|c|c|c|}
\hline
Dataset & Main Analysis & Cross Check Analysis & $\frac{CC}{Main}-1$ \\ 
\hline \hline
Prompt A &  405 &  405 & 0.00\% \\
Parked B & 1824 & 1824 & 0.00\% \\
Parked C & 3453 & 3452 & -0.03 \% \\
Parked D & 3374 & 3374 & 0.0\% \\
\hline \hline
Total & 9056 & 9055 & -0.01\% \\
\hline
\end{tabular}

\caption{Comparison between main and cross check analysis for the event yield of the event pre-selection. There is a difference of a single event between both analysis which is a difference in total yield around 0.01\%.}
\end{table}


\subsection{Signal region}

\begin{table}[htp]
\centering

\begin{tabular}{|l|c|c|c|c||c|}
\hline
 & \rotatebox{90}{Prompt Run A} & \rotatebox{90}{Parked Run B} & \rotatebox{90}{Parked Run C} & \rotatebox{90}{Parked Run D} & \rotatebox{90}{Total Data} \\
\hline \hline
Vertex Filter & 3606391 & 132346320 & 228049748 & 308041846 & 672044305 \\
Event Quality Filters & 2658960 & 131554431 & 226680352 & 305918529 & 666812272 \\
ECAL Laser Filter & 2634271 & 131543040 & 226680352 & 305918529 & 666776192 \\
HCAL Laser Filter & 2634080 & 131543040 & 226679741 & 305918529 & 666775390 \\
L1T ETM Filter & 2461217 & 88174347 & 160560859 & 227801622 & 478998045 \\
HLT Filter & 97522 & 75100422 & 137527238 & 152041761 & 364766943 \\
$N(Electrons_{veto})=0$ & 96600 & 74947192 & 137241812 & 151725585 & 364011189 \\
$N(Muon_{loose})=0$ & 94864 & 74913002 & 137179173 & 151652654 & 363839693 \\
Dijet cut & 18338 & 13678405 & 25090291 & 24082304 & 62869338 \\
MET cut & 4167 & 38178 & 68047 & 79723 & 190115 \\
$MET_{Significance}$ cut & 786 & 3396 & 5988 & 5567 & 15737 \\
$Min(\Delta\phi(MET,jets))$ cut & 34 & 91 & 205 & 178 & 508 \\
\hline
\end{tabular}
\caption{Event Yield for the Signal Region.}
\end{table}


\begin{table}[!htp]
\centering

\begin{tabular}{|l|c|c|c|}
\hline
Dataset & Main Analysis & Cross Check Analysis & $\frac{CC}{Main}-1$ \\ 
\hline \hline
Prompt A &  34 &  34 & 0.00\% \\
Parked B &  91 &  91 & 0.00\% \\
Parked C & 205 & 205 & 0.00 \% \\
Parked D & 178 & 178 & 0.00\% \\
\hline \hline
Total & 508 & 508 & 0.00\% \\
\hline
\end{tabular}

\caption{Comparison between main and cross check analysis for the event yield of signal region. No difference in yields is observed either in total or by acquisition era.}
\end{table}

% \section{Background estimation}

\subsection{W to \texorpdfstring{electron+\MET}{electron+MET}}

The selection consists of the following cuts:

\begin{itemize}
  \item Vertex cut
  \item Event quality filters (MET Filters)
  \item ECAL+HCAL Laser filters
  \item L1T ETM Filter (L1T\_ETM40 emulation)
  \begin{itemize}
    \item $ L1T\_ETM >= 40 $
  \end{itemize}
  \item HLT path filter
  \begin{itemize}
    \item From run 190456 to 193621 (Run 2012 A) use HLT\_DiPFJet40\_PFMETnoMu65\_MJJ800VBF\_AllJets* 
    \item From run 193833 to 196531 (Run 2012 B) use HLT\_DiJet35\_MJJ700\_AllJets\_DEta3p5\_VBF*
    \item From run 198022 to 203742 (Run 2012 C) use HLT\_DiJet35\_MJJ700\_AllJets\_DEta3p5\_VBF*
    \item From run 203777 to 208686 (Run 2012 D) use HLT\_DiJet30\_MJJ700\_AllJets\_DEta3p5\_VBF*
  \end{itemize}
  \item Exactly one $Electron_{Veto}$
  \begin{itemize}
    \item Using veto electron defined on this page
  \end{itemize}
  \item Exactly one $Electron_{Tight}$
  \begin{itemize}
    \item Using tight electron defined on this page
  \end{itemize}
  \item Muon Veto
  \begin{itemize}
    \item Using veto muons defined on this page (to be done)
  \end{itemize}
  \item $ MET > 90 $ GeV
  \item $ MET_{significance} > 4.0 $
  \item Dijet cut (leading dijet requirements):
  \begin{itemize}
    \item Lead dijet $ p_{\perp} > 50$ GeV
    \item Sub-lead dijet $ p_{\perp} > 45$ GeV
    \item Jets $ |\eta| < 4.7 $
    \item Dijet $ \Delta\eta < 3.6 $
    \item Dijet $ m_{jj} > 1200 $ GeV
  \end{itemize}
  \item $ Min(\Delta\phi(MET,Jet_{p_{\perp}>30})))>2.3 $
\end{itemize}

\begin{table}[!htp]
\centering

\begin{tabular}{|l|c|c|c|c||c|}
\hline
 & \rotatebox{90}{Prompt Run A} & \rotatebox{90}{Parked Run B} & \rotatebox{90}{Parked Run C} & \rotatebox{90}{Parked Run D} & \rotatebox{90}{Total Data} \\
\hline \hline
Vertex Filter & 3606391 & 132346320 & 228049748 & 308041846 & 672044305 \\
Event Quality Filters & 2658960 & 131554431 & 226680352 & 305918529 & 666812272 \\
ECAL Laser Filter & 2634271 & 131543040 & 226680352 & 305918529 & 666776192 \\
HCAL Laser Filter & 2634080 & 131543040 & 226679741 & 305918529 & 666775390 \\
L1T ETM Filter & 2461217 & 88174347 & 160560859 & 227801622 & 478998045 \\
HLT Filter & 97522 & 75100422 & 137527238 & 152041761 & 364766943 \\
$N(Electrons_{veto})=1$ & 899 & 151621 & 282481 & 312807 & 747808 \\
$N(Muon_{loose})=0$ & 852 & 151249 & 281801 & 312033 & 745935 \\
$N(Electrons_{tight})=1$ & 398 & 23751 & 44323 & 50279 & 118751 \\
Dijet cut & 64 & 1607 & 3145 & 3077 & 7893 \\
MET cut & 55 & 281 & 543 & 525 & 1404 \\
$MET_{Significance}$ cut & 31 & 123 & 242 & 197 & 593 \\
$Min(\Delta\phi(MET,jets))$ cut & 4 & 16 & 24 & 24 & 68 \\
\hline
\end{tabular}

\caption{Electron+MET Yields}
\end{table}


\begin{table}[!htp]
  \centering
  
\begin{tabular}{|l|c|c||c|}
  \hline
  Dataset & Main Analysis & Cross Check Analysis & $\frac{CC}{Main}-1$ \\ 
  \hline \hline
  Prompt Run A &  4 &  4 & 0.00\% \\
  Parked Run B & 16 & 16 & 0.00\% \\
  Parked Run C & 24 & 24 & 0.00\% \\
  Parked Run D & 24 & 24 & 0.00\% \\
  \hline \hline
  Total & 68 & 68 & 0.00\% \\
  \hline
\end{tabular}

\caption{Comparison between main and cross check analysis for the event yield of Electron+MET event selection. No difference in yields is observed either in total or by acquisition era.}
\end{table}

\subsection{W to \texorpdfstring{$\mu$+\MET}{muon+MET}}

The selection consists of the following cuts
\begin{itemize}
  \item Vertex cut 
  \item Event quality filters (MET Filters)
  \item ECAL+HCAL Laser filters
  \item L1T ETM Filter (L1T\_ETM40 emulation)
  \begin{itemize}
    \item $ L1T\_ETM >= 40 $
  \end{itemize}
  \item HLT path filter
  \begin{itemize}
    \item From run 190456 to 193621 (Run 2012 A) use HLT\_DiPFJet40\_PFMETnoMu65\_MJJ800VBF\_AllJets* 
    \item From run 193833 to 196531 (Run 2012 B) use HLT\_DiJet35\_MJJ700\_AllJets\_DEta3p5\_VBF*
    \item From run 198022 to 203742 (Run 2012 C) use HLT\_DiJet35\_MJJ700\_AllJets\_DEta3p5\_VBF*
    \item From run 203777 to 208686 (Run 2012 D) use HLT\_DiJet30\_MJJ700\_AllJets\_DEta3p5\_VBF*
  \end{itemize}
  \item Electron Veto
  \begin{itemize}
    \item Using veto electron defined on this page
  \end{itemize}
  \item Exactly one $Muon_{loose}$
  \begin{itemize}
    \item Using veto muons defined on this page (to be done)
  \end{itemize}
  \item Exactly one $Muon_{tight}$
  \begin{itemize}
    \item Using veto muons defined on this page (to be done)
  \end{itemize}
  \item $MET > 90 $ GeV
  \item $MET_{significance} > 4.0 $
  \item Dijet cut (leading dijet requirements):
  \begin{itemize}
    \item Lead dijet $ p_{\perp} > 50$ GeV
    \item Sub-lead dijet $ p_{\perp} > 45$ GeV
    \item Jets $ |\eta| < 4.7 $
    \item Dijet $ \Delta\eta < 3.6 $
    \item Dijet $ m_{jj} > 1200 $ GeV
  \end{itemize}
  \item $ Min(\Delta\phi(MET,Jet_{p_{\perp}>30})))>2.3 $
\end{itemize}

\begin{table}[!htp]
\centering

\begin{tabular}{|l|c|c|c|c||c|}
\hline
 & \rotatebox{90}{Prompt Run A} & \rotatebox{90}{Parked Run B} & \rotatebox{90}{Parked Run C} & \rotatebox{90}{Parked Run D} & \rotatebox{90}{Total Data} \\
\hline \hline
Vertex Filter & 3606391 & 132346320 & 228049748 & 308041846 & 672044305 \\
Event Quality Filters & 2658960 & 131554431 & 226680352 & 305918529 & 666812272 \\
ECAL Laser Filter & 2634271 & 131543040 & 226680352 & 305918529 & 666776192 \\
HCAL Laser Filter & 2634080 & 131543040 & 226679741 & 305918529 & 666775390 \\
L1T ETM Filter & 2461217 & 88174347 & 160560859 & 227801622 & 478998045 \\
HLT Filter & 97522 & 75100422 & 137527238 & 152041761 & 364766943 \\
$N(Electrons_{veto})=0$ & 96600 & 74947192 & 137241812 & 151725585 & 364011189 \\
$N(Muon_{loose})=1$ & 1625 & 33505 & 61325 & 71449 & 167904 \\
$N(Muon_{tight})=1$ & 1223 & 9662 & 17873 & 20088 & 48846 \\
Dijet cut & 176 & 1493 & 2740 & 2684 & 7093 \\
MET cut & 157 & 809 & 1545 & 1493 & 4004 \\
$MET_{Significance}$ cut & 86 & 487 & 910 & 825 & 2308 \\
$Min(\Delta\phi(MET,jets))$ cut & 10 & 60 & 124 & 106 & 300 \\
\hline
\end{tabular}
\caption{Muon+MET Region}
\end{table}


\begin{table}[!htp]
\centering

\begin{tabular}{|l|c|c||c|}
  \hline
  Dataset & Main Analysis & Cross Check Analysis & $\frac{CC}{Main}-1$ \\ 
  \hline \hline
  Prompt Run A &  10 &  10 & 0.00\% \\
  Parked Run B &  60 &  60 & 0.00\% \\
  Parked Run C & 124 & 124 & 0.00\% \\
  Parked Run D & 106 & 106 & 0.00\% \\
  \hline \hline
  Total & 300 & 300 & 0.00\% \\
  \hline
\end{tabular}

\caption{Comparison between main and cross check analysis for the event yield of Muon+MET event selection. No difference in yields is observed either in total or by acquisition era.}
\end{table}

\subsection{W to \texorpdfstring{$\tau$+\MET}{tau+MET}}

\begin{table}[!htp]
\centering

\begin{tabular}{|l|c|c|c|c||c|}
\hline
 & \rotatebox{90}{Prompt Run A} & \rotatebox{90}{Parked Run B} & \rotatebox{90}{Parked Run C} & \rotatebox{90}{Parked Run D} & \rotatebox{90}{Total Data} \\
\hline \hline
Vertex Filter & 3606391 & 132346320 & 228049748 & 308041846 & 672044305 \\
Event Quality Filters & 2658960 & 131554431 & 226680352 & 305918529 & 666812272 \\
ECAL Laser Filter & 2634271 & 131543040 & 226680352 & 305918529 & 666776192 \\
HCAL Laser Filter & 2634080 & 131543040 & 226679741 & 305918529 & 666775390 \\
L1T ETM Filter & 2461217 & 88174347 & 160560859 & 227801622 & 478998045 \\
HLT Filter & 97522 & 75100422 & 137527238 & 152041761 & 364766943 \\
$N(Electrons_{veto})=0$ & 96600 & 74947192 & 137241812 & 151725585 & 364011189 \\
$N(Muon_{loose})=0$ & 94864 & 74913002 & 137179173 & 151652654 & 363839693 \\
Dijet cut & 18338 & 13678405 & 25090291 & 24082304 & 62869338 \\
MET cut & 4167 & 38178 & 68047 & 79723 & 190115 \\
$MET_{Significance}$ cut & 786 & 3396 & 5988 & 5567 & 15737 \\
$N(Tau)=1$ & 12 & 47 & 63 & 59 & 181 \\
$M_{perp}(MET,\tau)$ cut & 5 & 35 & 46 & 38 & 124 \\
$Min(\Delta\phi(MET,Dijet))$ cut & 2 & 22 & 25 & 27 & 76 \\
\hline
\end{tabular}
\caption{Event Yield for the Tau+MET Region.}
\end{table}


\begin{table}[!htp]
\centering

\begin{tabular}{|l|c|c||c|}
  \hline
  Dataset & Main Analysis & Cross Check Analysis & $\frac{CC}{Main}-1$ \\ 
  \hline \hline
  Prompt Run A &  2 &  2 & 0.00\% \\
  Parked Run B & 22 & 22 & 0.00\% \\
  Parked Run C & 25 & 25 & 0.00\% \\
  Parked Run D & 27 & 27 & 0.00\% \\
  \hline \hline
  Total & 76 & 76 & 0.00\% \\
  \hline
\end{tabular}

\caption{Comparison between main and cross check analysis for the event yield of Tau+MET event selection. No difference in yields is observed either in total or by acquisition era.}
\end{table}

\subsection{Z to \texorpdfstring{$\mu\mu$}{mumu}}

The selection consists of the following cuts:
\begin{itemize}
  \item Vertex cut
  \item Event quality filters (MET Filters)
  \item ECAL+HCAL Laser filters
  \item !L1T ETM Filter (L1T\_ETM40 emulation)
  \begin{itemize}
    \item $ L1T\_ETM >= 40 $
  \end{itemize}
  \item HLT path filter
  \begin{itemize}
    \item From run 190456 to 193621 (Run 2012 A) use HLT\_DiPFJet40\_PFMETnoMu65\_MJJ800VBF\_AllJets* 
    \item From run 193833 to 196531 (Run 2012 B) use HLT\_DiJet35\_MJJ700\_AllJets\_DEta3p5\_VBF*
    \item From run 198022 to 203742 (Run 2012 C) use HLT\_DiJet35\_MJJ700\_AllJets\_DEta3p5\_VBF*
    \item From run 203777 to 208686 (Run 2012 D) use HLT\_DiJet30\_MJJ700\_AllJets\_DEta3p5\_VBF*
  \end{itemize}
  \item Electron Veto
  \begin{itemize}
    \item Using veto electron defined on this page
  \end{itemize}
  \item Exactly two $Muon_{loose}$
  \begin{itemize}
    \item Using veto muons defined on this page (to be done)
  \end{itemize}
  \item Exactly two $Muon_{tight}$
  \begin{itemize}
    \item Using veto muons defined on this page (to be done)
  \end{itemize}
  \item Dimuon with $60<mass<120$ GeV
  \item $ MET > 90 $ GeV
  \item $ MET_{significance} > 4.0 $
  \item Dijet cut (leading dijet requirements):
  \begin{itemize}
    \item Lead dijet $ p_{\perp} > 50$ GeV
    \item Sub-lead dijet $ p_{\perp} > 45$ GeV
    \item Jets $ |\eta| < 4.7 $
    \item Dijet $ \Delta\eta < 3.6 $
    \item Dijet $ m_{jj} > 1200 $ GeV
  \end{itemize}
\end{itemize}
  
\begin{table}[!htp]
\centering

\begin{tabular}{|l|c|c|c|c||c|}
\hline
 & \rotatebox{90}{Prompt Run A} & \rotatebox{90}{Parked Run B} & \rotatebox{90}{Parked Run C} & \rotatebox{90}{Parked Run D} & \rotatebox{90}{Total Data} \\
\hline \hline
Vertex Filter & 3606391 & 132346320 & 228049748 & 308041846 & 672044305 \\
Event Quality Filters & 2658960 & 131554431 & 226680352 & 305918529 & 666812272 \\
ECAL Laser Filter & 2634271 & 131543040 & 226680352 & 305918529 & 666776192 \\
HCAL Laser Filter & 2634080 & 131543040 & 226679741 & 305918529 & 666775390 \\
L1T ETM Filter & 2461217 & 88174347 & 160560859 & 227801622 & 478998045 \\
HLT Filter & 97522 & 75100422 & 137527238 & 152041761 & 364766943 \\
$N(Electrons_{veto})=0$ & 96600 & 74947192 & 137241812 & 151725585 & 364011189 \\
selTwoMuonsLoose & 111 & 683 & 1312 & 1480 & 3586 \\
selTwoMuonsTight & 73 & 450 & 822 & 935 & 2280 \\
Z Mass cut & 62 & 379 & 686 & 768 & 1895 \\
Dijet cut & 11 & 58 & 98 & 96 & 263 \\
MET cut & 9 & 41 & 74 & 70 & 194 \\
$MET_{Significance}$ cut & 7 & 18 & 55 & 44 & 124 \\
$Min(\Delta\phi(MET,jets))$ cut & 2 & 4 & 5 & 7 & 18 \\
\hline
\end{tabular}
\caption{Event Yield for the Z to $\mu\mu$ Region.}
\end{table}


\begin{table}[!htp]
\centering

\begin{tabular}{|l|c|c||c|}
  \hline
  Dataset & Main Analysis & Cross Check Analysis & $\frac{CC}{Main}-1$ \\ 
  \hline \hline
  Prompt A & 2 & 2 & 0.00\% \\
  Parked B & 4 & 4 & 0.00\% \\
  Parked C & 5 & 5 & 0.00\% \\
  Parked D & 7 & 7 & 0.00\% \\
  \hline \hline
  Total & 18 & 18 & 0.00\% \\
  \hline
\end{tabular}

\caption{Comparison between main and cross check analysis for the event yield of Z to $\mu\mu$ event selection. No difference in yields is observed either in total or by acquisition era.}
\end{table}

\subsection{Top}

The selection consists of the following cuts:
\begin{itemize}
  \item Vertex cut 
  \item Event quality filters (MET Filters)
  \item ECAL+HCAL Laser filters
  \item L1T ETM Filter (L1T\_ETM40 emulation)
  \begin{itemize}
    \item $ L1T\_ETM >= 40 $
  \end{itemize}
  \item HLT path filter
  \begin{itemize}
    \item From run 190456 to 193621 (Run 2012 A) use HLT\_DiPFJet40\_PFMETnoMu65\_MJJ800VBF\_AllJets* 
    \item From run 193833 to 196531 (Run 2012 B) use HLT\_DiJet35\_MJJ700\_AllJets\_DEta3p5\_VBF*
    \item From run 198022 to 203742 (Run 2012 C) use HLT\_DiJet35\_MJJ700\_AllJets\_DEta3p5\_VBF*
    \item From run 203777 to 208686 (Run 2012 D) use HLT\_DiJet30\_MJJ700\_AllJets\_DEta3p5\_VBF*
  \end{itemize}
  \item Exactly one $Electron_{Veto}$
  \begin{itemize}
    \item Using veto electron defined on this page
  \end{itemize}
  \item Exactly one $Electron_{Tight}$
  \begin{itemize}
    \item Using tight electron defined on this page
  \end{itemize}
  \item Exactly one $Muon_{loose}$
  \begin{itemize}
    \item Using veto muons defined on this page (to be done)
  \end{itemize}
  \item Exactly one $Muon_{tight}$
  \begin{itemize}
    \item Using veto muons defined on this page (to be done)
  \end{itemize}
  \item $MET > 90 $ GeV
  \item $MET_{significance} > 4.0 $
  \item Dijet cut (leading dijet requirements):
  \begin{itemize}
    \item Lead dijet $ p_{\perp} > 50$ GeV
    \item Sub-lead dijet $ p_{\perp} > 45$ GeV
    \item Jets $ |\eta| < 4.7 $
    \item Dijet $ \Delta\eta < 3.6 $
    \item Dijet $ m_{jj} > 1200 $ GeV
  \end{itemize}
\end{itemize}
    
\begin{table}[!htp]
\centering

\begin{tabular}{|l|c|c|c|c||c|}
\hline
 & \rotatebox{90}{Prompt Run A} & \rotatebox{90}{Parked Run B} & \rotatebox{90}{Parked Run C} & \rotatebox{90}{Parked Run D} & \rotatebox{90}{Total Data} \\
\hline \hline
Vertex Filter & 3606391 & 132346320 & 228049748 & 308041846 & 672044305 \\
Event Quality Filters & 2658960 & 131554431 & 226680352 & 305918529 & 666812272 \\
ECAL Laser Filter & 2634271 & 131543040 & 226680352 & 305918529 & 666776192 \\
HCAL Laser Filter & 2634080 & 131543040 & 226679741 & 305918529 & 666775390 \\
L1T ETM Filter & 2461217 & 88174347 & 160560859 & 227801622 & 478998045 \\
HLT Filter & 97522 & 75100422 & 137527238 & 152041761 & 364766943 \\
$N(Muon_{loose})=1$ & 1674 & 33877 & 61994 & 72215 & 169760 \\
$N(Muon_{tight})=1$ & 1259 & 9913 & 18322 & 20585 & 50079 \\
$N(Electrons_{veto})=1$ & 35 & 247 & 444 & 492 & 1218 \\
$N(Electrons_{tight})=1$ & 23 & 150 & 273 & 300 & 746 \\
Dijet cut & 0 & 15 & 21 & 17 & 53 \\
MET cut & 0 & 10 & 14 & 12 & 36 \\
$MET_{Significance}$ cut & 0 & 4 & 9 & 8 & 21 \\
\hline
\end{tabular}
\caption{Event Yield for the Top Region.}
\end{table}


\begin{table}[!htp]
\centering

\begin{tabular}{|l|c|c||c|}
  \hline
  Dataset & Main Analysis & Cross Check Analysis & $\frac{CC}{Main}-1$ \\ 
  \hline \hline
  Prompt A & 0 & 0 & 0.00\% \\
  Parked B & 4 & 4 & 0.00\% \\
  Parked C & 9 & 9 & 0.00\% \\
  Parked D & 8 & 8 & 0.00\% \\
  \hline \hline
  Total & 21 & 21 & 0.00\% \\
  \hline
\end{tabular}

\caption{Comparison between main and cross check analysis for the event yield of top event selection. No difference in yields is observed either in total or by acquisition era.}
\end{table}

\subsection{QCD}

\begin{table}[htp]
\centering

\begin{tabular}{|l|c|c|c|c||c|}
\hline
 & \rotatebox{90}{Prompt Run A} & \rotatebox{90}{Parked Run B} & \rotatebox{90}{Parked Run C} & \rotatebox{90}{Parked Run D} & \rotatebox{90}{Total Data} \\
\hline \hline
Vertex Filter & 3606391 & 132346320 & 228049748 & 308041846 & 672044305 \\
Event Quality Filters & 2658960 & 131554431 & 226680352 & 305918529 & 666812272 \\
ECAL Laser Filter & 2634271 & 131543040 & 226680352 & 305918529 & 666776192 \\
HCAL Laser Filter & 2634080 & 131543040 & 226679741 & 305918529 & 666775390 \\
L1T ETM Filter & 2461217 & 88174347 & 160560859 & 227801622 & 478998045 \\
HLT Filter & 97522 & 75100422 & 137527238 & 152041761 & 364766943 \\
$N(Electrons_{veto})=0$ & 96600 & 74947192 & 137241812 & 151725585 & 364011189 \\
$N(Muon_{loose})=0$ & 94864 & 74913002 & 137179173 & 151652654 & 363839693 \\
Dijet cut & 18338 & 13678405 & 25090291 & 24082304 & 62869338 \\
MET cut & 4167 & 38178 & 68047 & 79723 & 190115 \\
$MET_{Significance}$ cut & 2532 & 15594 & 27623 & 27068 & 72817 \\
$Min(\Delta\phi(MET,jets))$ cut & 2314 & 14691 & 25826 & 25326 & 68157 \\
\hline
\end{tabular}
\caption{Event Yield for the QCD Region.}
\end{table}





  \chapter{Technical work}

\section{Trigger work}
  \chapter{Physics Analysis}

\section{Prompt/Parked trigger studies}

\section{Prompt Analysis}

\section{Parked Analysis}

\section{Run II trigger studies}

\section{Run II Analysis}


  \chapter{Conclusions}

Summary of relevant results and their impact on Particle Physics

\end{mainmatter}

\begin{backmatter}
% \leavevmode

\begin{colophon}
  This thesis was made in \LaTeXe{} using the ``hepthesis'' class~\cite{hepthesis}.
\end{colophon}

%% You're recommended to use the eprint-aware biblio styles which
%% can be obtained from e.g. www.arxiv.org. The file mythesis.bib
%% is derived from the source using the SPIRES Bibtex service.
\bibliographystyle{h-physrev}
\bibliography{mythesis}

%% I prefer to put these tables here rather than making the
%% front matter seemingly interminable. No-one cares, anyway!
\listoffigures
\listoftables

%% If you have time and interest to generate a (decent) index,
%% then you've clearly spent more time on the write-up than the 
%% research ;-)
%\printindex


\end{backmatter}

\end{document}
