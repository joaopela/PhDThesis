\chapter{Conclusions}

This thesis has presented the work developed to implement effective monitoring tools for the \gls{CMS} \gls{L1T} system, two physics analysis performed using the \gls{LHC} Run I data using the \gls{CMS} detector and the preparation of the future Run II analysis. The presented physics analysis are focused on the search for the \gls{VBF} produced \gls{SM} Higgs boson decaying invisibly using initially \textit{prompt data} and later \textit{parked data}. This search main objective was to attempt to observe excesses which could point to the direct production of dark matter via Higgs decay.

The developed tools for the \gls{CMS} \gls{L1T} system, were using during Run I and provide the ability to monitoring benchmark trigger rates and synchronization, monitor the correct operation of the \gls{BPTX} system, detect problematic regions of the detector and allow easy trouble shooting for the shift crew. These tools also played a crucial role the data certification for physics usage allowing identification of problematic periods of data taking.

