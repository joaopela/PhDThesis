\chapter{Conclusions}
\label{CHAPTER:Conclusions}

%Status: DONE (reviewed J.Pela x1)

This thesis describes the study of a Higgs particle decaying invisibly. These studies start with the development of a dedicated trigger and continue to the cross checking of the final analysis on parked data. Finally, preparation for an updated analysis for Run II of the \gls{LHC} was presented.

The search for \gls{VBF} produced \gls{SM} Higgs boson decaying invisibly was preformed using the full promptly reconstructed $8\,\TeV$ Run I data. A dedicated trigger was used to collect events over which a single bin counting experiment was optimised to select events containing a dijet with \gls{VBF} characteristics and large \gls{MET}. Control regions were also defined to normalise the main background processes which were extrapolated to the signal region with the help of \gls{MC} simulation. In the signal region, 390 data events were observed, this yield is compatible with the background only prediction. Since no evidence of signal is observed 95\% \gls{CL} upper limits on the Higgs boson production cross section times branching were determined. Assuming the \gls{SM} \gls{VBF} production cross section and acceptance, corresponding to an observed (expected) upper limit on \BRinv of 0.65 (0.49) for $m_H=125\,\GeV$.

Following the reconstruction of the lower trigger thresholds parked datasets, after the end of the Run I data taking, a new analysis was designed to take advantage of the higher signal acceptance provided by dedicated triggers in this new available data. New variables were introduced, as well as new control regions and background estimation methods. A new cross check analysis was also implemented to validate the obtained results. In the signal region 508 data events were observed, this yield is again compatible with the background only prediction. In the absence of signal, 95\% \gls{CL} upper limits, were once again determined for the Higgs boson production cross section times branching. Under the assumption of the \gls{SM} production cross sections and acceptances, the observed (expected) 95\% C.L. limit on \BRinv of a \gls{SM} $125\,\GeV$ Higgs boson is 57\% (40\%).

In preparation for the Run II analysis new dedicated triggers, for both signal recording and systematics control, were developed and were successfully used during the 2015 Run II data taking. Additionally, a study leading to official production proposal of \gls{QCD} \gls{MC} datasets with signal like characteristics was performed. This new approach should for the first time allow the simulation of events with miss measured \gls{MET} to be used in future searches.

The developed tools for the \gls{CMS} \gls{L1T} system, were used during Run I and provided the ability to monitor trigger objects production rate, synchronization with the \gls{LHC}, assert the correct operation of the \gls{BPTX} system, detect problematic regions of the detector and allow easy trouble shooting for the shift crew. These tools also played a crucial role in the data certification for physics usage allowing identification of problematic periods of data taking.

The \gls{LHC} continues its ground breaking program exploring the $\TeV$ energy scale. During the \gls{LHC} Run I the discovery of a Higgs boson with mass around $125\,\GeV$ by both the CMS and ATLAS collaborations has lead to the 2013 Nobel Prize in Physics being awarded to Higgs and Englert. With the start of Run II in the beginning of 2015, a significant increase in centre of mass energy to $13\,\TeV$ has been achieved and plans are in place to record even greater volumes of data than before. With these new data physicists will be able to probe the Standard Model even farther, opening the door for new physics discovery.

% This thesis has also presented the development of effective monitoring tools for the \gls{CMS} \gls{L1T} system, two physics analysis performed with \gls{LHC} Run I data using the \gls{CMS} detector and the preparation of the future Run II analysis. The presented physics analysis are focused on the search for the \gls{VBF} produced \gls{SM} Higgs boson decaying invisibly, using at a first stage promptly reconstructed data and at later stage parked data. This main objective of this analysis was to search for excesses which could point to the direct production of dark matter through the Higgs decay.
