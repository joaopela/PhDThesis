\chapter{Experimental Apparatus}
\label{CHAPTER:ExperimentalApparatus}

%%%%%%%%%%%%%%%%%%%%%%%%%%%%%%%%%%%%%%%%%%%%%%%%%%%%%%%%%%%%%%%%%%%%%%%%%%%%%%%%%%%%%%%
%%% SECTION
%%%%%%%%%%%%%%%%%%%%%%%%%%%%%%%%%%%%%%%%%%%%%%%%%%%%%%%%%%%%%%%%%%%%%%%%%%%%%%%%%%%%%%%
\section{The Large Hadron Collider}
\label{SECTION:ExperimentalApparatus_LHC}

% Status: DONE (reviewed J.Pela x1)
%
% TODO:
% * DONE: LHC location, size, particles used, energy usage.
% * Basics of machine and operation
% * How instantaneous luminosity is calculated include Instantaneous luminosity equation
% * Delivered instantaneous luminosity Run I (proton-Proton)

% \colorbox{red}{
% \begin{minipage}{0.95\linewidth}
%  
% \begin{itemize}\item\end{itemize}
% \end{minipage}
%}

% CERN and LHC Location
The \gls{LHC}\cite{ARTICLE:LHC Machine} is currently the world's largest particle accelerator and is capable to produce the highest energy particle beam ever made by mankind. This gigantic machine with a total perimeter of $26.7\,\kilo\meter$ was built at \gls{CERN} in a circular tunnel, where previously the \gls{LEP}\cite{LEPTDR:LEPInjectorStudyGroup} was installed, at an average depth of $100\,\meter$ below ground under the Franco-Swiss border near Geneva, Switzerland. A diagram of the \gls{LHC} tunnel and its experiments can be found at figure \ref{FIGURE:ExperimentalApparatus_LHCLayoutUnderground}.

\begin{figure}[!htb]
  \centering
  \includegraphics[width=0.50\textwidth]{Chapter02/LHC/Images/LHC_layout_underground.jpg}
  \caption{Underground diagram of the Geneva area showing the \gls{LHC} and its experiments location.}
  \label{FIGURE:ExperimentalApparatus_LHCLayoutUnderground}
\end{figure}

% Structure and experiments
The \gls{LHC} is a synchrotron machine with the capability to accelerate two particles beams in opposite directions in two separated beam pipes. These beams only cross and are allowed to collide in four points of the accelerator where huge particle detectors are installed to detect the products of such collisions. This experiments are: \gls{ATLAS}\cite{ARTICLE:TheATLASExperiment}, \gls{CMS}\cite{ARTICLE:TheCMSExperiment}, \gls{LHCb}\cite{ARTICLE:TheLHCbExperiment} and \gls{ALICE}\cite{ARTICLE:TheALICEExperiment}.
% Add other experiments?

% Objective
The objective of the \gls{LHC} program is to investigate physics at the $\TeV$ scale, more specifically to understand the electroweak symmetry breaking and if this phenomena could be explained by the Higgs mechanism. And of course search for any indications of \gls{BSM} physics. \gls{ATLAS} and \gls{CMS} are general-purpose detectors which aim to investigate a broad spectrum of physics. The \gls{LHCb} detector is used to study processes that involve the decay of b-flavoured hadrons. The \gls{ALICE} detector is optimised to look at heavy-ion collisions and to investigate the properties of extreme high density medium formed in those collisions.

% CERN accelerator chain
The \gls{LHC} is only the last element of a complex accelerator chain which step by step increases the energy of the particles to eventually be collided. Protons are initially obtained by stripping the electrons of hydrogen gas. This process happens at the begging of the \gls{LINAC2} which then accelerates them up to the energy of $50\,\MeV$. After this initial step proton are injected into the \gls{PSB} and the energy ramps ups to $1.4\,\GeV$. Particles are then passed to the \gls{PS} where the energy futher increases to $25\,\GeV$, subsequently they are the injected into the \gls{SPS} where the particle energy level reached $450\,\GeV$. Finally, protons pass to the \gls{LHC} where they can be accelerated to a maximum energy of $7\,\TeV$. A simplified diagram of the \gls{CERN} accelerator chain can be found in figure \ref{FIGURE:ExperimentalApparatus_LHCAccelaratorChain}. 
Normal operation of the \gls{LHC} therefore depends on all the upstream accelerators availability. The typically turn around time, the time necessary to stop the accelerator from running and restart collisions is around 2 hours. When stable beams are achieved, a single proton fill can be used to collide protons up to 24 hours, but it is common to restart more frequently to take profit of the higher collision rates possible right at the beginning of a new fill.

\begin{figure}[!htb]
  \centering
  \includegraphics[width=0.50\textwidth]{Chapter02/LHC/Images/LHCAccelaratorChain.png}
  \caption{CERN Large Hadron Collider Experiment accelerator diagram.}
  \label{FIGURE:ExperimentalApparatus_LHCAccelaratorChain}
\end{figure}


% LHC modes of opperation
The \gls{LHC} as its name indicates can collide hadrons, more specifically proton or heavy ions. Three modes of operation have been tried according to the particles being collided: proton-proton, proton-lead and lead-lead. By changing the incoming particles we are changing the quantity of nucleons present at each interaction. The maximum design energy per proton is $7\,\TeV$ and for each lead nucleon $2.76\,\TeV$. The maximum design luminosity for proton-proton is of $10^{34}\,\cm^{-2}\second^{-1}$ and for lead-lead is of $10^{27}\,\cm^{-2}\second^{-1}$.

% LHC structure
Particles beams trajectory are curved by 1232 niobium-titanium superconducting dipole magnets each with a length of $14.3\,\meter$ which are cooled with superfluid helium to $1.9\,\kelvin$ to be able to produce a magnetic field of $8.4\,\tesla$. To accelerate the beam it uses eight \gls{RF} cavities which increase each bunch energy at every turn of the \gls{LHC}. At nominal operation the \gls{LHC} will steer 2808 bunches separated by $25\,\ns$ in each direction each bunch composed up to $10^{11}$ protons. Some of the key parameters of the \gls{LHC} proton-proton and lead-lead operation can be found in table \ref{TABLE:ExperimentalApparatus_LHCMachineParameters}.

\begin{table}[!htb]
  \centering
  \begin{threeparttable}
    \begin{tabular}{|lcccc|}
    \hline 
                                  &              &           \textit{pp} &         \textbf{HI} &  \\
    \hline
    Energy per nucleon            & E            &                     7 &                2.76 &                 $\TeV$ \\
    Dipole field at 7 TeV         & \textit{B}   &                  8.33 &                8.33 &               $\tesla$ \\
    Design Luminosity\tnote{*}    & $\mathcal{L}$ &            $10^{34}$ &           $10^{27}$ & $\cm^{-2}\second^{-1}$ \\
    Bunch separation              &              &                    25 &                 100 &                  $\ns$ \\
    No. of bunches                & $k_B$        &                  2808 &                 592 &                        \\
    No. particles per bunch       & $N_p$        & $1.15 \times 10^{11}$ & $7.0 \times 10^{7}$ &                        \\
    \hline
    \hline
    \textbf{Collisions}           &              &  &  &  \\
    \hline
    $\beta$-value at IP           & $\beta^{*}$  &                  0.55 &                 0.5 &        $\meter$ \\
    RMS beam radius at IP         & $\sigma^{*}$ &                  16.7 &                15.9 &  $\micro\meter$ \\
    Luminosity lifetime           & $\tau_L$     &                    15 &                   6 &         $\hour$ \\
    Number of collisions/crossing & $n_c$        &          $\approx 20$ &                   - &                 \\
    \hline
    \end{tabular}
    \begin{tablenotes}
      \item[*] For heavy-ion (HI) operation the design luminosity for Pb-Pb collisions is given.
    \end{tablenotes}
  \end{threeparttable}
  \caption[LHC parameters relevant for detectors]{The machine parameters relevant for the 
                                                  LHC detectors.\cite{CMSTDR:CMSPhysicsVol1}}
  \label{TABLE:ExperimentalApparatus_LHCMachineParameters}
\end{table}


At the \gls{LHC} we are looking for extremely rare processes as is can be seen in figure \ref{FIGURE:ExperimentalApparatus_LHCCrossSections} the production cross section of a \gls{SM} Higgs boson is more than 9 orders of magnitude smaller than the total proton-proton cross section. 

\begin{figure}[!htb]
  \centering
  \includegraphics[width=0.50\textwidth]{Chapter02/LHC/Images/crosssections2012_v5}
  \caption{Cross sections for several processes for collisions of antiproton-proton and proton-proton as a function of the center of mass energy\cite{ARTICLE:TheCMSExperiment}.}
  \label{FIGURE:ExperimentalApparatus_LHCCrossSections}
\end{figure}

To be able to record and study such rare processes we need to produce a significant amount of collisions. For this purpose the LHC was designed to operate at high instantaneous luminosity, L. This quantity is defined as,

\begin{equation}
L=\frac{N_{b}^{2}n_{b}f_{\text{rev}}\gamma}{4\pi\epsilon_{n}\beta^{*}}F,
\end{equation}

where $N_{b}$ is the number of protons per bunch, $n_{b}$ is the number of bunches, $f_{\text{rev}}$ is the frequency of revolution, $\gamma$ is the Lorentz factor, $\epsilon_{n}$ is the normalized emittance, $f_{\text{rev}}$ is the beta function at the collision point and $F$ is the reduction factor due to the crossing angle.

%%%%%%%%%%%%%%%%%%%%%%%%%%%%%%%%%%%%%%%%%%%%%%%%%%%%%%%%%%%%%%%%%%%%%%%%%%%%%%%%%%%%%%%
%%% SUBSECTION
%%%%%%%%%%%%%%%%%%%%%%%%%%%%%%%%%%%%%%%%%%%%%%%%%%%%%%%%%%%%%%%%%%%%%%%%%%%%%%%%%%%%%%%
\subsection{Running and performance}

%Historical
The \gls{LHC} has started its operation with first circulation beams in September 2008. Unfortunately, only a few days after a faulty weld between two dipole magnets caused a significant magnet quench which in turn damaged several dipoles with a simultaneous leak of a significant amount of helium. The event showed that beyond the repair of the affected systems the accelerator needed a significant consolidation program to allow it to return it to activity\cite{ARTICLE:CMSReportIncident19Sep2008}. This consolidation program took over one year to finalise and to prevent further possible problems and allow better understanding of the machine while maximizing physics reach, it was decided to run the \gls{LHC} at $7\,\TeV$ center-of-mass energy.
The period the follow in know as \gls{LHC} run I which started with first collisions at November 2009 just at the \gls{SPS} injection energy of $450\,\GeV$. 

The collision energy was finally ramped up to $7\,\TeV$ with first collisions being observed during March 2010. Operation at this energy continued until the end of 2011, with the peak luminosity being achieved of $3.7 \times 10^{33} \centi\meter^{-2}\second^{-1}$. The total amount of of integrated luminosity delivered to \gls{CMS} was $6.1\,\femto\barn^{-1}$ with the total actually recorded being $5.6\,\femto\barn^{-1}$. During 2012 with the increase knowledge of the accelerator it was possible to increase the centre-of-mass energy further to $8\,\TeV$ and eventually reaching peak luminosity of $7.7 \times 10^{33}\,\centi\meter^{-2}\second^{-2}$ and delivering $23.3\,\femto\barn^{-1}$ to \gls{CMS} of which $21.79,\femto\barn^{-1}$  $\femto\barn^{-1}$ were recorded. Figure \ref{FIGURE:ExperimentalApparatus_CMS_IntegratedLumi_pp_2010-2012} shows the delivered luminosity in the period 2010-2013 against time.

\begin{figure}[!htb]
  \centering
  \includegraphics[width=1.00\textwidth]{Chapter02/CMS/Images/CMS_IntegratedLumi_pp_2010-2012}
  \caption{Cumulative luminosity versus day delivered to CMS during stable beams and for p-p collisions. This is shown for 2010 (green), 2011 (red) and 2012 (blue) data-taking.}
  \label{FIGURE:ExperimentalApparatus_CMS_IntegratedLumi_pp_2010-2012}
\end{figure}

For actual physics usage data needs to undergo the process of certification. In this process specialist from each \gls{CMS} subsystem check that no problem has happened during data taking that would bias or invalidate the recorded events. For 2011 a total of $5.1\,\femto\barn^{-1}$ and for 2012 a total $19.7\,\femto\barn^{-1}$ were considered of good quality for physics. 

In order to achieve high integrated luminosity the \gls{LHC} collides particle bunches 40 millions times a second, each time this happens many interactions may happen simultaneously, this effect is called \gls{PU}. A figure of the distribution of the mean number of interaction per bunch crossing during 2012 at the CMS experiment can be found in figure \ref{FIGURE:ExperimentalApparatus_CMS_PileIp_pp_2012}.

\begin{figure}[!htb]
  \centering
  \includegraphics[width=0.60\textwidth]{Chapter02/CMS/Images/CMS_PileIp_pp_2012}
  \caption{Mean number of interactions per bunch crossing at the CMS experiment during 2012.}
  \label{FIGURE:ExperimentalApparatus_CMS_PileIp_pp_2012}
\end{figure}

%%%%%%%%%%%%%%%%%%%%%%%%%%%%%%%%%%%%%%%%%%%%%%%%%%%%%%%%%%%%%%%%%%%%%%%%%%%%%%%%%%%%%%%
%%% SECTION
%%%%%%%%%%%%%%%%%%%%%%%%%%%%%%%%%%%%%%%%%%%%%%%%%%%%%%%%%%%%%%%%%%%%%%%%%%%%%%%%%%%%%%%
\section{The Compact Muon Solenoid Experiment}
\label{SECTION:ExperimentalApparatus_CMS}

%Status: Done (needs review)

The \glsreset{CMS} experiment is a general purpose experiment located at point 5 of the \gls{LHC}. It was designed to be a high performance detector to study collisions at its centre. It is composed os several sub-systems in an classic onion shaped structure. A diagram of the experiment can be found in figure \ref{FIGURE:ExperimentalApparatus_CMS_Layout_Diagram}.

\begin{figure}[!htb]
  \centering
  \includegraphics[width=1.00\textwidth]{Chapter02/CMS/Images/CMS_Layout_Diagram.pdf}
  \caption{Diagram of \gls{CMS} experiment showing the experiment in an open configuration and highlighting the position of its sub-detectors.}
  \label{FIGURE:ExperimentalApparatus_CMS_Layout_Diagram}
\end{figure}

The main driving motivation for its design is to investigate the electroweak symmetry breaking for which the Higgs mechanism at the design time was presumed to be the most likely explanation. Many other alternative theories to the standard model predict new particles which could be observed at the $\TeV$ scale, \gls{CMS} as a multi-purpose experiment is well suited to search for new scenarios. If found, such new physics may allow us to understand some currently open particle physics issues, like providing particle candidates for dark dark matter. Further, some of this possible new physics signals could point the way towards a grand unified theory. \gls{CMS} is also capable of operating while the \gls{LHC} is colliding heavy ions and has a rich program covering the study of \gls{QCD} matter at extreme temperatures, density and parton momentum fraction (low-x).

The requirements imposed to \gls{CMS} design to meet its physics goals can be summarized in the following table\cite{ARTICLE:TheCMSExperiment}:

\begin{itemize}
  \item Good muon identification and momentum resolution over a wide range of momenta and angles, good dimuon mass resolution ($\approx 1\%$ at $100\,GeV$), and the ability to determine un-ambiguously the charge of muons with $\pt<1\,\TeV$;
  \item Good charged-particle momentum resolution and reconstruction efficiency in the inner tracker. Efficient triggering and offline tagging of $\tau$'s and b-jets, requiring pixel detectors close to the interaction region;
  \item Good electromagnetic energy resolution, good diphoton and dielectron mass resolution ($\approx 1\%$ at $100\,\GeV$), wide geometric coverage, $\pi^0$ rejection, and efficient photon and lepton
isolation at high luminosities;
  \item Good missing-transverse-energy and dijet-mass resolution, requiring hadron calorimeters with with a large hermetic geometric coverage and with fine lateral segmentation.
\end{itemize}

%TODO: Maybe re-write this!!!
The chosen detector design fulfils all this requirements. The experiment is compact compared with the other \gls{LHC} experiments being $22\,\meter$ long and $15\,\meter$ in diameter. But, it is the heaviest of the four big detectors at $12500\,\ton$. Its high density is a direct consequence of it producing the highest magnetic field at $4\,\tesla$ and therefore needing more material to serve as a return yoke. On the next section we will go in detail over the features and technologies used.

%%%%%%%%%%%%%%%%%%%%%%%%%%%%%%%%%%%%%%%%%%%%%%%%%%%%%%%%%%%%%%%%%%%%%%%%%%%%%%%%%%%%%%%
%%% SUBSECTION
%%%%%%%%%%%%%%%%%%%%%%%%%%%%%%%%%%%%%%%%%%%%%%%%%%%%%%%%%%%%%%%%%%%%%%%%%%%%%%%%%%%%
\subsection{Geometry and conventions}

% STATUS: DONE 

The adopted coordinate system has it origin in the center of \gls{CMS} where the nominal collision point is located, the y-axis points vertically upwards, and the x-axis points radially inward in the direction of the center of the \gls{LHC}. The z-axis points along the beam line towards the Jura mountains from the \gls{LHC} point 5. The azimuthal angle $\phi$ is measured from the x-axis in the x-y plane. The polar angle $\theta$ is measured from the z-axis.

We define pseudorapidity as $\eta = -ln(tan(\theta/2))$. All transverse quantities, like the transverse momentum ($p_\perp$), are measured relative to the beam direction. The imbalance of energy is also measured in the x-y plane and is denoted as $E^{miss}_\perp$.

%%%%%%%%%%%%%%%%%%%%%%%%%%%%%%%%%%%%%%%%%%%%%%%%%%%%%%%%%%%%%%%%%%%%%%%%%%%%%%%%%%%%%%%
%%% SUBSECTION
%%%%%%%%%%%%%%%%%%%%%%%%%%%%%%%%%%%%%%%%%%%%%%%%%%%%%%%%%%%%%%%%%%%%%%%%%%%%%%%%%%%%
\subsection{Inner tracking system}
\label{SUBSECTION:ExperimentalApparatus_CMS_Tracker}

% STATUS: DONE (reviwed J.Pela x1)
%
% Writing points:
% * Closest detector to beam, measures trajectories of charged particles
% * With magnetic field measures momentum and charge of this particles
% * Allows vertexing (primary and secundary)
% * Different regions different occupancy
% * Final arranagement

The inner tracking system is the closest detector to the beam axis and the interaction region. Its function is to measure the trajectory of all charged particles, like electrons, charged hadrons and muons with momentum above 1 $\GeV$ being produced at each \gls{LHC} collision. With the help of the strong magnetic field produced by the \gls{CMS} magnet particle trajectories are bent allowing for charge and momentum determination. With the resulting tracks is it then possible to determine the primary vertex as well as secondary vertexes like other lower energy proton-proton collision or displaced vertexes from the decay of long lived particles like B mesons.

Building a tracking system for an experiment at the \gls{LHC} is very challenging. At design luminosity an average of 1000 particles will hit such system at a rate approaching $40\,\mega\hertz$, leading to high hit density at high rate. It is therefore desirable to have a fast, efficient and high granularity detector where at each layer the occupancy should be at or below $1\%$. On the other hand each layer should be as thin as possible in order to not change the incoming particles trajectory or make them lose too much energy. The detector should also be radiation hard and survive for a period of at least 10 years due to its importance and location. This design requirements have lead to a tracker design entirely based on silicon detector technology. 

The volume near the interaction point can be split according to the charged particle flux into three regions:

\begin{itemize}
  \item $r<10\,\centi\meter$: highest particle flux, up to $\approx 10^8 \centi\meter^{-2}\second^{-1}$ at $r \approx 4 \centi\meter$, pixel detectors are used. The pixel size is $\approx 100 \times 150\,\micro\meter^2$, which translates into an occupancy of $10^{-4}$ per \gls{LHC} bunch crossing.
  \item $20<r<55\,\centi\meter$: particle flux decreases enough to use silicon micro-strips with a minimum cell size of 10 cm $\times 80\,\micro\meter$, leading to an occupancy of $\approx 2-3\%$ per \gls{LHC} bunch crossing.
  \item $50<r<110\,\centi\meter$: most outer region of the tracker, particle flux is low enough to use larger pitch silicon micro-strips. The maximum cell size is of $25\,\centi\meter$ $\times$ 180 $\micro\meter$, and occupancy is of the order of $\approx 1\%$.
\end{itemize}

The \gls{CMS} tracker final configuration is composed of a pixel detector with three barrel layers at radii between $4.4\,\centi\meter$ and $10.2\,\centi\meter$ and 2 disks on each side of the barrel. And a silicon strip tracker with 10 barrel detection layers extending up to $1.1\,\meter$ with 3 plus 9 disks on each side of the barrel. A schematic of the detector module distribution can be found at figure \ref{FIGURE:ExperimentalApparatus_CMS_Tracker_Layout}. This detector has an acceptance covering up to pseudorapidity of $|\eta|<2.5$ and has a total active area of about $200\,\meter^2$ making the largest silicon tracker ever built. 

\begin{figure}[!htb]
  \centering
  \includegraphics[width=1.0\textwidth]{Chapter02/CMS/Images/CMS_Tracker_Layout.png}
  \caption{Schematic cross section of the CMS tracker. Each line represent a detector module. Double lines represent dual surface back-to-back detector modules.}
  \label{FIGURE:ExperimentalApparatus_CMS_Tracker_Layout}
\end{figure}

%%%%%%%%%%%%%%%%%%%%%%%%%%%%%%%%%%%%%%%%%%%%%%%%%%%%%%%%%%%%%%%%%%%%%%%%%%%%%%%%%%%%%%%
%%% SUBSECTION
%%%%%%%%%%%%%%%%%%%%%%%%%%%%%%%%%%%%%%%%%%%%%%%%%%%%%%%%%%%%%%%%%%%%%%%%%%%%%%%%%%%%
\subsection{Electromagnetic Calorimeter}
\label{SUBSECTION:ExperimentalApparatus_CMS_ECAL}

% Status: DONE (just with MSc input, and needs review)

The \gls{ECAL} is the detector responsible for measuring the energy of electrons and photons. It is an hermetic energy measurement system comprised of 61200 lead tungstate ($PbWO_4$) crystals mounted in the barrel and 7324 crystals in each of the 2 endcaps and it has an acceptance up to $|\eta|<3.0$.

Lead tungstate has a fairly high density ($8.28\,\gram/\centi\meter^3$), has a short radiation length ($0.89\,\centi\meter$) and a small Moliere redius ($2.2\,\centi\meter$). The crystals also have a fast scintillation decay time emitting 80\% of the light yield in $25\,\nano\second$ (the minimal bunch crossing time at the \gls{LHC}). This characteristics make it a good choice for an electromagnetic calorimeter allowing a compact design with fine granularity. However, this crystals emit a fairly low light yield ($30\,\gamma/\MeV$) which requires the use of photo-detectors with intrinsic gain which will preform well inside a magnitic field. In the barrel region silicon \gls{APD} are used and \gls{VPT} are used in the endcaps. To guarantee good response from both crystals and \gls{APD} it is necessary to have system thermal stability, with the goal being temperature variation of less than $0.1 \celsius$.

The barrel section, the \gls{EB}, has an inner radius of $129\,\centi\meter$ and is composed of 36 identical ``supermodules``, each covers the barrel length and corresponding to a pseudo-rapidity interval of $0<|\eta|<1.479$. The crystals are quasi-projective (the axes are tilted at 3º with respect to the line from the nominal vertex position) and cover 0.0174 (i.e. 1º ) in $\Delta\phi$ and $\Delta\eta$. The crystals have a front face cross-section of $\approx 22 \times 22\,\milli\meter^2$ and a length of $230\,\milli\meter$, corresponding to 25.8 $X_0$.

The endcap section, the \gls{EE}, is at a distance of $314\,\centi\meter$ from the vertex and covering a pseudorapidity range of $1.479<|\eta|<3.0$, are each structured as 2 “Dees”, consisting of semi-circular aluminium plates from which are cantilevered structural units of $5\times 5$ crystals, known as “supercrystals”.

figure \ref{FIGURE:ExperimentalApparatus_CMS_ECAL_Layout}

% \begin{figure}[!htb]
%   \centering
%   \includegraphics[width=1.0\textwidth]{Chapter02/CMS/Images/CMS_ECAL_Layout.png}
%   \caption{}
%   \label{FIGURE:ExperimentalApparatus_CMS_ECAL_Layout}
% \end{figure}
% 
% The energy resolution of the ECAL can be parameterized as:
% \begin{equation}
% \frac{\sigma}{E} = \frac{S}{\sqrt{E}} \oplus \frac{N}{E} \oplus C
% \end{equation}
% where $E$ is the energy of the incident particle and $S$, $N$ and $C$ are known as the stochastic, noise and constant terms respectively. The stochastic term encapsulates fluctuations in the scintillation and lateral containment of the shower; the noise term originates from the electronics and digitisation; and the constant term from non-uniform longitudinal response and inter-calibration errors. These have been measured in an electron beam test as $S=0.028\,\GeV^{1/2}$, $N=0.12\,\GeV$ and $C=0.003$, although without the presence of a magnetic field or material in front of the \ac{ECAL}.

%%%%%%%%%%%%%%%%%%%%%%%%%%%%%%%%%%%%%%%%%%%%%%%%%%%%%%%%%%%%%%%%%%%%%%%%%%%%%%%%%%%%%%%
%%% SUBSECTION
%%%%%%%%%%%%%%%%%%%%%%%%%%%%%%%%%%%%%%%%%%%%%%%%%%%%%%%%%%%%%%%%%%%%%%%%%%%%%%%%%%%%
\subsection{Hadronic Calorimeter}
\label{SUBSECTION:ExperimentalApparatus_CMS_HCAL}

% STATUS: DONE

The \glsreset{HCAL} is a sampling calorimeter which is designed to measure the properties of hadron jets and indirectly neutrinos or other undiscovered particles that would result in apparent missing energy\cite{ARTICLE:CMSTechnicalProposal}. The design of the \gls{HCAL} was strongly influenced by the choice of the magnet parameters since most of the calorimetry is inside of the magnet. A diagram of the \gls{HCAL} subsystems and their location inside \gls{CMS} can be found in figure \ref{FIGURE:ExperimentalApparatus_CMS_HCAL_Layout}.

\begin{figure}[!htb]
  \centering
  \includegraphics[width=1.0\textwidth]{Chapter02/CMS/Images/CMS_HCAL_Layout.png}
  \caption{Longitudinal view of the CMS detector highlighting the location of the \gls{HCAL} components: \gls{HB}, \gls{HE} \gls{HO} and \gls{HF}.}
  \label{FIGURE:ExperimentalApparatus_CMS_HCAL_Layout}
\end{figure}

The \glsreset{HB} covers the region up to $|\eta|<1.3$ and is limited from the beam side by the \gls{ECAL} at radius $r=1.77\,\meter$ and outwards by the magnet at radius $r=2.95\,\meter$. This is a strict limitation on the amount of absorber material to be used. This detector is composed of 36 identical azimuthal wedges with for two half-barrels. They are constructed of brass absorber plates alternated with plastic scintillator. Brass has a short interaction length ($X_0=16.42\,\centi\meter$) and ins non-magnetic. The detector is composed of 2304 towers with a segmentation of $\Delta\eta \times \Delta\phi = 0.087 \times 0.087$ which corresponds to the same area of a $5 \times 5$ arrays of \gls{ECAL} crystals.

To improve the measurement capability, an outer calorimeter, the \glsreset{HO}, is placed outside of the magnet as a \textit{tail catcher}. It increases the effective thickness of the hadronic calorimeter by over 10 interaction lengths. This detector covers the range $|\eta|<1.26$, it is composed or iron absorber and scintillator and is subdivided into sectors that cover 30º azimuthal angle in each of the barrel wheels. 

The \glsreset{HE} covers the range of $1.3<|\eta|<3.0$. It is composed by 2034 towers with a 14 towers segmentation in $\eta$ and 5º segmentation in $\phi$. The 8 inner most towers the segmentation is 10º in $\phi$, whilst the $\eta$ segmentation increases in $\eta$ from 0.09 to 0.35.

Additionally, to extend acceptance to $|\eta|<5.2$ the \gls{HF} is installed at $11.2\,\meter$ from the interaction point providing excellent hermeticity for $E_{\perp}^{miss}$ measurement. Its steel absorber is $1.65\,\meter$ deep and had quartz fibers running through it parallel to the beam line. The energy measurement is made via Cerenkov light produced by the incoming particles inside the fibers. There are 13 tower in $\eta$ with segmentation of $\approx \Delta\eta=0.175$ except the lowest $\eta$ tower with $\approx \Delta\eta=0.1$ and highest $\eta$ tower with $\approx \Delta\eta=0.3$. The segmentation in $\phi$ is of $\Delta\phi=10º$ except in the highest $\eta$ towers which is $\Delta\phi=20º$. There are a total of tower 900 per \gls{HF} module. 

% The HCAL is designed to provide good energy resolution for the measurement of hadronic jets, with single charged pion resolution measured in a test beam \cite{Abdullin:2009zz} and found to be approximately
% \begin{equation}
% \frac{\sigma}{E} = \frac{94.3\%}{\sqrt{E}} \oplus 8.4\%.
% \end{equation}
% The hermetic design and shower containment of the HCAL is driven by the need for accurate measurement of the transverse energy balance in an event and to ensure unambiguous identification of muons by minimising hadronic punch-through into the muon chambers.


%%%%%%%%%%%%%%%%%%%%%%%%%%%%%%%%%%%%%%%%%%%%%%%%%%%%%%%%%%%%%%%%%%%%%%%%%%%%%%%%%%%%%%%
%%% SUBSECTION
%%%%%%%%%%%%%%%%%%%%%%%%%%%%%%%%%%%%%%%%%%%%%%%%%%%%%%%%%%%%%%%%%%%%%%%%%%%%%%%%%%%%
\subsection{Solenoid Magnet}
\label{SUBSECTION:ExperimentalApparatus_CMS_Magnet}

%Status: DONE

The design requirements for correct charge assignment and \pt determination for charge particles and specially muons imply drive the magnet parameters choice. For muons, unambiguously charge determination requires momentum resolution of $\Delta p/p \approx 10\%$ at $p = 1 \TeV$. This requirements are specially difficult to obtain in the forward regions but with the correct length/radius ratio can be obtained with a modestly sized solenoid magnet but with large field\cite{CMSTDR:CMSUpgradeTDR}.

The choice of the \gls{CMS} collaboration was to build a Niobium-Titanium (NbTi) superconducting solenoid magnet which has been design to operate at fields up to $4\,\tesla$ it has a diameter of $6\,\meter$ and a length of $12.5\,\meter$ at maximum field the stored energy reaches $2.7\,\giga\joule$. Typically, the magnet is only run at $3.8\,\tesla$ in order to maximize its lifetime. To contain such an enormous magnetic flux $10\,\kilo\ton$ return yoke envelopes the magnet with 5 wheels in the barrel region and 2 endcaps composed of 3 disks closing the sides\cite{ARTICLE:TheCMSExperiment}. A summary of the most important magnet parameters can be found at table \ref{TABLE:ExperimentalApparatus_CMSMagnetParameters}.

\begin{table}[!htb]
  \centering
  \begin{tabular}{|l|c|}
  \hline
  Parameter & Value \\
  \hline\hline
  Field           & 4 T \\
  Inner Bore      & 5.9 m \\
  Length          & 12.9 m \\
  Number of turns & 2168 \\
  Current         & 19.5 kA \\
  Stored Energy   & 2.7 GJ \\
  Hoop Stress     & 64 atm \\
  \hline
  \end{tabular}
  \caption[Parameters of the CMS superconducting solenoid]{Parameters of the CMS superconducting solenoid}
  \label{TABLE:ExperimentalApparatus_CMSMagnetParameters}
\end{table}


%%%%%%%%%%%%%%%%%%%%%%%%%%%%%%%%%%%%%%%%%%%%%%%%%%%%%%%%%%%%%%%%%%%%%%%%%%%%%%%%%%%%%%%
%%% SUBSECTION
%%%%%%%%%%%%%%%%%%%%%%%%%%%%%%%%%%%%%%%%%%%%%%%%%%%%%%%%%%%%%%%%%%%%%%%%%%%%%%%%%%%%
\subsection{Muon System}
\label{SUBSECTION:ExperimentalApparatus_CMS_Moun}

%Status: DONE (needs review)

The muon detection is an important part of the mission of \gls{CMS} as the middle name of the experiment indicates. Muons are fairly easy to detect when compared with other elementary particles and are only rarely produced in proton-proton collisions. Lets take the example of the \gls{SM} Higgs boson, while the decay mode involving a pair of Z bosons is fairly unlikely compared with other decays the Z bosons can decay into 4 mouns. This decay while rare does not have significant backgrounds making it a ''golden channel`` for discovery, which indeed was proven the case. Many other models, like SUSY, use muon final states in their searches exactly for the same reason. The \gls{CMS} muon system is composed of 3 types of gaseous detectors depending on they location and momentum reconstruction needs. A diagram of the disposition of this system inside \gls{CMS} can be found on figure \ref{FIGURE:ExperimentalApparatus_CMS_Muon_Layout}.

\begin{figure}[!htb]
  \centering
  \includegraphics{Chapter02/CMS/Images/CMS_Muon_Layout.png}
  \caption{Diagram of the \gls{CMS} muon systems. The location of each muon chamber for each subsystem is showed.}
  \label{FIGURE:ExperimentalApparatus_CMS_Muon_Layout}
\end{figure}

In the barrel and up to $|\eta|<1.2$, \gls{DT} are used. since the neutron background is small and the field is constant. This system is composed 250 chambers and is arranged in 4 concentric cylindrical layers which are installed inside of the return yoke. This chambers have a total of 172000 wires with a length of $2.4\,\meter$ which are housed inside of tubes filled with a mixture of argon and carbon-dioxide. Each of the wheels of the barrel is split into 12 sectors covering an angle of 30º azimuthal angle. The maximum gas ionization drift is of $2.0 \centi\meter$ and results in a single point resolution is $\approx 200\,\micro\meter$ per wire. For each station for each measured muon the $\phi$ resolution is better than $200\,\micro\meter$ and direction resolution is $\approx 1\,\milli\radian$.

In the endcaps the region between $2.4>|\eta|>0.9$, \gls{CSC} are used. In the region the muon and background rates are high and the magnetic field is not uniform. This systems have fast response and are radiation resistant and is composed by 468 chambers arranged in 4 stations per side. Each chamber is trapezoidal in shape and made of 6 gas gaps and covers either 10º or 20º in $\phi$. Each gap contains a plane of cathode strips and a plane of anode wires. For each chamber the spacial resolution is of the order of $200\,\micro\meter$ and the angular resolution is $\approx 10\,\milli\radian$ in $\phi$.

Finally the \gls{RPC} cover the $|\eta|<1.6$ range. This system overlaps with the 2 other muon systems. This system is very fast with an ionization event being much faster than the bunch crossing time. This fast response allows in conjunction with a dedicated trigger system to select the correct bunch crossing associated with the detection of a muon. In the barrel there 480 rectangular chambers arranged in 4 stations with 6 \gls{RPC} layers (2 layers are present in the 2 stations closeste to the beam pipe). In the endcaps there are 3 \gls{RPC} disk shaped stations on each side which are composed by trapezoidal shaped detectors.

The combined muon system offline momentum resolution is of the order of 9\% for small values of $\eta$ and $p$ for transverse momenta of up to $200\,\GeV$. At higher energies of around $1\,\TeV$ the standalone momentum resolution is in the range of 15-40\% depending on $|\eta|$. This values are limited by the muon multiple-scattering before arriving to the muon system. If we combine the tracker information into a global fit the resolution for lower \pt tracks improves an order of magnitude while higher momenta (around $1\,\TeV$) it is of about 5\% which is well inside the \gls{CMS} design requirements.

%%%%%%%%%%%%%%%%%%%%%%%%%%%%%%%%%%%%%%%%%%%%%%%%%%%%%%%%%%%%%%%%%%%%%%%%%%%%%%%%%%%%%%%
%%% SUBSECTION
%%%%%%%%%%%%%%%%%%%%%%%%%%%%%%%%%%%%%%%%%%%%%%%%%%%%%%%%%%%%%%%%%%%%%%%%%%%%%%%%%%%%
\subsection{Data Acquisition System}
\label{SUBSECTION:ExperimentalApparatus_CMS_DAQ}

%Status: DONE (needs review)

The \gls{CMS} \gls{DAQ} system is designed to process, analyses and ultimately store the information collected by the detector. The \gls{LHC} produces bunch crossings at a rate of $40\,\mega\hertz$ but we are only capable of storing between $10^2-10^3$ events per second. At design luminosity each collision will have an average of 20 simultaneous collisions. Each collision will produce a zero-suppressed data payload of around $1\,\mega\byte$. To reduce the initial incoming rate a first level of trigger was designed in order to reduce the amount of events to be processed to a maximum of $100 \kilo\hertz$. Even with this event suppression the \gls{DAQ} will have to handle a $\approx 100 \giga\byte\second^{-1}$ which come from approximately 650 data sources. The information is collected and passed to a computer farm where a software filter is installed which serves as a second level of trigger, and is known as the \gls{HLT}. In this system the event rate is further reduced by a factor of 1000 and the resulting events are saved into permanent storage. A diagram of this system can be found on figure TODO:XXX.

%TODO: Image of the DAQ arragement

%%%%%%%%%%%%%%%%%%%%%%%%%%%%%%%%%%%%%%%%%%%%%%%%%%%%%%%%%%%%%%%%%%%%%%%%%%%%%%%%%%%%%%%
%%% SUBSECTION
%%%%%%%%%%%%%%%%%%%%%%%%%%%%%%%%%%%%%%%%%%%%%%%%%%%%%%%%%%%%%%%%%%%%%%%%%%%%%%%%%%%%
\subsection{Trigger System}
\label{SUBSECTION:ExperimentalApparatus_CMS_Trigger}

The \gls{L1T} and \glsreset{HLT}

CMS Tridas TDR\cite{CMSTDR:CMSTridasTDR} 

\begin{figure}[!htb]
  \centering
  \includegraphics{Chapter02/CMS/Images/CMS_L1T_Layout.png}
  \caption{TODO}
  \label{FIGURE:ExperimentalApparatus_CMS_L1T_Layout}
\end{figure}

%%%%%%%%%%%%%%%%%%%%%%%%%%%%%%%%%%%%%%%%%%%%%%%%%%%%%%%%%%%%%%%%%%%%%%%%%%%%%%%%%%%%%%%
%%% SUBSECTION
%%%%%%%%%%%%%%%%%%%%%%%%%%%%%%%%%%%%%%%%%%%%%%%%%%%%%%%%%%%%%%%%%%%%%%%%%%%%%%%%%%%%
\subsection{Computing}
\label{SUBSECTION:ExperimentalApparatus_CMS_Computing}

%Status: DONE

The quantity of data produced by \gls{LHC} and the processing necessary are so big that it would be difficult to have all computing resources in a single place. For this reason a tiered system was developed all computing sites are connected and have specific roles and responsibilities in the data taking and processing.

The \gls{CERN} Data Centre is the tier 0 of this network, know as the Grid, All data produced by the \gls{LHC} passes through it. Only about 20\% of the total capacity of the Grid is hosted here, but \gls{CERN} had the very important mission of safe keeping all the raw data produced by the \gls{LHC} experiments. It also has the task of doing the first attempt at reconstructing this data into meaningful physics objects. 

There are 13 tier 1 computer centres around the world. They are responsible to store a proportional amount of raw and reconstructed data among them. If any reprocessing of the data is needed, this centres are responsible for this task and storing the resulting output as well. Tier 1 centres also host simulated data and distribute their data to affiliated tier 2 centres. 

Local research centres like universities of scientific laboratories are normally at the tier 2 level. They should have enough computing power and storage space to the analysis those centres are involved. This centres will have the responsibility of handling a proportional share of simulated data production and reconstruction. Currently there are over 150 tier 2 centres around the world

Individual computers or local clusters without any formal engagement with the Grid structure, are at the so called tier 3 level.

% The \gls{DQM} 

%%%%%%%%%%%%%%%%%%%%%%%%%%%%%%%%%%%%%%%%%%%%%%%%%%%%%%%%%%%%%%%%%%%%%%%%%%%%%%%%%%%%%%%
%%% SUBSECTION
%%%%%%%%%%%%%%%%%%%%%%%%%%%%%%%%%%%%%%%%%%%%%%%%%%%%%%%%%%%%%%%%%%%%%%%%%%%%%%%%%%%%
\subsection{Run II Upgrades}
\label{SUBSECTION:ExperimentalApparatus_CMS_RUNII}

%Status: Writing

An extensive upgrade program for the \gls{L1T} electronics was planed and is being executed in order to cope with the increase of luminosity and pile-up predicted for the period after \gls{LS1}\cite{CMSTDR:CMSL1Upgrade}. It is expected that center-of-mass energy will almost double from $8\,\TeV$ to $13\,\TeV$, instantaneous luminosity will also increase as will average pile-up. With the change from bunch separation from $50\,\nano\second$ to $25\,\nano\second$ out-of-time pile-up will also become a significant problem. 

To ensure physics performance during 2015 only a partial upgrade is planned for the 2015 run, which is known at the \textit{Stage-1} upgrade system. 

