\chapter{Experimental Apparatus}
\label{CHAPTER:ExperimentalApparatus}

\section{The Large Hadron Collider}
\label{SECTION:ExperimentalApparatus_LHC}

\colorbox{red}{
\begin{minipage}{0.95\linewidth}
TODO: 

\begin{itemize}
  \item DONE: LHC location, size, particles used, energy usage.
  \item Basics of machine and operation
  \item How instantaneous luminosity is calculated include Instantaneous luminosity equation
  \item Delivered instantaneous luminosity Run I (proton-Proton)
\end{itemize}

\end{minipage}
}

The \gls{LHC}\cite{ARTICLE:LHC Machine} is currently the world's largest particle accelerator and is capable to produce the highest energy particle beam ever made by mankind. This gigantic machine with a total perimeter of 26.7 kilometer was built at \gls{CERN} in a circular tunnel at an average depth of 100 meters below ground under the Franco-Swiss border near Geneva, Switzerland. I Diagram of the LHC tunnel can be found in figure \ref{FIGURE:ExperimentalApparatus_LHCLayoutUnderground}.

\begin{figure}[!htb]
  \centering
  \includegraphics[width=0.50\textwidth]{Chapter02/LHC/Images/LHC_layout_underground.jpg}
  \caption{Underground diagram of the Geneva area showing the \gls{LHC} location.}
  \label{FIGURE:ExperimentalApparatus_LHCLayoutUnderground}
\end{figure}

The \gls{LHC} is a synchrotron machine with the capability to accelerate particles in two separated beam pipes with travel in opposite direction.  These beams only cross and are allowed to collide in four specific points of the accelerator where huge particle detectors are installed to detect the products of such collisions. This experiments are name ATLAS\cite{ARTICLE:TheATLASExperiment}, CMS\cite{ARTICLE:TheCMSExperiment}, LHCb\cite{ARTICLE:TheLHCbExperiment} and ALICE\cite{ARTICLE:TheALICEExperiment}.

The accelerator as its name indicates can collide hadrons, more specifically proton or heavy ions. Up to now 3 modes of operation have been tried according to the particles being collided: proton-proton, proton-lead and lead-lead. Depending on the which configuration is chosen we are basically changing the quantity of nucleons available to each colliding element. The maximum design energy per proton is 7 TeV and for each lead nucleon 2.76 TeV. The design luminosity for proton-proton is of $10^{34}$ $cm^{-2}s^{-1}$ and for lead-lead is of $10^{27}$ $cm^{-2}s^{-1}$.

The \gls{LHC} is only the last element of a complex accelerator chain which step by step increases the energy of the particles. Protons are initially obtained by stipping the electrons of hydrogen gas. The are then accelerated at the \gls{LINAC2} up to the energy of 50 MeV. After this initial step they are injected into the \gls{PSB} and the energy ramps ups to 1.4 GeV. After protons are passed to the \gls{PS} where energy futher increases to 25 GeV subsequently they are the injected into the \gls{SPS} where the particle energy level reached 450 GeV. Finally, protons pass to the \gls{LHC} where they can be accelerated to a maximum energy of 7 TeV. A simplified diagram of the \gls{CERN} accelerator chain can be found in figure \ref{FIGURE:ExperimentalApparatus_LHCAccelaratorChain}.

\begin{figure}[!htb]
  \centering
  \includegraphics[width=0.50\textwidth]{Chapter02/LHC/Images/LHCAccelaratorChain.png}
  \caption{CERN Large Hadron Collider Experiment accelerator diagram.}
  \label{FIGURE:ExperimentalApparatus_LHCAccelaratorChain}
\end{figure}

Normal operation of the \gls{LHC} therefore depends on the the upstream accelerators availability. The typically turn around time, the time necessary to stop the accelerator from running and restart collisions is around 2 hours. When stable beams are achieved, a single proton fill can be used to collide protons up to 24 hours, but it is common to restart more frequently to take profit of the higher collision rates possible right at the beginning of a new fill.

Some of the key parameters of the LHC proton-proton and lead-lead operation can be found in table \ref{TABLE:ExperimentalApparatus_LHCMachineParameters}.

\begin{table}[!htb]
  \centering
  \begin{threeparttable}
    \begin{tabular}{|lcccc|}
    \hline 
                                  &              &           \textit{pp} &         \textbf{HI} &  \\
    \hline
    Energy per nucleon            & E            &                     7 &                2.76 &                 $\TeV$ \\
    Dipole field at 7 TeV         & \textit{B}   &                  8.33 &                8.33 &               $\tesla$ \\
    Design Luminosity\tnote{*}    & $\mathcal{L}$ &            $10^{34}$ &           $10^{27}$ & $\cm^{-2}\second^{-1}$ \\
    Bunch separation              &              &                    25 &                 100 &                  $\ns$ \\
    No. of bunches                & $k_B$        &                  2808 &                 592 &                        \\
    No. particles per bunch       & $N_p$        & $1.15 \times 10^{11}$ & $7.0 \times 10^{7}$ &                        \\
    \hline
    \hline
    \textbf{Collisions}           &              &  &  &  \\
    \hline
    $\beta$-value at IP           & $\beta^{*}$  &                  0.55 &                 0.5 &        $\meter$ \\
    RMS beam radius at IP         & $\sigma^{*}$ &                  16.7 &                15.9 &  $\micro\meter$ \\
    Luminosity lifetime           & $\tau_L$     &                    15 &                   6 &         $\hour$ \\
    Number of collisions/crossing & $n_c$        &          $\approx 20$ &                   - &                 \\
    \hline
    \end{tabular}
    \begin{tablenotes}
      \item[*] For heavy-ion (HI) operation the design luminosity for Pb-Pb collisions is given.
    \end{tablenotes}
  \end{threeparttable}
  \caption[LHC parameters relevant for detectors]{The machine parameters relevant for the 
                                                  LHC detectors.\cite{CMSTDR:CMSPhysicsVol1}}
  \label{TABLE:ExperimentalApparatus_LHCMachineParameters}
\end{table}


\begin{figure}[!htb]
  \centering
  \includegraphics[width=0.50\textwidth]{Chapter02/LHC/Images/crosssections2012_v5}
  \caption{Cross sections for several processes for collisions of antiproton-proton and proton-proton as a function of the center of mass energy\cite{ARTICLE:TheCMSCollaboration}.}
  \label{FIGURE:ExperimentalApparatus_LHCCrossSections}
\end{figure}

At the \gls{LHC} we are looking for extremely rare processes as is can be seen in figure \ref{FIGURE:ExperimentalApparatus_LHCCrossSections} the production cross section of a \gls{SM} Higgs boson is more than 9 orders of magnitude smaller than the total proton-proton cross section. 

To be able to record and study such rare processes we need to produce a significant amount of collisions. For this purpose the LHC was designed to operate at high instantaneous luminosity, L. This quantity is defined as,

\begin{equation}
L=\frac{N_{b}^{2}n_{b}f_{\text{rev}}\gamma}{4\pi\epsilon_{n}\beta^{*}}F,
\end{equation}

where $N_{b}$ is the number of protons per bunch, $n_{b}$ is the number of bunches, $f_{\text{rev}}$ is the frequency of revolution, $\gamma$ is the Lorentz factor, $\epsilon_{n}$ is the normalized emittance, $f_{\text{rev}}$ is the beta function at the collision point and $F$ is the reduction factor due to the crossing angle.

\section{The Compact Muon Solenoid Experiment}
\label{SECTION:ExperimentalApparatus_CMS}

The \gls{CMS} experiment is a general purpose experiment located at point 5 of the \gls{LHC}. It was designed to study collisions at its centre and is composed os several sub-systems in an classic onion shaped structure.





\subsection{Inner tracking system}
\label{SUBSECTION:ExperimentalApparatus_CMS_Tracker}

The volume near the interaction point can be split according to the charged particle flux into three regions depending on the charged particle flux.

\begin{itemize}
  \item one
  \item two
  \item three
\end{itemize}


\subsection{Electromagnetic Calorimeter}
\label{SUBSECTION:ExperimentalApparatus_CMS_ECAL}

The \gls{ECAL} 


\subsection{Hadronic Calorimeter}
\label{SUBSECTION:ExperimentalApparatus_CMS_HCAL}

The \gls{HCAL}

\subsection{Solenoid Magnet}
\label{SUBSECTION:ExperimentalApparatus_CMS_Magnet}

\begin{table}[!htb]
  \centering
  \begin{tabular}{|l|c|}
  \hline
  Parameter & Value \\
  \hline\hline
  Field           & 4 T \\
  Inner Bore      & 5.9 m \\
  Length          & 12.9 m \\
  Number of turns & 2168 \\
  Current         & 19.5 kA \\
  Stored Energy   & 2.7 GJ \\
  Hoop Stress     & 64 atm \\
  \hline
  \end{tabular}
  \caption[Parameters of the CMS superconducting solenoid]{Parameters of the CMS superconducting solenoid}
  \label{TABLE:ExperimentalApparatus_CMSMagnetParameters}
\end{table}


\subsection{Muon System}
\label{SUBSECTION:ExperimentalApparatus_CMS_Mouns}

\subsection{Data Acquisition System}
\label{SUBSECTION:ExperimentalApparatus_CMS_DAQ}

The \gls{DAQ}

\subsection{Trigger System}
\label{SUBSECTION:ExperimentalApparatus_CMS_Trigger}

The \gls{L1T} and \gls{HLT}

\subsection{Computing}
\label{SUBSECTION:ExperimentalApparatus_CMS_Computing}

The \gls{DQM} 

\subsection{Run II Updgrades}
\label{SUBSECTION:ExperimentalApparatus_CMS_RUNII}

The upgrade tdr \cite{CMSTDR:CMSL1Upgrade}.
