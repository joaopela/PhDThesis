\chapter{Event Reconstruction and simulation}
\label{CHAPTER:EventReconstructionAndSimulation}

% \glsresetall % Resetting all acronyms

%Status: DONE - (Reviewed x1 J.Pela) (reviewed D. Colling x1)

This chapter describes how the \gls{CMS} detector produces physics objects from the information collected at each event. The \gls{VBF} Higgs to invisible analysis uses almost all the physics objects reconstructed by the detector, making use of information from all the experiment sub-detectors. The following sections describe in detail how each of these objects are reconstructed and what filters are placed upon them. The last section describes how \gls{MC} methods are used to simulate physics processes and emulate the detector response.

%%%%%%%%%%%%%%%%%%%%%%%%%%%%%%%%%%%%%%%%%%%%%%%%%%%%%%%%%%%%%%%%%%%%%%%%%%%%%%%%%%%%%%%
%%% SECTION
%%%%%%%%%%%%%%%%%%%%%%%%%%%%%%%%%%%%%%%%%%%%%%%%%%%%%%%%%%%%%%%%%%%%%%%%%%%%%%%%%%%%%%%
\section{Tracks}
\label{SECTION:EventReconstructionAndSimulation_Tracks}

%Status: DONE - (Reviewed x1 J.Pela) (reviewed D. Colling x1)

Reconstructing the trajectories of charged particles allows us to measure their momentum and determining their charge. This is possible by analysing the hit patterns in the inner tracking system. In \gls{CMS} this reconstruction is made with the \gls{CTF} algorithm~\cite{ARTICLE:CMSTrackReconstruction}. The relevant steps for track generation are described below:

\begin{itemize}
  \item Seed generation is made with hits at the pixel detector. A track seed can be made with two or three hits. In the first case a known vertex or the beam spot is used to constrain the seed momentum. The parameters of each seed are estimated using the assumption that the trajectory is a helix, but it takes into account hit errors and multiple scattering~\cite{ARTICLE:CMSTrackReconstructionSeedGeneration}.
  \item The track seed is extrapolated through the tracker layers with a combinatorial Kalman filter~\cite{ARTICLE:KalmanFilteringTrackVertexFitting}. For each additional layer, the best matching hit, if any, is added and track parameters are recomputed. This procedure continues until the last layer is reached~\cite{ARTICLE:CMSTrackReconstruction}.
  \item Ambiguity resolution may be necessary since it is possible to have the same track being reconstructed from different seeds, or a seed may results in more than a single trajectory candidate. To resolve this possible double counting, when considering a pair of tracks with more than 50\% of shared hits, we discard the one with the fewer hits. In case of equal number of hits, the track with the lowest $\chi^2$ is kept. 
  \item After the track building and cleaning stages are done, final refitting is performed. This procedure is aimed at removing possible bias by constraints at the seed forming stage. A standard Kalman filter and smoother are used.
\end{itemize}

The process of track finding is repeated up to six times, with the hits for each successfully reconstructed track removed for the next iteration. Using early \gls{LHC} data and a dataset of pions and muons, it was possible to estimate that the tracking efficiency is $>98\%$ for all track $\pt > 500\,\MeV$ and $>99\%$ for tracks with $\pt>2\,\GeV$~\cite{ARTICLE:CMSMeasurmentTrackEfficiency}.

%%%%%%%%%%%%%%%%%%%%%%%%%%%%%%%%%%%%%%%%%%%%%%%%%%%%%%%%%%%%%%%%%%%%%%%%%%%%%%%%%%%%%%%
%%% SECTION
%%%%%%%%%%%%%%%%%%%%%%%%%%%%%%%%%%%%%%%%%%%%%%%%%%%%%%%%%%%%%%%%%%%%%%%%%%%%%%%%%%%%%%%
\section{Vertex Reconstruction}
\label{SECTION:EventReconstructionAndSimulation_Vertex}

%Status: DONE - (Reviewed x1 J.Pela) (reviewed D. Colling x1)

The \gls{LHC} can produce extreme collision intensities which are obtained partially by having multiple collisions happening at each bunch crossing. As discussed in section \ref{SUBSECTION:ExperimentalApparatus_CMS_RunningAndPerformance} an average of 21 simultaneous collisions happened per bunch crossing during 2012. In this environment, it is crucial to identify the \gls{PV} and the particles that come from it. This information can then be used to reject particles coming from other additional collisions and to identify displaced vertices which can be the signature of long lived particles like b-hadrons.

The individual tracks are reconstructed making use of the inner tracker. Each vertex is initially seeded by two tracks with separation in \textit{z} less than $1\,\centi\meter$. Then remaining tracks are clustered to the vertex seeds with the \gls{DA} algorithm~\cite{ARTICLE:DeterministicAnnealing}. After the clustering process is done, the position of each vertex is recomputed using the \gls{AVF} algorithm~\cite{ARTICLE:AdaptiveVertexFitting}. In this algorithm, weights $w_{i}$, are assigned to tracks according to how compatible they are with the fitted vertex position. Weights vary from 1 to 0, tracks assigned weights of close 1 are highly compatible with the vertex and close 0 are given to low compatibility tracks. Then we can define the number of degrees of freedom of the new fit as:

\begin{equation}
n_{dof}(vertex)=2\sum\limits_{i}^{tracks} w_i - 3
\end{equation}

This variable can be used to distinguish real proton-proton interactions from misclustered vertices, since it is correlated with the number of tracks compatible with that specific vertex~\cite{ARTICLE:CMSTrackingAndPrimaryVertex}. The vertex position and resolution have been measured with \gls{LHC} data and compared with simulation. The resulting plots can be found in figure \ref{FIGURE:EventReconstructionAndSimulation_Vertex} as a function of the number of tracks.

\begin{figure}[htp]%
\centering
\subfloat[][]{\includegraphics[width=0.45\linewidth]{Chapter04/Vertex/Images/vtx-res.pdf}}\qquad
\subfloat[][]{\includegraphics[width=0.45\linewidth]{Chapter04/Vertex/Images/vtx-eff.pdf}}\\
\caption[Primary vertex resolution in the $z$ coordinate and vertex reconstruction efficiency as a function of the number of constituent tracks.]{(a) Primary vertex resolution in the $z$ coordinate as function of the number of associated tracks. Results are give for three ranges of average track \pt. (b) Primary vertex efficiency as a function of the number of associated track~\cite{ARTICLE:CMSTrackingAndPrimaryVertex}}
\label{FIGURE:EventReconstructionAndSimulation_Vertex}
\end{figure}

The \gls{PV} is defined as the vertex with the highest sum of associated tracks \pt squared. In situations were no vertex can be reconstructed, for example if there is a tracking failure, the beam spot position is assumed. Knowing precisely the interaction point allows us to determine particle candidate quantities relative to it, which allow for better object identification and pile-up control. 

\newpage
Most \gls{CMS} analyses, including the ones presented in this thesis, require explicitly that a good vertex is reconstructed with the following characteristics:

\begin{itemize}
  \item Real reconstructed vertex from tracks, not the beam spot.
  \item A minimum number of degrees of freedom: $n_{dof}>4$.
  \item Collision must be near the interaction region. We require longitudinal distance to be $|z| \leq 24\,\centi\meter$ (longitudinal impact parameter).
  \item Collision must be close to the beam line. Radial distance to beam line: $d_{xy}<2\,\centi\meter$ (transverse impact parameter). 
\end{itemize}

%%%%%%%%%%%%%%%%%%%%%%%%%%%%%%%%%%%%%%%%%%%%%%%%%%%%%%%%%%%%%%%%%%%%%%%%%%%%%%%%%%%%%%%
%%% SECTION
%%%%%%%%%%%%%%%%%%%%%%%%%%%%%%%%%%%%%%%%%%%%%%%%%%%%%%%%%%%%%%%%%%%%%%%%%%%%%%%%%%%%%%%
\section{Particle Flow}
\label{SECTION:EventReconstructionAndSimulation_ParticleFlow}

%Status: DONE - (Reviewed x1 J.Pela) (reviewed D. Colling x1)

The \gls{PF} algorithm~\cite{ARTICLE:CMSComissioningOfParticleFlow, ARTICLE:CMSParticleFlowEventRecontruction, ARTICLE:CMSComissioningOfParticleFlowWithMinBias} is used in the \gls{CMS} experiment with the objective of reconstructing every stable particle produced in the event. This is achieved by combining information from all \gls{CMS} sub-detectors in order to identify electrons, photons, muons, charged hadrons and neutral hadrons and measure their direction, energy and type. The identified particles can in turn be used in jet clustering, determining the missing transverse energy, reconstructing and identifying taus, calculating particle isolation, identifying b-quark jets, etc.

The \gls{CMS} experiment is very well suited to this approach as it is equipped with a high precision silicon tracker, which is immersed in a uniform axial magnetic field and has a dual calorimeter design with high hermeticity and resolution. The tracker system allows very precise direction/momentum reconstruction for charged particles, down to transverse momentum as low as $150\,\MeV$. The high granularity of the \gls{ECAL} allows for photons to be identified through deposit separation even inside high energy jets. In turn electrons can be reconstructed by combining their track and the energy deposits of the electron itself and its emissions, this algorithm will be explained further in section \ref{SECTION:EventReconstructionAndSimulation_Electrons}. The tracker information also allows the separation of charged and neutral hadrons in close proximity, a task which is not possible with just the \gls{HCAL} due to its coarser granularity. We can determine the charged hadron momentum from the track information, and then, by removing its deposit from the calorimeter system we can determine the neutral hadron deposits. In areas outside the tracker and/or \gls{ECAL} coverage, measurements are more coarse since we have less information available.

The clustering is performed separately in the \gls{ECAL} and \gls{HCAL}. We start by identifying \textit{seed clusters} which are local maxima of calorimeter cell energy deposits. We add neighbouring cells into \textit{topological clusters} if their energy deposit is bigger than two standard deviations of the electronics noise. This value was determined to be $80\,\MeV$ for the \gls{ECAL} barrel, up to $300\,\MeV$ for the \gls{ECAL} endcap and $800\,\MeV$ for the \gls{HCAL}. The energy of each cell may be shared between multiple clusters.

Tracks and clusters are \gls{PF} elements that need to be linked together to reconstruct the particle they came from and also to avoid double counting. We pair elements based on a metric of distance between elements and if they are compatible we merge them into \textit{blocks} which can be interpreted as particle candidates. As an example, a pair of a track and energy cluster on the calorimeter system would be linked if you could extrapolate the track to the cluster volume.

%~\cite{ARTICLE:CMSComissioningOfParticleFlow}            AG 73
%~\cite{ARTICLE:CMSParticleFlowEventRecontruction}        AG 74
%~\cite{ARTICLE:CMSComissioningOfParticleFlowWithMinBias} AG 75

%%%%%%%%%%%%%%%%%%%%%%%%%%%%%%%%%%%%%%%%%%%%%%%%%%%%%%%%%%%%%%%%%%%%%%%%%%%%%%%%%%%%%%%
%%% SUBSECTION
%%%%%%%%%%%%%%%%%%%%%%%%%%%%%%%%%%%%%%%%%%%%%%%%%%%%%%%%%%%%%%%%%%%%%%%%%%%%%%%%%%%%%%%
\subsection{Isolation}
\label{SUBSECTION:EventReconstructionAndSimulation_ParticleFlow_LeptonIsolation}

%Status: DONE - (Reviewed x1 J.Pela) (reviewed D. Colling x1)

To reduce the probability of misidentification of a lepton coming from \gls{QCD} jets as opposed to a lepton originating directly from the hard scatter process we can require isolation~\cite{ARTICLE:CMSElectronReconstruction8TeV, ARTICLE:CMSMuonReconstruction7TeV}. We compute the isolation by summing the transverse momenta of all particles inside a cone around the selected lepton. In this sum, we include all charged particles, neutral hadrons and photons. But we do not want to include the \gls{PU} contribution to this sum so we only include the charged candidates with a \gls{PV} impact parameter smaller than $0.1 \,\centi\meter$. Different methods are used for each particle to estimate and subtract the neutral component of the \gls{PU} depending on \gls{POG} recommendations.

Normally, for physics analysis we defined the more meaningful \textit{relative isolation} as $I_{rel} = I/\pt^{lepton}$. By using, a quantity that is relative to the candidate \pt and not an absolute cut we avoid wrongly accepting low energy candidates or rejecting high energy candidates. In the next sections, the steps taken to calculate this quantity for each particle candidate are explained.

%%%%%%%%%%%%%%%%%%%%%%%%%%%%%%%%%%%%%%%%%%%%%%%%%%%%%%%%%%%%%%%%%%%%%%%%%%%%%%%%%%%%%%%
%%% SECTION
%%%%%%%%%%%%%%%%%%%%%%%%%%%%%%%%%%%%%%%%%%%%%%%%%%%%%%%%%%%%%%%%%%%%%%%%%%%%%%%%%%%%%%%
\section{Electrons}
\label{SECTION:EventReconstructionAndSimulation_Electrons}

%Status: DONE - (Reviewed x1 J.Pela) (reviewed D. Colling x1)

In the \gls{CMS} experiment, electrons are reconstructed by matching energy clusters in the \gls{ECAL} with tracks coming from the inner tracking system. Unfortunately, electrons can loose and disperse significant amounts of energy before they reach the \gls{ECAL}. While they go through the inner tracker they may emit photons via bremsstrahlung and in turn these photon can convert to $e^+e^-$ pairs. About 35\% of the electrons radiate at least 70\% of their energy in this way~\cite{ARTICLE:CMSElectronReconstruction}. This spread of energy is mostly in $\phi$ due to the applied magnetic field~\cite{ARTICLE:CMSElectronReconstructionECAL}. Dedicated algorithms were developed to combine the the \gls{ECAL} energy deposits, into a so called \textit{supercluster}, of the initial electron and its emissions.

Different algorithms are used in the barrel and endcap regions. In the barrel region, we explore the simple $\eta-\phi$ geometry with the \textit{hybrid clustering algorithm}~\cite{ARTICLE:CMSElectronReconstruction8TeV}. The procedure starts by identifying \textit{seed crystals} with $E_T>1\,\GeV$. A domino-shaped cluster is formed around this seed in the $\eta$ direction of $3 \times 1$ or $5 \times 1$ crystals centred at the seed. Additional dominoes are added in both $\phi$ directions in an attempt to collect the bremsstrahlung emissions up to $\Delta\phi \approx 0.3\,\radian$. Any domino with energy below $100\,\MeV$ is disregarded. The resulting additional sub-clusters must have their own seed with $E_T>350\,\MeV$ and they are all combined to form the final \textit{supercluster}. 

In the encaps, the \textit{$\text{Multi-}5 \times 5$ algorithm}~\cite{ARTICLE:CMSElectronReconstruction8TeV} is used. In this region of the detector the geometry is more complex and does not follow a simple $\eta-\phi$ symmetry. The seeds for this clustering procedure are the crystals which are local maxima over their four direct neighbours and have a deposit of $E_T>0.18\,\GeV$. Then, and starting with the seeds with highest $E_T$, we collect the energy around them into clusters of $5 \times 5$ crystals. We then search for similar seeds and form clusters that can overlap within $\Delta\eta<0.07$ and $\Delta\phi<0.3\,\radian$ of the initial seed. Those clusters are then combined into a single \textit{supercluster} which needs to have at least $E_T>1\,\GeV$. The \textit{supercluster} is then extrapolated to the \gls{ECAL} preshower by clustering the energy within $\Delta\eta<0.15$ and $\Delta\phi<0.45$ around the most energetic cluster and adding it to the \textit{supercluster} itself~\cite{ARTICLE:CMSElectronReconstruction8TeV}.

In order to reconstruct the electron track, we need to take into account the bremsstrahlung emissions. The \gls{CTF} algorithm, which was described in section \ref{SECTION:EventReconstructionAndSimulation_Tracks}, is not appropriate for this purpose so a different track-finding algorithm had to be developed. For high \pt electrons, we use the \gls{ECAL} supercluster energy deposit weighted mean impact point as a seed. If we combine this information with the determined $E_T$ we can define two $\eta-\phi$ search regions in the pixel detector depending on the charge hypothesis. If we find two compatible hits, the electron trajectory is updated. From this point normal track building is performed but instead of a Kalman filter algorithm we use a \gls{GSF} algorithm~\cite{ARTICLE:CMSReconstructionElectronsGSF}. This method performs better in the presence of non-Gaussian losses like the one coming from the bremsstrahlung emissions.

The typical background to real electrons are collimated hadronic jets, with $\pi^0$ and $\pi^{\pm}$ overlap or from $\pi^{\pm}$ showers~\cite{ARTICLE:CMSElectronReconstruction}. There are many useful variables that may be used to reduce such background and which are often used in \textit{electron identification} criteria:

\begin{itemize}
  \item $\Delta\eta_{in}$ and $\Delta\phi_{in}$, are the distance between the track direction at the vertex and extrapolated to the \gls{ECAL} and supercluster.
  \item $\sigma_{\eta\eta}$ is the energy-weighted $\eta$ width of the cluster. For real prompt electrons this is normally small, since this quantity is not significantly affected by the magnetic field.
  \item $H/E$ is the ration of hadronic to electromagnetic energy in the region of the seed cluster. 
\end{itemize}

Distributions of these variables for simulated electrons and jets can be found in figure \ref{FIGURE:PhysicsObjects_Electrons}. 

\begin{figure}[htp]%
\centering
\subfloat[]{\includegraphics[width=0.45\linewidth]{Chapter04/Electrons/Images/elecs_deta.pdf}}\qquad
\subfloat[]{\includegraphics[width=0.45\linewidth]{Chapter04/Electrons/Images/elecs_dphi.pdf}}\\
\subfloat[]{\includegraphics[width=0.45\linewidth]{Chapter04/Electrons/Images/elecs_hovere.pdf}}\qquad
\subfloat[]{\includegraphics[width=0.45\linewidth]{Chapter04/Electrons/Images/elecs_sigmaietaieta.pdf}}
\caption[Distributions for the variables $\Delta\eta_{\text in}$, $\Delta\phi_{\text in}$, $\sigma_{i\eta i\eta}$ and $H/E$ for simulated electrons and misidentified jets.]{Distributions for (a) $\Delta\eta_{in}$, (b) $\Delta\phi_{in}$, (c) $H/E$ and (d) $\sigma_{\eta\eta}$. Here \textit{golden electrons} are those who emit minimal bremmstrahlung photons, \textit{showering} are electrons that lose a large faction of their energy in emissions and \textit{jets} are the typical distributions for hadronic jets~\cite{ARTICLE:CMSElectronReconstruction}.}
\label{FIGURE:PhysicsObjects_Electrons}
\end{figure}

%%%%%%%%%%%%%%%%%%%%%%%%%%%%%%%%%%%%%%%%%%%%%%%%%%%%%%%%%%%%%%%%%%%%%%%%%%%%%%%%%%%%%%%
%%% SUBSECTION
%%%%%%%%%%%%%%%%%%%%%%%%%%%%%%%%%%%%%%%%%%%%%%%%%%%%%%%%%%%%%%%%%%%%%%%%%%%%%%%%%%%%%%%
\subsection{Isolation}
\label{SUBSECTION:EventReconstructionAndSimulation_LeptonIsolation_Isolation}

% More information
%  * https://twiki.cern.ch/twiki/bin/view/CMS/EgammaCutBasedIdentification#Detector_Isolation
%  * https://twiki.cern.ch/twiki/bin/view/CMS/EgammaEARhoCorrection#Isolation_cone_R_0_3

%Status: DONE - (Reviewed x1 J.Pela) (reviewed D. Colling x1)

For electrons, we calculate isolation with the \textit{effective area-corrected isolation} method over a cone of $\Delta R<0.3$ around the electron. For the neutral \gls{PU} subtraction we use a look-up table of effective areas according to electron $|\eta|$ which is multiplied by the estimated neutral \gls{PU} energy density by unit of effective area. The definition for this isolation can be found in equation \ref{EQUATION:ElectronIsolation}.

\begin{equation}
I = \sum^{\substack{\text{charged} \\ \text{non-pileup}}}\pT +
\text{max}\left(0,\sum^{\text{neutral}}\pT+\sum^{\text{photon}}\pT - \rho(\text{lepton}) \times \text{Eff. Area}(\text{lepton})\right)
\label{EQUATION:ElectronIsolation}
\end{equation}

\subsection{Veto electrons}

%Status: DONE - (Reviewed x1 J.Pela) (reviewed D. Colling x1)

The minimum electron candidate requirements to veto an event if an electron is present is defined as the \textit{veto electron} criteria. For this purpose we required an electron candidate with $\pt>10\,\GeV$ and $|\eta|<2.4$ which passes the \gls{CMS} Electron/Gamma \gls{POG}~\cite{ARTICLE:CMSElectronReconstruction7TeV} requirements of the cut-based electron \gls{ID} \textit{veto electron} working point. A summary of these conditions can be found in table \ref{TABLE:PhysicsObjects_ElectronPOG_CutBased_VetoElectronRequirements}.
 
\input{Chapter04/Electrons/Tables/ElectronPOG_CutBased_VetoElectronRequirements}

\subsection{Tight electrons}

% More information at: 
%  * https://twiki.cern.ch/twiki/bin/viewauth/CMS/EgammaIDRecipes
%  * https://twiki.cern.ch/twiki/bin/view/CMS/EgammaCutBasedIdentification
%  * https://twiki.cern.ch/twiki/bin/view/CMS/EgammaEARhoCorrection

%Status: DONE - (Reviewed x1 J.Pela) (reviewed D. Colling x1)

We define a \textit{tight electron} as an electron candidate with $\pt>20\,\GeV$ and $|\eta|<2.4$ which passes the \gls{CMS} Electron/Gamma \gls{POG} requirements of the cut-based electron \gls{ID} \textit{tight electron} working point. This working point is similar to the 2011 very tight WP70 working point. A summary of these conditions can be found in table \ref{TABLE:PhysicsObjects_ElectronPOG_CutBased_TightElectronRequirements}.

\begin{table}[!htp]
\centering
\begin{tabular}{|l|c|c|}
\hline
Variable  $\pt > 20 (p_{T} <= 20)$ & Barel & Endcap \\
\hline\hline
$| \Delta\eta(track,supercluster) |$                           & $<0.004$ & $<0.005$ \\
$| \Delta\phi(track,supercluster) |$                           & $<0.3  $ & $<0.2  $ \\
$ \sigma(i\eta,i\eta)$                                         & $<0.01 $ & $<0.03 $ \\
$H/E$                                                          & $<0.12 $ & $<0.10 $ \\
\cline{2-3}
$|d_{0}(vertex)|$                                              & \multicolumn{2}{c|}{$<0.02\,\centi\meter$} \\
$|d_{Z}(vertex)|$                                              & \multicolumn{2}{c|}{$<0.1\,\centi\meter$}  \\
$|\frac{1}{E}-\frac{1}{p}| $                                   & \multicolumn{2}{c|}{$<0.05$} \\
\cline{2-3}
$\frac{PF_{isolation}}{p_{\perp}}$ for $ \Delta R_{cone}=0.3$ & $<0.10 $ & $<0.10(0.07)$ \\
\cline{2-3}
Conversion rejection: vertex fit probability                   & \multicolumn{2}{c|}{$<1 \times 10^{6}$}  \\
Conversion rejection: missing hits                             & \multicolumn{2}{c|}{$=0$} \\
\hline
\end{tabular}
\caption{Details of the \gls{CMS} Electron-Gamma \gls{POG} recommendations for a \textit{tight electron}. Here barrel is defined as $ |\eta_{supercluster}|<=1.479 $ and endcap is $ 1.479 < |\eta_{supercluster}| < 2.5 $.}
\label{TABLE:PhysicsObjects_ElectronPOG_CutBased_TightElectronRequirements}
\end{table}


%%%%%%%%%%%%%%%%%%%%%%%%%%%%%%%%%%%%%%%%%%%%%%%%%%%%%%%%%%%%%%%%%%%%%%%%%%%%%%%%%%%%%%%
%%% SECTION
%%%%%%%%%%%%%%%%%%%%%%%%%%%%%%%%%%%%%%%%%%%%%%%%%%%%%%%%%%%%%%%%%%%%%%%%%%%%%%%%%%%%%%%
\section{Muons}
\label{SECTION:EventReconstructionAndSimulation_Muons}

%Status: DONE - (Reviewed x1 J.Pela) (reviewed D. Colling x1)

Muon track reconstruction starts independently at the inner-tracker (\textit{tracker track}) and in the muon systems (\textit{standalone muon track})~\cite{ARTICLE:CMSMuonReconstruction7TeV}. Then this information can be combined into a single muon track in two possible ways.

\textit{Global muon reconstruction} is an \textit{outside-in algorithm}. We start by finding a tracker track match for each standalone muon track. This is done by propagating the match candidate pair to a common surface and comparing track parameters. For each matched pair, a \textit{global-muon fit} is performed using all hits from the two tracks in a Kalman-filter algorithm~\cite{ARTICLE:KalmanFilteringTrackVertexFitting}. For muons of $\pt\gtrsim 200\,\GeV/c$, it has been shown that a \textit{global-muon fit} improves the momentum resolution compared to a \textit{tracker-only fit}~\cite{CMSTDR:CMSPhysicsVol1, ARTICLE:CMSPerformanceMuonReconstructionCosmicRay}.

\textit{Tracker muon reconstruction} is an \textit{inside-out algorithm}. In this method we start by selecting all tracker tracks with $\pt>0.5\,\GeV$ and $p>2.5\,\GeV$. We extrapolate those tracks to the muon system, taking into account the magnetic field, energy loss and scattering. If we find a match with at least one muon segment in the muon system (track stub in the \gls{DT} or \gls{CSC}) this tracker track now becomes a Tracker Muon. 

Tracker muon reconstruction is more efficient than the global muon reconstruction at low momenta ($p\lesssim 5\,\GeV$). This difference is due to tracker muon reconstruction only requiring one segment on the muon system. Global muon reconstruction is more efficient for higher energies, where the muons are more likely to pass several muon stations.

Muons can also be classified as prompt or non-prompt. Prompt muons are the ones produced directly in the hard process, like the decays of vector bosons or quarkonia particle decays. On the other hand, non-prompt muons typically come from in-flight decays of light hadrons, from taus or heavy quark decays. 

The Global muon reconstruction has high purity reconstructing prompt muons, but sometimes hadronic activity can ``punch-through'' the calorimeter system and appear in the muon system, generating fakes. To reduce this type of background we can use different muon identification criteria.

Studies have been performed with the \gls{CMS} detector to measure muon reconstruction efficiency~\cite{ARTICLE:CMSMuonReconstruction7TeV}. Muon candidates were obtained using a global fit of both tracker and muon chamber hits with a $\chi^2$ per degree of freedom of less than 10. This fit must include at least one segment in the muon chamber, track segments in at least 2 muon stations, use more than 10 hits in the inner tracker of which at least one in the a pixel layer and finally a small transverse impact paramenter $|d_{xy}|< 2\,\milli\meter$. The efficiency for such a criteria for muon candidates with $\pt>10\,\GeV$ has been measured both in data and Monte Carlo using $J/\psi \rightarrow \mu^+ \mu^-$ and $Z \rightarrow \mu^+ \mu^-$ and it plateaus at 96-99\%.

%%%%%%%%%%%%%%%%%%%%%%%%%%%%%%%%%%%%%%%%%%%%%%%%%%%%%%%%%%%%%%%%%%%%%%%%%%%%%%%%%%%%%%%
%%% SUBSECTION
%%%%%%%%%%%%%%%%%%%%%%%%%%%%%%%%%%%%%%%%%%%%%%%%%%%%%%%%%%%%%%%%%%%%%%%%%%%%%%%%%%%%%%%
\subsection{Isolation}
\label{SUBSECTION:EventReconstructionAndSimulation_LeptonIsolation_MuonsIsolation}

% More information:
%  * https://twiki.cern.ch/twiki/bin/view/CMSPublic/SWGuideMuonId#Muon_Isolation

%Status: DONE - (Reviewed x1 J.Pela) (reviewed D. Colling x1)

For muons we use the \textit{combined isolation} over a cone of $\Delta R < 0.4$ around the muon. For neutral \gls{PU} subtraction we use the charged \gls{PU} component inside the cone and multiply it by a factor of 0.5 which is determined from simulation. The definition for this isolation can be found in equation \ref{EQUATION:MuonIsolation}.

\begin{equation}
I = \sum^{\substack{\text{charged} \\ \text{non-pileup}}}\pT +
\text{max}\left(0,\sum^{\text{neutral}}\pT+\sum^{\text{photon}}\pT-\frac{1}{2}\sum^{\substack{\text{charged}
\\ \text{pileup}}}\pT\right)
\label{EQUATION:MuonIsolation}
\end{equation}

%\cite{ARTICLE:CMSMuonReconstruction7TeV} %AG 72

%%%%%%%%%%%%%%%%%%%%%%%%%%%%%%%%%%%%%%%%%%%%%%%%%%%%%%%%%%%%%%%%%%%%%%%%%%%%%%%%%%%%
%%% SUBSECTION
%%%%%%%%%%%%%%%%%%%%%%%%%%%%%%%%%%%%%%%%%%%%%%%%%%%%%%%%%%%%%%%%%%%%%%%%%%%%%%%%%%%%
\subsection{Loose Muons}

% From: https://twiki.cern.ch/twiki/bin/view/CMSPublic/SWGuideMuonId
%
% bool muon::isLooseMuon (const reco::Muon & recoMu);
% Particle-Flow muon id     & recoMu.isPFMuon()                               & Can be complemented by muon quality cuts similar to those used in the Tight Muon selection.
% Is Global OR Tracker Muon & recoMu.isGlobalMuon() || recoMu.isTrackerMuon() & Avoid using muons which are only Standalone Muons. (~0.01% of PF muons) 
% REFERENCE: MUO-10-004

%Status: DONE - (Reviewed x1 J.Pela) (reviewed D. Colling x1)

We can define a \textit{loose muon} using the cut-based definitions recommend by the \gls{CMS} Muon \gls{POG}~\cite{ARTICLE:CMSPerformanceOfMuonID7TeV} with the same name, where we require the muon candidate to be a \gls{PF} muon that is also a tracker or global muon. We exclude only standalone muons, which are only $\approx 0.01\%$ of the \gls{PF} muons. Additionally we require the muon candidate to have $\pt>10\,\GeV$, $|\eta|<2.1$ and relative combined isolation $<0.2$.

%%%%%%%%%%%%%%%%%%%%%%%%%%%%%%%%%%%%%%%%%%%%%%%%%%%%%%%%%%%%%%%%%%%%%%%%%%%%%%%%%%%%
%%% SUBSECTION
%%%%%%%%%%%%%%%%%%%%%%%%%%%%%%%%%%%%%%%%%%%%%%%%%%%%%%%%%%%%%%%%%%%%%%%%%%%%%%%%%%%%
\subsection{Tight Muons}

%Status: DONE - (Reviewed x1 J.Pela) (reviewed D. Colling x1)

We can also define \textit{tight muon} as a muon candidate with $\pt>20\,\GeV$, $|\eta|<2.1$ passing relative combined isolation $<0.12$. Additionally, we require compatibility of being produced at the primary vertex by requiring $d_{xy}<0.045\,\centi\meter$ and $d_z < 0.2\,\centi\meter$. We also require the muon to pass the  \gls{CMS} Muon \gls{POG} recommended cut-based \textit{tight muon} identification criteria that require the candidate to be a \gls{PF} muon which is also a global muon, where the global track fit has at least one muon chamber hit and $\chi^2/ndof < 10$, the presence of muon segments in at least two chambers, at least five tracker layers with hits and at least one pixel hit.

%%%%%%%%%%%%%%%%%%%%%%%%%%%%%%%%%%%%%%%%%%%%%%%%%%%%%%%%%%%%%%%%%%%%%%%%%%%%%%%%%%%%%%%
%%% SECTION
%%%%%%%%%%%%%%%%%%%%%%%%%%%%%%%%%%%%%%%%%%%%%%%%%%%%%%%%%%%%%%%%%%%%%%%%%%%%%%%%%%%%%%%
\section{Jets}
\label{SECTION:EventReconstructionAndSimulation_Jets}

%Status: DONE - (Reviewed x1 J.Pela) (reviewed D. Colling x1)

When we collide hadrons the most probable hard processes will be the scattering quarks and gluons. However, these do not reach our detectors. They quickly hadronize and fragment generating a collimated spray of particles which is commonly referred to as a jet. To determine the properties of these outgoing quarks and gluons, we need to look at the characteristics of their associated jets. To achieve this goal, we need to combine the measured jet remnants in a way that preserves the physical properties of the original parton.

%%%%%%%%%%%%%%%%%%%%%%%%%%%%%%%%%%%%%%%%%%%%%%%%%%%%%%%%%%%%%%%%%%%%%%%%%%%%%%%%%%%%%%%
%%% SUBSECTION
%%%%%%%%%%%%%%%%%%%%%%%%%%%%%%%%%%%%%%%%%%%%%%%%%%%%%%%%%%%%%%%%%%%%%%%%%%%%%%%%%%%%%%%
\subsection{Jet Clustering}
\label{SECTION:EventReconstructionAndSimulation_Jets_JetClustering}

%Status: DONE - (Reviewed x1 J.Pela) (reviewed D. Colling x1)

Jet clustering algorithms are sets of rules that allow us to combine particle candidates into jets~\cite{ARTICLE:TowardsJetography}. These algorithms normally are controlled by parameters that define how close particles need to be in order to be associated into a jet and a way to combine their momentum. However, a jet definition should be robust and provide consistent measurements about the parton underlying the jet. There are two major families of problems that may affect jet algorithms. These problems appear when the number of jets in an event changes by adding a soft collinear gluon emission (collinear safety) or by parton splitting (infrared safety). 

In \gls{CMS}, we use a sequential recombination algorithm known as anti-$k_T$~\cite{ARTICLE:AntiKtAlgorithm}, which is both infrared and collinear safe. This algorithm starts be determining a measurement of distance between every pair of objects $d_{ij}$ and the distance of each object to the beamline $d_{iB}$. The definition of these distances can be found in equations \ref{EQUATION:EventReconstructionAndSimulation_AKDistances1} and \ref{EQUATION:EventReconstructionAndSimulation_AKDistances2} respectively

\begin{align}
d_{ij} &= \text{min}({\pT}_{i}^{2p},{\pT}_{j}^{2p})\frac{\Delta R_{ij}^{2}}{R^{2}} \label{EQUATION:EventReconstructionAndSimulation_AKDistances1} \\
d_{iB} &= {\pT}_{i}^{2p}\ \label{EQUATION:EventReconstructionAndSimulation_AKDistances2} 
\end{align}

where $\Delta R$ is the separation the $\eta-\phi$ plane and $R$ is the maximum radius for the jet. The parameter $p$ determines the type of algorithm. When $p$ is equal to 1 it is a $k_T$ algorithm, 0 for the Cambridge/Aachen algorithm and -1 for the anti-$k_T$. 

After determining all the $d_{ij}$ and $d_{iB}$, we determine which is the minimum distance. If it is a $d_{ij}$ we combine those two objects and recalculate all the distances. If the minimum is a $d_{iB}$ we declare $i$ to be a \textit{final state jet}, remove it from the list of particles and recalculate all distances again. The procedure continues until there are no more objects remaining.

The anti-$k_T$ algorithm tends to cluster particles around the hardest particle in a region, which normally leads to a cone-like jet area in the $\eta-\phi$ plane. In the \gls{VBF} Higgs to invisible analysis, the clustering is made over \gls{PF} particle candidates using the implementation in the \textsc{fastjet} software package~\cite{ARTICLE:FastJetUserManual}. The \gls{CMS} recommended cone radius size for 2012-13 analysis is 0.5 while for 2015 it is 0.4.

%\cite{ARTICLE:TowardsJetography} %AG 76
%\cite{ARTICLE:AntiKtAlgorithm}   %AG 77
%\cite{ARTICLE:FastJetUserManual} %AG 78

%%%%%%%%%%%%%%%%%%%%%%%%%%%%%%%%%%%%%%%%%%%%%%%%%%%%%%%%%%%%%%%%%%%%%%%%%%%%%%%%%%%%%%%
%%% SUBSECTION
%%%%%%%%%%%%%%%%%%%%%%%%%%%%%%%%%%%%%%%%%%%%%%%%%%%%%%%%%%%%%%%%%%%%%%%%%%%%%%%%%%%%%%%
\subsection{Particle Flow Jet Identification} 
\label{SECTION:EventReconstructionAndSimulation_Jets_ParticleFlowJetID}

% More information at:
% https://twiki.cern.ch/twiki/bin/view/CMS/JetID

%Status: DONE - (Reviewed x1 J.Pela) (reviewed D. Colling x1)

The \gls{CMS} Jet-MET \gls{POG} has defined criteria to reject fake, badly reconstructed, and noisy \gls{PF} jets while keeping 98-99\% real jets~\cite{ARTICLE:CMSAN-14-227}. All the presented analyses in this thesis have used the recommended \gls{PF} jet \gls{ID} in the loose working point. In this working point all jets are required to have at least two constituents, and both the neutral hadron fraction and the neutral \gls{EM} fraction to be below 99\%. Additionally, for jets inside the tracker acceptance with $|\eta| < 2.4$, we require the charged particle multiplicity and charged hadron fraction to be greater than zero, and the charged \gls{EM} fraction to be less than 99\%.

%%%%%%%%%%%%%%%%%%%%%%%%%%%%%%%%%%%%%%%%%%%%%%%%%%%%%%%%%%%%%%%%%%%%%%%%%%%%%%%%%%%%%%%
%%% SUBSECTION
%%%%%%%%%%%%%%%%%%%%%%%%%%%%%%%%%%%%%%%%%%%%%%%%%%%%%%%%%%%%%%%%%%%%%%%%%%%%%%%%%%%%%%%
\subsection{Pileup Jet Identification}
\label{SECTION:EventReconstructionAndSimulation_Jets_PileupJetID}

% More information at:
% https://twiki.cern.ch/twiki/bin/view/CMS/PileupJetID

%Status: DONE - (Reviewed x1 J.Pela) (reviewed D. Colling x1)

To identify if a \gls{PF} jet has come from \gls{PU} or from the primary vertex we make use of a \gls{BDT}. This machine learning algorithm was trained with information about the trajectory of the tracks associated with the jet, the jet shape, and object multiplicity. In the presented analyses we have used the recommended loose working point of the \textit{full \gls{BDT} method}~\cite{ARTICLE:CMS-PAS-JME-13-005}. This method was applied to each jet, which would only be accepted if the \gls{BDT} output score would pass the cuts defined in table \ref{TABLE:EventReconstructionAndSimulation_PileupJetIDFullBDTLooseWorkingPoint} depending on jet \pt and \eta.

\begin{table}[!htb]
\centering
\begin{tabular}{|c|c|c|}
\hline
Jet $p_{T}$          & Jet $|\eta|$              & $BDT_{score}$ \\
\hline \hline
$20 < p_{T} \leq 30$ & $|\eta| < 2.5$            & $> -0.80$ \\
$20 < p_{T} \leq 30$ & $2.50 \leq |\eta| < 2.75$ & $> -0.85$ \\
$20 < p_{T} \leq 30$ & $2.75 \leq |\eta| < 3.00$ & $> -0.84$ \\
$20 < p_{T} \leq 30$ & $3.00 \leq |\eta| < 5.00$ & $> -0.85$ \\
$30 < p_{T}$         & $|\eta| < 2.5$            & $> -0.80$ \\
$30 < p_{T}$         & $2.50 \leq |\eta| < 2.75$ & $> -0.74$ \\
$30 < p_{T}$         & $2.75 \leq |\eta| < 3.00$ & $> -0.68$ \\
$30 < p_{T}$         & $3.00 \leq |\eta| < 5.00$ & $> -0.77$ \\
\hline
\end{tabular}
\caption[Table of the minimum values of full BDT method score for a PF jet to be accepted as coming from the PV using a loose working point. Required minimum values have been binned in jet \pt and \eta.]{Table of the minimum values of \textit{full \gls{BDT} method} score for a \gls{PF} jet to be accepted as coming from the \gls{PV} using a loose working point. Required minimum values have been binned in jet \pt and \eta.}
\label{TABLE:EventReconstructionAndSimulation_PileupJetIDFullBDTLooseWorkingPoint}
\end{table}

%%%%%%%%%%%%%%%%%%%%%%%%%%%%%%%%%%%%%%%%%%%%%%%%%%%%%%%%%%%%%%%%%%%%%%%%%%%%%%%%%%%%%%%
%%% SUBSECTION
%%%%%%%%%%%%%%%%%%%%%%%%%%%%%%%%%%%%%%%%%%%%%%%%%%%%%%%%%%%%%%%%%%%%%%%%%%%%%%%%%%%%%%%
\subsection{Lepton cleaning}
\label{SECTION:EventReconstructionAndSimulation_Jets_LeptonCleaning}

%Status: DONE - (Reviewed x1 J.Pela) (reviewed D. Colling x1)

To avoid having leptons being misidentified as jets we filter out all jets which are located at $\Delta R < 0.5$ to any veto electron or loose muons.

%%%%%%%%%%%%%%%%%%%%%%%%%%%%%%%%%%%%%%%%%%%%%%%%%%%%%%%%%%%%%%%%%%%%%%%%%%%%%%%%%%%%%%%
%%% SUBSECTION
%%%%%%%%%%%%%%%%%%%%%%%%%%%%%%%%%%%%%%%%%%%%%%%%%%%%%%%%%%%%%%%%%%%%%%%%%%%%%%%%%%%%%%%
\subsection{Jet Energy Corrections}
\label{SECTION:EventReconstructionAndSimulation_Jets_JetEnergyCorrections}

%Status: DONE - (Reviewed x1 J.Pela) (reviewed D. Colling x1)

When reconstructing a jet, the clustered energy often does not match the original parton energy. There are many reasons for this effect, like non-linearity of the calorimeters response, detector noise, overlap with problematic detector areas, additional energy from \gls{PU}, miscalibration, etc. To fix this problem corrections are determined and applied to each jet in order to have, on average, an energy measurement that is equal to the original parton. These corrections can be factorized into components as it is represented in equation \ref{EQUATION:EventReconstructionAndSimulation_JetEnergyCorrections}~\cite{ARTICLE:CMSDeterminationJetEnergyCalibration}.

\begin{equation}
P_{\text{corr}}^{\mu} = C_{\text{offset}}(\pT^{\text{raw}},\eta) \cdot C_{\text{rel}}(\pT^{\text{off}},\eta) \cdot C_{\text{abs}}(\pT^{\text{rel}},\eta) \cdot P_{\text{raw}}^{\mu}
\label{EQUATION:EventReconstructionAndSimulation_JetEnergyCorrections}
\end{equation}

The $C_{\text{offset}}$ term accounts for and subtracts the contribution of \gls{PU} and noise in the detector measurements. Its value is determined by taking into account the specific event \pt-density, \rho, and the individual jet area A~\cite{ARTICLE:PileupSubtractionJetAreas}. The event \rho is calculated as the median \pt-density of all jets present in the event. Since the median is taken, it will not be affected by the presence of hard jets. Unfortunately, the \gls{UE} activity has similar characteristics to the \gls{PU} and should not be subtracted. To avoid this effect, the correction takes the form of $\rho - \langle \rho \rangle_{UE} \cdot A$, where $\langle \rho \rangle_{UE}$ is the average expected \gls{UE} contribution.

The $C_{\text{rel}}$ term is applied to make the energy response flat as a function of $\eta$. It is applied to the offset-corrected transverse momentum $\pT^{off}$. To determine its value the \pt-balancing method is used~\cite{ARTICLE:CMSDeterminationJetEnergyCalibration}. In this method, we select a reference jet located in the central region where energy measurement is expected to be flat and a probe jet at any value of $\eta$. We can calculate the average of balance quantity as $(\pt^{\text{probe}} - \pt^{\text{reference}} ) / \pt^{\text{average}}$ which is used to determine the correction to response in bins of jet \eta and dijet average \pt. 

The $C_{\text{abs}}$ term is intended to make the response uniform in \pt. It is applied to the \eta-corrected transverse momentum $\pT^{rel}$ and is calculated using the \gls{MPF} method~\cite{ARTICLE:CDFDijetAngularDistribution}. In this method, we use the good experimental resolution for leptons and photons in processes like $\gamma + \text{jets}$ and $Z + \text{jets}$ to infer on the properties of the recoil jets. Since these processes should not have \gls{MET}, if MET is observed, it can be used to calibrate the jet response for the jets present in the event.

%TODO: Check with patrick
The total uncertainty on the jet energy scale is obtained by summing in quadrature the estimated uncertainties of each one of the correction terms. The total uncertainty is in the range of $\approx 3-5\%$ depending on \pt and \eta~\cite{ARTICLE:CMSDeterminationJetEnergyCalibration}.

%~\cite{ARTICLE:CMSDeterminationJetEnergyCalibration} % AG 79
%~\cite{ARTICLE:PileupSubtractionJetAreas}            % AG 80
%~\cite{ARTICLE:CDFDijetAngularDistribution}          % AG 81

%%%%%%%%%%%%%%%%%%%%%%%%%%%%%%%%%%%%%%%%%%%%%%%%%%%%%%%%%%%%%%%%%%%%%%%%%%%%%%%%%%%%%%%
%%% SECTION
%%%%%%%%%%%%%%%%%%%%%%%%%%%%%%%%%%%%%%%%%%%%%%%%%%%%%%%%%%%%%%%%%%%%%%%%%%%%%%%%%%%%%%%
\section{Hadronic Taus}
\label{SECTION:EventReconstructionAndSimulation_Taus}

%Status: DONE - (Reviewed x1 J.Pela)  (reviewed D. Colling x1)

Taus can decay leptonically and hadronically. In leptonic decays the tau decays directly to an electron or a muon and two additional neutrinos. Therefore it is very difficult to identify such decays experimentally. On the other hand, a hadronic tau decay produces a characteristic signature of a narrow jet containing an odd number of charged particles and additional neutral hadrons, as well as a tau neutrino. In all the analyses presented in this thesis, when referring to a tau, we refer an hadronically decaying tau. The most probable decay modes have one or three charged $\pi$ mesons and are summarized in table \ref{TABLE:EventReconstructionAndSimulation_TauDecays}. 

\begin{table}[!htb]
\begin{tabular}{|l|c|r|r|}
\hline
Decay Channel & Resonance & Mass [$\MeV$] & Branching Fraction [\%] \\
\hline\hline
$\tau^{\pm} \rightarrow \pi^{\pm} \nu_\tau$                              &                &      & 11.6 \\
$\tau^{\pm} \rightarrow \pi^{\pm} \pi^{0}   \nu_\tau$                    & $\rho$         &  770 & 26.0 \\
$\tau^{\pm} \rightarrow \pi^{\pm} \pi^{0}   \pi^{0}   \nu_\tau$          & $\text{a}_{1}$ & 1260 & 10.8 \\
$\tau^{\pm} \rightarrow \pi^{\pm} \pi^{\mp} \pi^{\pm} \nu_\tau$          & $\text{a}_{1}$ & 1260 &  9.8 \\
$\tau^{\pm} \rightarrow \pi^{\pm} \pi^{\mp} \pi^{\pm} \pi^{0} \nu_\tau$  &                &      &  4.8 \\
\hline
Other hadronic modes                                                     &                &      &  1.7 \\
\hline\hline
Total & &  & 64.7 \\
\hline
\end{tabular}
\caption[Summary of the hadronic tau decay modes.]{Summary of the hadronic tau decay modes, with the branching fractions and intermediate resonances listed where relevant~\cite{ARTICLE:PDG}.}
\label{TABLE:EventReconstructionAndSimulation_TauDecays}
\end{table}

Reconstruction of hadronic tau neutrinos with \gls{PF} is done by identifying the specific decay mode visible products. The approach is at the core of the \gls{HPS} algorithm~\cite{ARTICLE:CMSPerformaceOfTauLeptonReconstruction,ARTICLE:CMSReconstructionIndentificationTau}. It combines reconstructed charged hadrons with strips of clustered photons which are interpreted as $\pi_0$. The reconstructed system is constrained by the tau mass and intermediate resonances and is a highly collimated jet when compared with a typical \gls{QCD} jet.

%%%%%%%%%%%%%%%%%%%%%%%%%%%%%%%%%%%%%%%%%%%%%%%%%%%%%%%%%%%%%%%%%%%%%%%%%%%%%%%%%%%%%%%
%%% SUBSECTION
%%%%%%%%%%%%%%%%%%%%%%%%%%%%%%%%%%%%%%%%%%%%%%%%%%%%%%%%%%%%%%%%%%%%%%%%%%%%%%%%%%%%%%%
% \subsection{Hadron Plus Stips Algorithm}
% \label{SECTION:EventReconstructionAndSimulation_Taus_HPSAlgorithm}
% 
% %Status: DONE - (Reviewed x1 J.Pela)  (reviewed D. Colling x1)
% 
% The \gls{HPS} algorithm utilizes \gls{PF} candidates to reconstruct charged pions and photons resulting from neutral pions decay. These photon can convert in into electron-positron pairs in the tracker material. This factors are taken into consideration, as well as deflection in the magnetic field. We also attempt to find the intermediate resonances listed in table \ref{TABLE:EventReconstructionAndSimulation_TauDecays} has a handle to determine the tau decay channel. A tau neutrino is present on all decays and cannot be directly measured, this results in a smearing of the measured tau mass when considering only the visible products.
% 
% We seed the algorithm with \gls{PF} anti-$k_T$ jets with $R = 0.5$ where $\pt>14\,\GeV$ and $|\eta|<2.5$. To search for the $\pi^{0}$ decay products we try to identify strips by clustering \gls{PF} electrons and photons with $\pt>0.5\,\GeV$. We start from the most energetic electromagnetic particle inside the jet area and make that the centre of our candidate strip. We look for other electromagnetic objects within a window of $\Delta\eta = 0.05$ and $\Delta\phi = 0.20$ of the centre of the strip. If an object is found it gets associated with the strip and its four-momentum gets recalculated. We repeat the procedure until we cannot find any more unassociated \gls{EM} objects inside the strip area. If the final strip object has a mass compatible with a $\pi^{0}$, in the interval between $50-200\,\MeV$, and has $\pt > 2.5\,\GeV$ it is kept. We then start the next strip clustering with the highest \pt electron or gamma not already belonging to a strip.
% 
% The charged pion candidates are required to have $\pt > 0.5\,\GeV$ and its track pass $d_z < 0.4\,\centi\meter$ and $d_{xy} < 0.03\,\centi\meter$ to the vertex associated with the highest \pt track in the jet, which is assumed to be the \tau production vertex.
% 
% The following topologies are taken into account by the \gls{HPS} algorithm:
% 
% \begin{enumerate}
%   \item \textit{single hadron}: tries to identify tau decays into $\pi^{\pm} \nu_\tau$ or $\pi^{\pm} \pi^{0} \nu_\tau$ where the netral pion decay cannot be identified as a strips.
%   \item \textit{One hadron + one strip}: tries to identify tau decays into $\pi^{\pm} \pi^{0} \nu_\tau$ where the $\pi^{0}$ decay photons are close together. In this case we are selecting the $\rho(770)$ intermediate resonance. The mass of the reconstructed $\tau_{had}$ is required to be in the interval $0.4<m_{\tau_{had}}<1.3\,\GeV$ for $\pt^{\tau_{had}} < 200\,\GeV$. The upper limit in the mass window can go up to $2.1\,\GeV$ for candidates with $\pt^{\tau_{had}} > 200\,\GeV$ to account for resolution effects.
%   \item \textit{One hadron + two strip}: tries to identify tau decays into $\pi^{\pm} \pi^{0} \pi^{0} \nu_\tau$. In this case we are selecting the $a_1(1260)$ intermediate resonance. The mass of the reconstructed $\tau_{had}$ is required to be in the interval $0.4<m_{\tau_{had}}<1.2\,\GeV$ for $\pt^{\tau_{had}} < 200\,\GeV$. The upper limit in the mass window can go up to $2.0\,\GeV$ if the $\pt^{\tau_{had}}$ increases above $200\,\GeV$.
%   \item \textit{Three hadrons}: tries to identify tau decays into $\pi^{\pm} \pi^{\mp} \pi^{\pm} \nu_\tau$. The hadrons are required to have mass in the interval $0.8-1.5\,\GeV$ since we assume the $a_{1}(1260)$ intermediate resonance. Total charged is required to be one.
% \end{enumerate}
% 
% There is no dedicated search for $\pi^{\pm} \pi^{\mp} \pi^{\pm} \pi^{0} \nu_\tau$ or higher pion multiplicity decay modes. These topologies are reconstructed with the currently defined criteria.
% 
% All selected hadrons and strips are required to be inside of cone of $\Delta R < 2.8\,\GeV/\pt^{\tau_{had}}$. The cone size is is constrained to the interval $\Delta R=0.05-0.10$. 

%%%%%%%%%%%%%%%%%%%%%%%%%%%%%%%%%%%%%%%%%%%%%%%%%%%%%%%%%%%%%%%%%%%%%%%%%%%%%%%%%%%%%%%
%%% SUBSECTION
%%%%%%%%%%%%%%%%%%%%%%%%%%%%%%%%%%%%%%%%%%%%%%%%%%%%%%%%%%%%%%%%%%%%%%%%%%%%%%%%%%%%%%%
\subsection{Isolation and Discriminants}
\label{SECTION:EventReconstructionAndSimulation_Taus_IsolationAndDiscriminants}

%Status: DONE - (Reviewed x1 J.Pela) (reviewed D. Colling x1)

Isolation for taus is calculated in a similar way to electrons ans muons. The isolation variable is defined by summing the \pt of all \gls{PF} hadron and photon candidates in a cone of $\Delta R < 0.5$ around the tau axis. Here the charged hadron tracks are required to have $d_z < 2\,\centi\meter$ to the tau production vertex. We can subtract the contribution to isolation coming from \gls{PU} estimating its density in a cone of $\Delta R < 0.8$ around the tau and considering tracks with $d_z > 2\,\centi\meter$. All tau constituents are ignored in this sum. Working points have been defined for loose, medium and tight isolation~\cite{ARTICLE:CMSReconstructionIndentificationTau}.

Electrons can be reconstructed as taus when they make isolated deposits in the calorimeter or emit enough energy via bremsstrahlung to form a strip. A \gls{BDT} has been trained with a set of variables similar to the ones used in electron identification to exclude such misreconstructions. Similarly to the electron and muon isolation three working points have been defined~\cite{ARTICLE:CMSPerformaceOfTauLeptonReconstruction,ARTICLE:CMSReconstructionIndentificationTau}. 

Muons are less likely to be reconstructed as a tau. We can exclude such tau candidates by requiring that the track of the leading charged hadron is not also a tracker muon. This discriminator also has three possible working points~\cite{ARTICLE:CMSPerformaceOfTauLeptonReconstruction,ARTICLE:CMSReconstructionIndentificationTau}.

We can now define a hadronic tau candidate as a \gls{PF} tau candidate with $\pt > 20 \,\GeV$, $|\eta|<2.3$ and $d_z<0.2 \,\centi\meter$ to the primary vertex. We require that the candidate passes decay-mode identification, tight isolation and finally tight discriminators against electrons and muons.

%%%%%%%%%%%%%%%%%%%%%%%%%%%%%%%%%%%%%%%%%%%%%%%%%%%%%%%%%%%%%%%%%%%%%%%%%%%%%%%%%%%%%%%
%%% SECTION
%%%%%%%%%%%%%%%%%%%%%%%%%%%%%%%%%%%%%%%%%%%%%%%%%%%%%%%%%%%%%%%%%%%%%%%%%%%%%%%%%%%%%%%
\section{Missing Transverse Energy}
\label{SECTION:EventReconstructionAndSimulation_MET}

%Status: DONE - (Reviewed x1 J.Pela) (reviewed D. Colling x1)

The Standard Model describes neutrinos as particles which only interact via the weak force. They can pass through our detectors without interacting and therefore not allowing any direct measurement. Many new models describe additional particles that would also be able to escape detection by leaving very small or no energy deposits in our experiments. The appearance of such particles can only be inferred through the measurement of an imbalance of transverse momentum of all detected particles. This effect can be quantified as the negative sum off all visible particle candidates transverse momentum in an event. 

The magnitude of that vector is referred to as \acrfull{MET}. Particle flow methodology provides a complete list of object candidates in the event with excellent resolution achieved by combining all available information, making it well suited to be the input for \gls{MET} calculation. Although \gls{CMS} has an excellent individual particle resolution, the calculation of \gls{MET} is affected by the combined resolution of the measurement of all particles in the event. Figure \ref{FIGURE:EventReconstructionAndSimulation_METDistributionZmumu} shows the distributions of \gls{PF} \gls{MET} for both data and simulation for event selections of $Z \rightarrow \mu\mu$ and $\gamma +jets$ processes at $\sqrt{s}=8\,\TeV$. 

\begin{figure}[!htp]%
\centering
\subfloat[]{\includegraphics[width=0.45\textwidth]{Chapter04/MET/Images/met_zmm.pdf}}\qquad
\subfloat[]{\includegraphics[width=0.45\textwidth]{Chapter04/MET/Images/met_gamma.pdf}}\\
 \caption[Distributions of the particle flow $\MET$ in $Z\rightarrow\mu\mu$ and $\gamma$+jets events in $\sqrt{s}=8\,\TeV$ data and simulation.]{Distributions of the particle flow $\MET$ in (a) $Z\rightarrow\mu\mu$ and (b) $\gamma$+jets events in $\sqrt{s}=8\,\TeV$ data and simulation. The uncertainty in the muon, photon, jet and neutral hadron energy responses is showed by the shaded band~\cite{ARTICLE:CMSMETPerformance8TeV}.}
\label{FIGURE:EventReconstructionAndSimulation_METDistributionZmumu}
\end{figure}

Both photon and muon energy measurements have good resolution in the \gls{CMS} experiment and these processes do not involve real \gls{MET}. The observed distributions in both plots are predominantly shaped by the resolution of jet energy measurement.

During data taking, issues with the detector or data acquisition can happen creating anomalously high \gls{MET} and rendering such events unusable. The groups responsible for each part of the detector and individual physics objects, check the data after collection to detect such problems. Event filters are produced to reject such problematic events when performing physics analyses.  The \gls{CMS} JET-MET \gls{POG} compiled a list of the recommended filters for analysis using 2012-13 data to remove events affected by energy deposits from beam halo, noise in \gls{HCAL} readout electronics, particles directly hitting the \gls{ECAL} photodiodes, track reconstruction problems and finally \gls{ECAL} and \gls{HCAL} mistimed laser calibration sequence. These filters have been used in both prompt and parked \gls{VBF} Higgs to invisible analyses.

There are many factors that affect \gls{MET} response and resolution. These include zero suppression thresholds, which dictate the minimum energy a calorimeter cell will report, dead or non-instrumented regions of the detector and reconstruction inefficiencies. Techniques have been developed to correct both response and resolution when using \gls{PF} \gls{MET}~\cite{ARTICLE:CMSMissingTransverseEnergyPerformance}. These corrections include accounting for the bias in response due to using incorrect energy scale of the jets, and reducing the impact of pileup on the resolution~\cite{ARTICLE:CMSMETPerformance8TeV}.

In the \gls{VBF} Higgs to invisible analysis, the irreducible background of $Z \rightarrow \nu\nu$ is investigated using $Z \rightarrow \mu\mu$ as a proxy. To record this process with the same \gls{HLT} trigger that is used for signal, \gls{MET} is calculated without including muons, $MET_{no-\mu}$. This quantity is also used in the offline analysis, where muons are vetoed in the signal region to recover the usual \gls{MET} value, and are required in some control regions to estimate backgrounds.


%~\cite{ARTICLE:CMSMETPerformance8TeV}                 % AG 83
%~\cite{ARTICLE:CMSMissingTransverseEnergyPerformance} % AG 84

%%%%%%%%%%%%%%%%%%%%%%%%%%%%%%%%%%%%%%%%%%%%%%%%%%%%%%%%%%%%%%%%%%%%%%%%%%%%%%%%%%%%%%%
%%% SECTION
%%%%%%%%%%%%%%%%%%%%%%%%%%%%%%%%%%%%%%%%%%%%%%%%%%%%%%%%%%%%%%%%%%%%%%%%%%%%%%%%%%%%%%%
\section{Monte Carlo Simulation}
\label{SECTION:EventReconstructionAndSimulation_MonteCarloSimulation}

%Status: DONE - (Reviewed x1 J.Pela) (reviewed D. Colling x1)

To simulate one event in the \gls{CMS} experiment, we first start by the physics process itself. It can be split into two sub-processes: hard scattering and hadronization. There are many purpose-built software programs that will perform each of these steps. An illustration of how the simulation of proton-proton collision is done with \gls{MC} programs can be found in figure \ref{FIGURE:EventReconstructionAndSimulation_MCShower}. A review of the available generators for \gls{LHC} physics can be found in reference~\cite{ARTICLE:GeneralPurposeEventGeneratorsForLHCPhysics}.

\begin{figure}[!htb]
\centering
\includegraphics[width=0.8\textwidth]{Chapter04/MonteCarlo/Images/MCShower.png}
\caption[Illustration a proton-proton collision as implemented in MC event generators.]{Illustration of a proton-proton collision as implemented in some MC event generators~\cite{IMAGEREF:krauss-diag}. Sub-processes are represented, the hard-scattering in the center of the diagram, the parton showering in red, hadronization in green. We can also observe the \gls{UE} interaction and its showering in purple.}
\label{FIGURE:EventReconstructionAndSimulation_MCShower}
\end{figure}

General purpose particle physics event generators like \textsc{PYTHIA 8}~\cite{ARTICLE:Pythia6p4PhysicsAndManual,ARTICLE:Pythia8p1Introduction}, \textsc{HERWIG++}~\cite{ARTICLE:HERWIGPhysicsAndManual} and \textsc{SHERPA}~\cite{ARTiCLE:SherpaEventGenerator} are able to do both hard scattering and hadronization steps for a wide variety of physics processes. Typically these programs are restricted to $2 \rightarrow 2$ and $2 \rightarrow 1$ hard processes calculated at \gls{LO}.

There are many other dedicated matrix-element generators, like \textsc{MADGRAPH 5}~\cite{ARTICLE:MadGraph5}, \textsc{ALPGEN}~\cite{ARTICLE:ALPGENGenerator} and also \textsc{SHERPA} that focus on the hard process simulation. These programs provide $2 \rightarrow X$ hard scattering, where a higher number of partons in the final state is possible. Some generators have also implemented \gls{NLO} calculations, which provide better kinematics discription and lower uncertainties. Two examples of such generators are a\textsc{MC@NLO}~\cite{ARTICLE:aMCatNLO} and \textsc{POWHEG}~\cite{ARTICLE:POWHEG_2004,ARTICLE:POWHEG_2007,ARTICLE:POWHEG_2009v1,ARTICLE:POWHEG_2009v2,ARTICLE:POWHEG_2010v1,ARTICLE:POWHEG_2010v2,ARTICLE:POWHEG_2011v1,ARTICLE:POWHEG_2011v2}. Both these generators are extensively used by the \gls{CMS} collaboration, and the latter was used to simulate the signal sample in the analysis presented in this thesis. The simulated parton level events then need to be passed to a general purpose event generator for hadronization. 

Overlapping of the phase-space description of matrix-element and showering programs needs to be avoided when simulating multi-jet events. This problem rises from software like \textsc{PYTHIA} or \textsc{HERWIG} describing parton radiation as a Markov Chain process based on Sudakov form factors. This approach is only formally correct in the limit of soft and collinear emissions. On the other hand \gls{ME} programs like \textsc{MadGraph} work well in the hard scattering high energy limit but diverge when the partons become soft or collinear. 

There are a few jet-parton matching schemes developed to account for this overlap~\cite{ARTICLE:MatchingPartonShowersAndMatrixElements}. Showering can be vetoed and the event reweighed accordingly, like in the Catani-Krauss-Kuhn-Webber (CKKW) scheme~\cite{ARTICLE:CKKWSchemeRef1,ARTICLE:CKKWSchemeRef2,ARTICLE:CKKWSchemeRef3} or events can be rejected altogether like in the MLM scheme~\cite{ARTICLE:MLMScheme}. Depending on the generator used for the showering, different schemes are implemented and care must be taken in the definition of the matching parameters.

Most event generators can be finely tuned so important aspects of the simulation can be adjusted to experimental conditions. As an example, in the \gls{CMS} experiment, \textsc{PYTHIA} is used with the Z2 tune, which was produced using measurements made using minimum bias data at the Tevatron and at the \gls{LHC}~\cite{ARTICLE:CMSMeasurementUnderlyingEventActivity}.  

After the physics event is simulated, the interaction with the detector and the corresponding electronics response is estimated using a precise model of the experiment. In the \gls{CMS} experiment \textsc{GEANT 4}~\cite{ARTICLE:GEANT4ASimulationToolkit,ARTICLE:Geant4DevelopmentsAndApplications} software is used for this task which also relies heavily on \gls{MC} methods.




