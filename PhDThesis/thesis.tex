%start% Brought this in from Patrick thesis
\documentclass{thesis}
\usepackage[utf8]{inputenc}
\usepackage{parskip}
\usepackage{hepnames}
% \usepackage[printonlyused]{acronym}
\usepackage{thesis}
\usepackage{ptdr-definitions}
\usepackage{hepnicenames}
\usepackage{hepunits}
\usepackage{abhep}
\usepackage{slashed}
\usepackage{multirow}
\usepackage{colortbl}
\usepackage{float}
%end%
%\documentclass[hyperpdf]{hepthesis}
%\documentclass[a4paper,10pt]{book}
%\usepackage{fullpage}
%\usepackage[margin=2cm]{geometry} % Defining geometry of the usable text space

%% Using Babel allows other languages to be used and mixed-in easily
%\usepackage[ngerman,english]{babel}
% \usepackage[english]{babel}
% \selectlanguage{english}

%% Citation system tweaks
% \usepackage{cite}

\usepackage{amsmath}
\usepackage{bm}
% \usepackage{amsthm}
% \usepackage{amsfonts}
\usepackage{amssymb}
% \usepackage{graphicx}
% \usepackage{color}                 % For color (highlights,...) 
% \usepackage{caption}               % ?????
% \usepackage{soul}                  % For text highlights
\usepackage{threeparttable}          % Better tables
% \usepackage{siunitx}                 % Allows easy x10^ for numbers
% \usepackage{hyperref}              % Need after glossaries (allows links)
\usepackage[toc,acronym]{glossaries} %(recommended) for the indexing phase, as opposed to makeindex (the default)
\makeglossaries
% \newacronym[longplural={Frames per Second}]{fpsLabel}{FPS}{Frame per Second}

%A
\newacronym{ALICE}{ALICE}{A Large Ion Collider Experiment}
\newacronym{ATLAS}{ATLAS}{A Toroidal LHC ApparatuS}
\newacronym{APD}  {APD}  {Avalanche photo-diodes}

%B
\newacronym{BSM}{BSM}{Beyond the Standard Model}
\newacronym{BDT}{BDT}{Boosted Decision Tree}

%C
\newacronym{CERN}{CERN}{European Organization for Nuclear Research} %derived from the name Conseil Européen pour la Recherche Nucléaire
\newacronym{CMS} {CMS} {Compact Muon Solenoid}
\newacronym{CSC} {CSC} {Cathode Strip Chamber}
\newacronym{CPU} {CPU} {Central Processing Unit}
\newacronym{CTF} {CTF} {Combinatorial Track Finder}
%D
\newacronym{DA} {DA} {Deterministic Annealing}
\newacronym{DAQ}{DAQ}{Data Acquisition}
\newacronym{DQM}{DQM}{Data Quality Monitoring}
\newacronym{DT} {DT} {Drift Tube}

%E
\newacronym{EB}  {EB}  {ECAL Barrel}
\newacronym{EE}  {EE}  {ECAL Endcap}
\newacronym{ECAL}{ECAL}{Electromagnetic Calorimeter}

%F
\newacronym{FCT}{FCT}{Fundação para a Ciência e a Tecnologia}

%G
\newacronym{GCT}{GCT}{Global Calorimeter Trigger}

%H
\newacronym{HCAL}{HCAL}  {Hadronic Calorimeter}
\newacronym{HB}  {HB}    {HCAL Barrel}
\newacronym{HE}  {HE}    {HCAL Endcap}
\newacronym{HO}  {HO}    {HCAL Outer}
\newacronym{HF}  {HF}    {HCAL Forward}
\newacronym{HLT} {HLT}   {High Level Trigger}

%I

%J
\newacronym{JER}{JER}{jet energy resolution}
\newacronym{JES}{JES}{jet energy scale}

%K

%L
\newacronym{L1T}   {L1T}   {Level 1 Trigger}
\newacronym{LEP}   {LEP}   {Large Electron Positron collider}
\newacronym{LINAC2}{LINAC2}{Linear Particle Accelerator 2}
\newacronym{LHC}   {LHC}   {Large Hadron Collider}
\newacronym{LHCb}  {LHCb}  {Large Hadron Collider beauty}
\newacronym{LS1}   {LS1}   {Long Shutdown 1}

%M
\newacronym{MC} {MC} {Monte Carlo}
\newacronym{MET}{MET}{Missing Transverse Energy}

%N

%O

%P
\newacronym{PAG}{POG}{Physics Analysis Group}
\newacronym{PS} {PS} {Proton Synchrotron}
\newacronym{PSB}{PSB}{Proton Synchrotron Booster}
\newacronym{PU} {PU} {Pile-Up}
\newacronym{PV} {PV} {Primary Vertex}
\newacronym{POG}{POG}{Particle Object Group}

%Q
\newacronym{QCD}{QCD}{Quantum Chromodynamics}

%R
\newacronym{RF} {RF} {Radio Frequency}
\newacronym{RPC}{RPC}{Resistive Plate Chamber}

%S
\newacronym{SM} {SM} {Standard Model}
\newacronym{SPS}{SPS}{Super  Proton Synchrotron}

%T

%U

%V
\newacronym{VBF}{VBF}{vector Boson Fusion}
\newacronym{VPT}{VPT}{Vacuum Photo-Triodes}
%W

%X

%Y

%Z

% \acro{CL}{confidence level}
% \acro{CSC}{cathode strip chamber}
% \acro{CSV}{combined secondary vertex}
% \acro{DA}{deterministic annealing}
% \acro{DT}{drift tube}
% \acro{GSF}{Gaussian sum filter}
% \acro{HB}{hadron barrel}
% \acro{HE}{hadron endcaps}
% \acro{HF}{hadron forward}
% \acro{HLT}{high-level trigger}
% \acro{HO}{hadron outer}
% \acro{HPS}{hadron plus strips}
% \acro{JER}{jet energy resolution}
% \acro{JES}{jet energy scale}
% \acro{L1}{Level-1}
% \acro{LHCHXSWG}{LHC Higgs Cross Section Working Group}
% \acro{LO}{leading order}
% \acro{MPF}{missing transverse energy projection fraction}
% \acro{MPI}{multi-parton interaction}
% \acro{MVA}{multi-variate analysis}
% \acro{MSSM}{minimal supersymmetric standard model}
% \acro{NLO}{next-to-leading order}
% \acro{NNLO}{next-to-next-to-leading order}
% \acro{NNLL}{next-to-next-to-leading logarithmic}
% \acro{pdf}{probability density function}
% \acro{PDF}{parton distribution function}
% \acro{PF}{particle flow}
% \acro{PS}{Proton Synchrotron}
% \acro{PSB}{Proton Synchrotron Booster}
% \acro{PU}{pile-up}
% \acro{PV}{Primary Vertex}
% \acro{QCD}{Quantum Chromodynamics}
% \acro{RF}{radio frequency}
% \acro{RPC}{resistive plate chamber}
% \acro{SPS}{Super Proton Synchrotron}
% \acro{SSV}{simple secondary vertex}
% \acro{SM}{standard model}
% \acro{TEC}{tracker endcaps}
% \acro{TIB}{tracker inner barrel}
% \acro{TID}{tracker inner disks}
% \acro{TOB}{tracker outer barrel}
% \acro{UE}{underlying event}
% \acro{VBF}{vector boson fusion}
% \acro{WLCG}{Worldwide LHC Computing Grid}


% \hypersetup{
% colorlinks=false
% }

% \newcommand{\PW}   {\ensuremath{W}}
\newcommand{\W}    {\ensuremath{W}} % plain W (no superscript +/-)
\newcommand{\Z}    {\ensuremath{Z}} % plain Z (no superscript 0)
\newcommand{\tauh} {\ensuremath{\Pgt_\mathrm{h}}\xspace}
\newcommand{\ttbar}{\ensuremath{t\overline{t}}\xspace} % t-tbar

\newcommand{\BRinv}{\ensuremath{\mathcal{B}(H\to\text{inv})}}
\newcommand{\lagr}{\hat{\mathscr{L}}}
\newcommand{\stat}{\ensuremath{\,\text{(stat)}}\xspace}
\newcommand{\syst} {\ensuremath{\,\text{(syst)}}\xspace}
\newcommand{\sutwol}{SU(2)_{L}}
\newcommand{\uone}{U(1)_{Y}}

\newcommand{\specialcell}[2][c]{%
  \begin{tabular}[#1]{@{}c@{}}#2\end{tabular}}
  
% DEBUG: Adding line numbers
% \linenumbers

\title{\LARGE Search for Higgs Decays to Dark Matter and Trigger Studies} 
\author{João Carlos Arnauth Pela}
% \email[Contact email: ]{joaopela@gmail.com}
% \affiliation{Imperial College London}
% \keywords{Particle Physics}
\date{30 December 2015}


%% Doc-specific PDF metadata
\makeatletter
\@ifpackageloaded{hyperref}{
\hypersetup{
  pdftitle    = {Search for Higgs Decay to Dark Matter and Trigger Studies},
  pdfsubject  = {João Pela's PhD thesis},
  pdfkeywords = {LHC, CMS, Higgs, Dark Matter, Trigger},
  pdfauthor   = {\textcopyright\ João Pela}
}
}{}
\makeatother

\begin{document}

\begin{frontmatter}
  %% Title
%!!UPDATE THIS
\titlepage[of Imperial College London]%
{A dissertation submitted to Imperial College London\\
  for the degree of Doctor of Philosophy}
  
\input{cpright}

%% Abstract
\begin{abstract}%[\smaller \thetitle\\ \vspace*{1cm} \smaller {\theauthor}]
Here the abstract of the thesis
\end{abstract}


%% Declaration
\begin{declaration}
  This dissertation is the result of my own work, except where explicit
  reference is made to the work of others, and has not been submitted
  for another qualification to this or any other university. This
  dissertation does not exceed the word limit for the respective Degree
  Committee.
  \vspace*{1cm}
  \begin{flushright}
    João Pela
  \end{flushright}
\end{declaration}


%% Acknowledgements
\begin{acknowledgements}

% Replace FCT for acronym

\colorbox{red}{
\begin{minipage}{0.8\linewidth}
  
TODO:

\begin{itemize}
  \item Family
  \item Friends
  \item Work colleagues (include CMS collaboration)
  \item more
\end{itemize}

\end{minipage}
}
\end{acknowledgements}

\clearpage

\vspace*{\fill}
The work presented in this thesis was supported by the Portuguese Government through \gls{FCT} in the form of my PhD grant with the reference SFRH/BD/77979/2011. I am thankful for their support which allowed me to attain higher education.
\vspace*{\fill}

\begin{center}
\resizebox{1.0\linewidth}{!}{
\includegraphics{FrontMatter/Images/FundingBand.png}
}
\end{center}


\begin{preface}
Thesis structure and so on...
\end{preface}

\dedication{To my grand mother}

%% Preface
%\begin{preface}
%\end{preface}

%% ToC
\tableofcontents

%% Strictly optional!
%\frontquote%
%{Writing in English is the most ingenious torture\\
%   ever devised for sins committed in previous lives.}%
%  {James Joyce}
 
% \titlepage[of the High Energy Physics Group]{%
%   A dissertation submitted to the Imperial College London\\ for the degree of Doctor of Philosophy}
                     %Status: DONE
\end{frontmatter}

\begin{mainmatter}
   \chapter{Theory and motivations}
\label{CHAPTER:TheoryAndMotivations}

\glsresetall % Resetting all acronyms

%%%%%%%%%%%%%%%%%%%%%%%%%%%%%%%%%%%%%%%%%%%%%%%%%%%%%%%%%%%%%%%%%%%%%%%%%%%%%%%%%%%%%%%
%%% SECTION
%%%%%%%%%%%%%%%%%%%%%%%%%%%%%%%%%%%%%%%%%%%%%%%%%%%%%%%%%%%%%%%%%%%%%%%%%%%%%%%%%%%%%%%
\section{Standard Model of Particle Physics}

% Status: DONE

The \gls{SM} of particle physics is a quantum field theory that describes the electromagnetic, weak nuclear and strong forces and their interaction with matter. This theory is one of the most successful theories ever made and was able to describe data from a wide range of experimental measurements. Before its discovery in 2012 \cite{ARTICLE:ATLAS_HiggsDiscovery,ARTICLE:CMS_HiggsDiscovery} the Higgs boson was the only missing particle that was predicted by this theory and not yet found. 

Although its success, the \gls{SM} does not explain some phenomena observed in nature, like the presence of large quantity of \textit{dark matter} in the universe, or the even more mysterious \textit{dark energy}. The discovery of the Higgs boson could allow to probe the production of dark matter directly, through its decay into these elusive particles. This chapter briefly describes the theory behind the \gls{SM}, the Higgs mechanism, how to search for Higgs invisible decays.

%%%%%%%%%%%%%%%%%%%%%%%%%%%%%%%%%%%%%%%%%%%%%%%%%%%%%%%%%%%%%%%%%%%%%%%%%%%%%%%%%%%%%%%
%%% SUBSECTION
%%%%%%%%%%%%%%%%%%%%%%%%%%%%%%%%%%%%%%%%%%%%%%%%%%%%%%%%%%%%%%%%%%%%%%%%%%%%%%%%%%%%%%%
\subsection{Particles and forces}
\label{SUBSECTION:Theory_SM_ParticlesAndForces}



\begin{table}[!htb]
  \centering
  \begin{tabular}{|c|c|c|c|c|}
  \hline
  \multicolumn{5}{|c|}{Leptons (J=1/2)} \\
  \hline
  Generation & Particle Name & Symbol & Mass & Q/e \\
  \hline
  \hline
  \multirow{2}{*}{$1^{st}$} & Electron          & e          &     $511\,\keV$ & 1 \\
                            & Electron Neutrino & $\nu_e$    &     $< 2\, \eV$ & 0 \\
  \hline
  \hline
  \multirow{2}{*}{$2^{nd}$} & Muon              & $\mu$      &  $   106\,\MeV$ & 1 \\
                            & Muon Neutrino     & $\nu_\mu$  &  $< 0.19\,\MeV$ & 0 \\
  \hline
  \hline
  \multirow{2}{*}{$3^{rd}$} & Tau               & $\tau$     & $   1777\,\MeV$ & 1 \\
                            & Tau Neutrino      & $\nu_\tau$ & $< 18.2 \,\MeV$ & 0 \\
  \hline
  \end{tabular}
  \caption[List of leptons and their fundamental properties]{List of leptons and their fundamental properties \cite{ARTICLE:PDG2014}}
  \label{TheoreticalIntroduction_LeptonProperties}
\end{table}

\begin{table}[!htb]
  \centering
  \begin{tabular}{|c|c|c|c|c|}
  \hline
  \multicolumn{5}{|c|}{Quarks (J=1/2)} \\
  \hline
  Generation & Particle Name & Symbol & Mass ($GeV/c^2$) & Q/e \\
  \hline
  \hline
  \multirow{2}{*}{$1^{st}$} & Up      & u & $1.5-3.3 \times 10^{-3}$ & -2/3 \\
                            & Down    & d &   $3.5-6 \times 10^{-3}$ &  1/3 \\
  \hline
  \hline
  \multirow{2}{*}{$2^{nd}$} & Charm   & c &                1.16-1.34 & -2/3 \\
                            & Strange & s &  $70-130 \times 10^{-3}$ &  1/3 \\
  \hline
  \hline
  \multirow{2}{*}{$3^{rd}$} & Top     & t &                  169-173 & -2/3 \\
                            & Bottom  & b &              $4.13-4.37$ &  1/3 \\
  \hline
  \end{tabular}
  \caption[List of quarks and their fundamental properties]{List of quarks and their fundamental properties}
  \label{TheoreticalIntroduction_QuarkProperties}
\end{table}

\begin{table}[!htb]
  \centering
  \begin{tabular}{|c|c|c|c|c|}
  \hline
  \multicolumn{4}{|c|}{Bosons} \\
  \hline
   Particle Name & Mass ($GeV/c^2$) &     Q/e & Spin \\
  \hline
  \hline
  Photon ($\gamma$) &               0 &       0 &    1 \\
  \hline
  $W^\pm$           &            80.4 & $\mp 1$ &    1 \\
  $Z^0$             &            91.2 &       0 &    1 \\
  \hline
  Gloun (g)         &               0 &       0 &    1 \\
  Higgs ($H^0$)     &         $> 114$ &       0 &    0 \\
  \hline
  \end{tabular}
  \caption[List of bosons and their fundamental properties]{List of bosons and their fundamental properties}
  \label{TheoreticalIntroduction_BosonProperties}
\end{table}


%%%%%%%%%%%%%%%%%%%%%%%%%%%%%%%%%%%%%%%%%%%%%%%%%%%%%%%%%%%%%%%%%%%%%%%%%%%%%%%%%%%%%%%
%%% SUBSECTION
%%%%%%%%%%%%%%%%%%%%%%%%%%%%%%%%%%%%%%%%%%%%%%%%%%%%%%%%%%%%%%%%%%%%%%%%%%%%%%%%%%%%%%%
\subsection{The Higgs mechanism}

Summary of the Higgs Mechanism. Should include
\begin{itemize}
 \item Motivations 
 \item Explanation of the mechanism itself
 \item Consequences 
 \item Possible decays
\end{itemize}

%%%%%%%%%%%%%%%%%%%%%%%%%%%%%%%%%%%%%%%%%%%%%%%%%%%%%%%%%%%%%%%%%%%%%%%%%%%%%%%%%%%%%%%
%%% SUBSECTION
%%%%%%%%%%%%%%%%%%%%%%%%%%%%%%%%%%%%%%%%%%%%%%%%%%%%%%%%%%%%%%%%%%%%%%%%%%%%%%%%%%%%%%%
\subsection{Searching for the SM Higgs boson}


\begin{figure}[htbp]
\subfloat[]{\includegraphics[width=0.45\textwidth]{Chapter01/Images/feynman_ggH.pdf}} \qquad
\subfloat[]{\includegraphics[width=0.45\textwidth]{Chapter01/Images/feynman_qqH.pdf}} \\
\subfloat[]{\includegraphics[width=0.45\textwidth]{Chapter01/Images/feynman_VH.pdf}}  \qquad
\subfloat[]{\includegraphics[width=0.45\textwidth]{Chapter01/Images/feynman_ttH.pdf}} \\
\caption[Feynman diagrams for the dominant production processes of the SM Higgs
boson.]{Feynman diagrams for the dominant production processes of the SM Higgs
boson. Shown is a) gluon fusion, b) vector boson fusion and
associated production with c) vector bosons and d) top quarks.}
\label{fig:SMFeynmanDiagrams}
\end{figure}

\begin{figure}[htbp]
 \includegraphics[width=0.7\textwidth]{Chapter01/Images/Higgs_XS_8TeV_lx.pdf}
\caption[Cross sections for Higgs production processes at $\sqrt{s}=8\,\TeV$ for
a range of Higgs boson masses.]{Cross sections for Higgs production processes at
$\sqrt{s}=8\,\TeV$ for a range of Higgs boson masses $m_{\PH}$~\cite{Heinemeyer:2013tqa}. Across the
mass range the gluon-fusion mode dominates, followed by the vector boson fusion
and associated production modes. The widths of the lines represent the
theoretical uncertainties on the cross section calculation.}
\label{fig:SMHiggsXS}
\end{figure}

\begin{figure}[htbp]
 \includegraphics[width=0.6\textwidth]{Chapter01/Images/Higgs_BR.pdf}
\caption[Higgs boson branching ratios in the SM for a range of Higgs boson
masses.]{Higgs boson branching ratios in the SM for a range of Higgs boson
masses $m_{\PH}$ \cite{Heinemeyer:2013tqa}. At high masses, above their
kinematic thresholds, the $\PW\PW$,
$\PZ\PZ$ and $\ttbar$ (shown in red) decay modes dominate. 
At lower masses a wide range of different final states is possible. 
The widths of the lines represent the
theoretical uncertainties on the branching ratio calculation.}
\label{fig:SMHiggsBRs}
\end{figure}

\subsection{Higgs Invisible decays}

\colorbox{red}{
\begin{minipage}{0.95\linewidth}
TODO: 
\begin{itemize}
  \item Explain what are SM Higgs invisible decays.
  \item Go over the possibility of BSM invisible decays.
\end{itemize}

\end{minipage}
}

% Adding a bunch of references
\cite{ARTICLE:Higgs_SpontaneousSymmetryBreakdown}
\cite{BOOK:Griffiths}

                        %Status: DONE
   \chapter{Experimental Apparatus}
\label{CHAPTER:ExperimentalApparatus}

%%%%%%%%%%%%%%%%%%%%%%%%%%%%%%%%%%%%%%%%%%%%%%%%%%%%%%%%%%%%%%%%%%%%%%%%%%%%%%%%%%%%%%%
%%% SECTION
%%%%%%%%%%%%%%%%%%%%%%%%%%%%%%%%%%%%%%%%%%%%%%%%%%%%%%%%%%%%%%%%%%%%%%%%%%%%%%%%%%%%%%%
\section{The Large Hadron Collider}
\label{SECTION:ExperimentalApparatus_LHC}

% Status: DONE (reviewed J.Pela x1)
%
% TODO:
% * DONE: LHC location, size, particles used, energy usage.
% * Basics of machine and operation
% * How instantaneous luminosity is calculated include Instantaneous luminosity equation
% * Delivered instantaneous luminosity Run I (proton-Proton)

% \colorbox{red}{
% \begin{minipage}{0.95\linewidth}
%  
% \begin{itemize}\item\end{itemize}
% \end{minipage}
%}

% CERN and LHC Location
The \gls{LHC}\cite{ARTICLE:LHC Machine} is currently the world's largest particle accelerator and is capable to produce the highest energy particle beam ever made by mankind. This gigantic machine with a total perimeter of $26.7\,\kilo\meter$ was built at \gls{CERN} in a circular tunnel, where previously the \gls{LEP}\cite{LEPTDR:LEPInjectorStudyGroup} was installed, at an average depth of $100\,\meter$ below ground under the Franco-Swiss border near Geneva, Switzerland. A diagram of the \gls{LHC} tunnel and its experiments can be found at figure \ref{FIGURE:ExperimentalApparatus_LHCLayoutUnderground}.

\begin{figure}[!htb]
  \centering
  \includegraphics[width=0.50\textwidth]{Chapter02/LHC/Images/LHC_layout_underground.jpg}
  \caption{Underground diagram of the Geneva area showing the \gls{LHC} and its experiments location.}
  \label{FIGURE:ExperimentalApparatus_LHCLayoutUnderground}
\end{figure}

% Structure and experiments
The \gls{LHC} is a synchrotron machine with the capability to accelerate two particles beams in opposite directions in two separated beam pipes. These beams only cross and are allowed to collide in four points of the accelerator where huge particle detectors are installed to detect the products of such collisions. This experiments are: \gls{ATLAS}\cite{ARTICLE:TheATLASExperiment}, \gls{CMS}\cite{ARTICLE:TheCMSExperiment}, \gls{LHCb}\cite{ARTICLE:TheLHCbExperiment} and \gls{ALICE}\cite{ARTICLE:TheALICEExperiment}.
% Add other experiments?

% Objective
The objective of the \gls{LHC} program is to investigate physics at the $\TeV$ scale, more specifically to understand the electroweak symmetry breaking and if this phenomena could be explained by the Higgs mechanism. And of course search for any indications of \gls{BSM} physics. \gls{ATLAS} and \gls{CMS} are general-purpose detectors which aim to investigate a broad spectrum of physics. The \gls{LHCb} detector is used to study processes that involve the decay of b-flavoured hadrons. The \gls{ALICE} detector is optimised to look at heavy-ion collisions and to investigate the properties of extreme high density medium formed in those collisions.

% CERN accelerator chain
The \gls{LHC} is only the last element of a complex accelerator chain which step by step increases the energy of the particles to eventually be collided. Protons are initially obtained by stripping the electrons of hydrogen gas. This process happens at the begging of the \gls{LINAC2} which then accelerates them up to the energy of $50\,\MeV$. After this initial step proton are injected into the \gls{PSB} and the energy ramps ups to $1.4\,\GeV$. Particles are then passed to the \gls{PS} where the energy futher increases to $25\,\GeV$, subsequently they are the injected into the \gls{SPS} where the particle energy level reached $450\,\GeV$. Finally, protons pass to the \gls{LHC} where they can be accelerated to a maximum energy of $7\,\TeV$. A simplified diagram of the \gls{CERN} accelerator chain can be found in figure \ref{FIGURE:ExperimentalApparatus_LHCAccelaratorChain}. 
Normal operation of the \gls{LHC} therefore depends on all the upstream accelerators availability. The typically turn around time, the time necessary to stop the accelerator from running and restart collisions is around 2 hours. When stable beams are achieved, a single proton fill can be used to collide protons up to 24 hours, but it is common to restart more frequently to take profit of the higher collision rates possible right at the beginning of a new fill.

\begin{figure}[!htb]
  \centering
  \includegraphics[width=0.50\textwidth]{Chapter02/LHC/Images/LHCAccelaratorChain.png}
  \caption{CERN Large Hadron Collider Experiment accelerator diagram.}
  \label{FIGURE:ExperimentalApparatus_LHCAccelaratorChain}
\end{figure}


% LHC modes of opperation
The \gls{LHC} as its name indicates can collide hadrons, more specifically proton or heavy ions. Three modes of operation have been tried according to the particles being collided: proton-proton, proton-lead and lead-lead. By changing the incoming particles we are changing the quantity of nucleons present at each interaction. The maximum design energy per proton is $7\,\TeV$ and for each lead nucleon $2.76\,\TeV$. The maximum design luminosity for proton-proton is of $10^{34}\,\cm^{-2}\second^{-1}$ and for lead-lead is of $10^{27}\,\cm^{-2}\second^{-1}$.

% LHC structure
Particles beams trajectory are curved by 1232 niobium-titanium superconducting dipole magnets each with a length of $14.3\,\meter$ which are cooled with superfluid helium to $1.9\,\kelvin$ to be able to produce a magnetic field of $8.4\,\tesla$. To accelerate the beam it uses eight \gls{RF} cavities which increase each bunch energy at every turn of the \gls{LHC}. At nominal operation the \gls{LHC} will steer 2808 bunches separated by $25\,\ns$ in each direction each bunch composed up to $10^{11}$ protons. Some of the key parameters of the \gls{LHC} proton-proton and lead-lead operation can be found in table \ref{TABLE:ExperimentalApparatus_LHCMachineParameters}.

\begin{table}[!htb]
  \centering
  \begin{threeparttable}
    \begin{tabular}{|lcccc|}
    \hline 
                                  &              &           \textit{pp} &         \textbf{HI} &  \\
    \hline
    Energy per nucleon            & E            &                     7 &                2.76 &                 $\TeV$ \\
    Dipole field at 7 TeV         & \textit{B}   &                  8.33 &                8.33 &               $\tesla$ \\
    Design Luminosity\tnote{*}    & $\mathcal{L}$ &            $10^{34}$ &           $10^{27}$ & $\cm^{-2}\second^{-1}$ \\
    Bunch separation              &              &                    25 &                 100 &                  $\ns$ \\
    No. of bunches                & $k_B$        &                  2808 &                 592 &                        \\
    No. particles per bunch       & $N_p$        & $1.15 \times 10^{11}$ & $7.0 \times 10^{7}$ &                        \\
    \hline
    \hline
    \textbf{Collisions}           &              &  &  &  \\
    \hline
    $\beta$-value at IP           & $\beta^{*}$  &                  0.55 &                 0.5 &        $\meter$ \\
    RMS beam radius at IP         & $\sigma^{*}$ &                  16.7 &                15.9 &  $\micro\meter$ \\
    Luminosity lifetime           & $\tau_L$     &                    15 &                   6 &         $\hour$ \\
    Number of collisions/crossing & $n_c$        &          $\approx 20$ &                   - &                 \\
    \hline
    \end{tabular}
    \begin{tablenotes}
      \item[*] For heavy-ion (HI) operation the design luminosity for Pb-Pb collisions is given.
    \end{tablenotes}
  \end{threeparttable}
  \caption[LHC parameters relevant for detectors]{The machine parameters relevant for the 
                                                  LHC detectors.\cite{CMSTDR:CMSPhysicsVol1}}
  \label{TABLE:ExperimentalApparatus_LHCMachineParameters}
\end{table}


At the \gls{LHC} we are looking for extremely rare processes as is can be seen in figure \ref{FIGURE:ExperimentalApparatus_LHCCrossSections} the production cross section of a \gls{SM} Higgs boson is more than 9 orders of magnitude smaller than the total proton-proton cross section. 

\begin{figure}[!htb]
  \centering
  \includegraphics[width=0.50\textwidth]{Chapter02/LHC/Images/crosssections2012_v5}
  \caption{Cross sections for several processes for collisions of antiproton-proton and proton-proton as a function of the center of mass energy\cite{ARTICLE:TheCMSExperiment}.}
  \label{FIGURE:ExperimentalApparatus_LHCCrossSections}
\end{figure}

To be able to record and study such rare processes we need to produce a significant amount of collisions. For this purpose the LHC was designed to operate at high instantaneous luminosity, L. This quantity is defined as,

\begin{equation}
L=\frac{N_{b}^{2}n_{b}f_{\text{rev}}\gamma}{4\pi\epsilon_{n}\beta^{*}}F,
\end{equation}

where $N_{b}$ is the number of protons per bunch, $n_{b}$ is the number of bunches, $f_{\text{rev}}$ is the frequency of revolution, $\gamma$ is the Lorentz factor, $\epsilon_{n}$ is the normalized emittance, $f_{\text{rev}}$ is the beta function at the collision point and $F$ is the reduction factor due to the crossing angle.

%%%%%%%%%%%%%%%%%%%%%%%%%%%%%%%%%%%%%%%%%%%%%%%%%%%%%%%%%%%%%%%%%%%%%%%%%%%%%%%%%%%%%%%
%%% SUBSECTION
%%%%%%%%%%%%%%%%%%%%%%%%%%%%%%%%%%%%%%%%%%%%%%%%%%%%%%%%%%%%%%%%%%%%%%%%%%%%%%%%%%%%%%%
\subsection{Running and performance}

%Historical
The \gls{LHC} has started its operation with first circulation beams in September 2008. Unfortunately, only a few days after a faulty weld between two dipole magnets caused a significant magnet quench which in turn damaged several dipoles with a simultaneous leak of a significant amount of helium. The event showed that beyond the repair of the affected systems the accelerator needed a significant consolidation program to allow it to return it to activity\cite{ARTICLE:CMSReportIncident19Sep2008}. This consolidation program took over one year to finalise and to prevent further possible problems and allow better understanding of the machine while maximizing physics reach, it was decided to run the \gls{LHC} at $7\,\TeV$ center-of-mass energy.
The period the follow in know as \gls{LHC} run I which started with first collisions at November 2009 just at the \gls{SPS} injection energy of $450\,\GeV$. 

The collision energy was finally ramped up to $7\,\TeV$ with first collisions being observed during March 2010. Operation at this energy continued until the end of 2011, with the peak luminosity being achieved of $3.7 \times 10^{33} \centi\meter^{-2}\second^{-1}$. The total amount of of integrated luminosity delivered to \gls{CMS} was $6.1\,\femto\barn^{-1}$ with the total actually recorded being $5.6\,\femto\barn^{-1}$. During 2012 with the increase knowledge of the accelerator it was possible to increase the centre-of-mass energy further to $8\,\TeV$ and eventually reaching peak luminosity of $7.7 \times 10^{33}\,\centi\meter^{-2}\second^{-2}$ and delivering $23.3\,\femto\barn^{-1}$ to \gls{CMS} of which $21.79,\femto\barn^{-1}$  $\femto\barn^{-1}$ were recorded. Figure \ref{FIGURE:ExperimentalApparatus_CMS_IntegratedLumi_pp_2010-2012} shows the delivered luminosity in the period 2010-2013 against time.

\begin{figure}[!htb]
  \centering
  \includegraphics[width=1.00\textwidth]{Chapter02/CMS/Images/CMS_IntegratedLumi_pp_2010-2012}
  \caption{Cumulative luminosity versus day delivered to CMS during stable beams and for p-p collisions. This is shown for 2010 (green), 2011 (red) and 2012 (blue) data-taking.}
  \label{FIGURE:ExperimentalApparatus_CMS_IntegratedLumi_pp_2010-2012}
\end{figure}

For actual physics usage data needs to undergo the process of certification. In this process specialist from each \gls{CMS} subsystem check that no problem has happened during data taking that would bias or invalidate the recorded events. For 2011 a total of $5.1\,\femto\barn^{-1}$ and for 2012 a total $19.7\,\femto\barn^{-1}$ were considered of good quality for physics. 

In order to achieve high integrated luminosity the \gls{LHC} collides particle bunches 40 millions times a second, each time this happens many interactions may happen simultaneously, this effect is called \gls{PU}. A figure of the distribution of the mean number of interaction per bunch crossing during 2012 at the CMS experiment can be found in figure \ref{FIGURE:ExperimentalApparatus_CMS_PileIp_pp_2012}.

\begin{figure}[!htb]
  \centering
  \includegraphics[width=0.60\textwidth]{Chapter02/CMS/Images/CMS_PileIp_pp_2012}
  \caption{Mean number of interactions per bunch crossing at the CMS experiment during 2012.}
  \label{FIGURE:ExperimentalApparatus_CMS_PileIp_pp_2012}
\end{figure}

%%%%%%%%%%%%%%%%%%%%%%%%%%%%%%%%%%%%%%%%%%%%%%%%%%%%%%%%%%%%%%%%%%%%%%%%%%%%%%%%%%%%%%%
%%% SECTION
%%%%%%%%%%%%%%%%%%%%%%%%%%%%%%%%%%%%%%%%%%%%%%%%%%%%%%%%%%%%%%%%%%%%%%%%%%%%%%%%%%%%%%%
\section{The Compact Muon Solenoid Experiment}
\label{SECTION:ExperimentalApparatus_CMS}

%Status: Done (needs review)

The \glsreset{CMS} experiment is a general purpose experiment located at point 5 of the \gls{LHC}. It was designed to be a high performance detector to study collisions at its centre. It is composed os several sub-systems in an classic onion shaped structure. A diagram of the experiment can be found in figure \ref{FIGURE:ExperimentalApparatus_CMS_Layout_Diagram}.

\begin{figure}[!htb]
  \centering
  \includegraphics[width=1.00\textwidth]{Chapter02/CMS/Images/CMS_Layout_Diagram.pdf}
  \caption{Diagram of \gls{CMS} experiment showing the experiment in an open configuration and highlighting the position of its sub-detectors.}
  \label{FIGURE:ExperimentalApparatus_CMS_Layout_Diagram}
\end{figure}

The main driving motivation for its design is to investigate the electroweak symmetry breaking for which the Higgs mechanism at the design time was presumed to be the most likely explanation. Many other alternative theories to the standard model predict new particles which could be observed at the $\TeV$ scale, \gls{CMS} as a multi-purpose experiment is well suited to search for new scenarios. If found, such new physics may allow us to understand some currently open particle physics issues, like providing particle candidates for dark dark matter. Further, some of this possible new physics signals could point the way towards a grand unified theory. \gls{CMS} is also capable of operating while the \gls{LHC} is colliding heavy ions and has a rich program covering the study of \gls{QCD} matter at extreme temperatures, density and parton momentum fraction (low-x).

The requirements imposed to \gls{CMS} design to meet its physics goals can be summarized in the following table\cite{ARTICLE:TheCMSExperiment}:

\begin{itemize}
  \item Good muon identification and momentum resolution over a wide range of momenta and angles, good dimuon mass resolution ($\approx 1\%$ at $100\,GeV$), and the ability to determine un-ambiguously the charge of muons with $\pt<1\,\TeV$;
  \item Good charged-particle momentum resolution and reconstruction efficiency in the inner tracker. Efficient triggering and offline tagging of $\tau$'s and b-jets, requiring pixel detectors close to the interaction region;
  \item Good electromagnetic energy resolution, good diphoton and dielectron mass resolution ($\approx 1\%$ at $100\,\GeV$), wide geometric coverage, $\pi^0$ rejection, and efficient photon and lepton
isolation at high luminosities;
  \item Good missing-transverse-energy and dijet-mass resolution, requiring hadron calorimeters with with a large hermetic geometric coverage and with fine lateral segmentation.
\end{itemize}

%TODO: Maybe re-write this!!!
The chosen detector design fulfils all this requirements. The experiment is compact compared with the other \gls{LHC} experiments being $22\,\meter$ long and $15\,\meter$ in diameter. But, it is the heaviest of the four big detectors at $12500\,\ton$. Its high density is a direct consequence of it producing the highest magnetic field at $4\,\tesla$ and therefore needing more material to serve as a return yoke. On the next section we will go in detail over the features and technologies used.

%%%%%%%%%%%%%%%%%%%%%%%%%%%%%%%%%%%%%%%%%%%%%%%%%%%%%%%%%%%%%%%%%%%%%%%%%%%%%%%%%%%%%%%
%%% SUBSECTION
%%%%%%%%%%%%%%%%%%%%%%%%%%%%%%%%%%%%%%%%%%%%%%%%%%%%%%%%%%%%%%%%%%%%%%%%%%%%%%%%%%%%
\subsection{Geometry and conventions}

% STATUS: DONE 

The adopted coordinate system has it origin in the center of \gls{CMS} where the nominal collision point is located, the y-axis points vertically upwards, and the x-axis points radially inward in the direction of the center of the \gls{LHC}. The z-axis points along the beam line towards the Jura mountains from the \gls{LHC} point 5. The azimuthal angle $\phi$ is measured from the x-axis in the x-y plane. The polar angle $\theta$ is measured from the z-axis.

We define pseudorapidity as $\eta = -ln(tan(\theta/2))$. All transverse quantities, like the transverse momentum ($p_\perp$), are measured relative to the beam direction. The imbalance of energy is also measured in the x-y plane and is denoted as $E^{miss}_\perp$.

%%%%%%%%%%%%%%%%%%%%%%%%%%%%%%%%%%%%%%%%%%%%%%%%%%%%%%%%%%%%%%%%%%%%%%%%%%%%%%%%%%%%%%%
%%% SUBSECTION
%%%%%%%%%%%%%%%%%%%%%%%%%%%%%%%%%%%%%%%%%%%%%%%%%%%%%%%%%%%%%%%%%%%%%%%%%%%%%%%%%%%%
\subsection{Inner tracking system}
\label{SUBSECTION:ExperimentalApparatus_CMS_Tracker}

% STATUS: DONE (reviwed J.Pela x1)
%
% Writing points:
% * Closest detector to beam, measures trajectories of charged particles
% * With magnetic field measures momentum and charge of this particles
% * Allows vertexing (primary and secundary)
% * Different regions different occupancy
% * Final arranagement

The inner tracking system is the closest detector to the beam axis and the interaction region. Its function is to measure the trajectory of all charged particles, like electrons, charged hadrons and muons with momentum above 1 $\GeV$ being produced at each \gls{LHC} collision. With the help of the strong magnetic field produced by the \gls{CMS} magnet particle trajectories are bent allowing for charge and momentum determination. With the resulting tracks is it then possible to determine the primary vertex as well as secondary vertexes like other lower energy proton-proton collision or displaced vertexes from the decay of long lived particles like B mesons.

Building a tracking system for an experiment at the \gls{LHC} is very challenging. At design luminosity an average of 1000 particles will hit such system at a rate approaching $40\,\mega\hertz$, leading to high hit density at high rate. It is therefore desirable to have a fast, efficient and high granularity detector where at each layer the occupancy should be at or below $1\%$. On the other hand each layer should be as thin as possible in order to not change the incoming particles trajectory or make them lose too much energy. The detector should also be radiation hard and survive for a period of at least 10 years due to its importance and location. This design requirements have lead to a tracker design entirely based on silicon detector technology. 

The volume near the interaction point can be split according to the charged particle flux into three regions:

\begin{itemize}
  \item $r<10\,\centi\meter$: highest particle flux, up to $\approx 10^8 \centi\meter^{-2}\second^{-1}$ at $r \approx 4 \centi\meter$, pixel detectors are used. The pixel size is $\approx 100 \times 150\,\micro\meter^2$, which translates into an occupancy of $10^{-4}$ per \gls{LHC} bunch crossing.
  \item $20<r<55\,\centi\meter$: particle flux decreases enough to use silicon micro-strips with a minimum cell size of 10 cm $\times 80\,\micro\meter$, leading to an occupancy of $\approx 2-3\%$ per \gls{LHC} bunch crossing.
  \item $50<r<110\,\centi\meter$: most outer region of the tracker, particle flux is low enough to use larger pitch silicon micro-strips. The maximum cell size is of $25\,\centi\meter$ $\times$ 180 $\micro\meter$, and occupancy is of the order of $\approx 1\%$.
\end{itemize}

The \gls{CMS} tracker final configuration is composed of a pixel detector with three barrel layers at radii between $4.4\,\centi\meter$ and $10.2\,\centi\meter$ and 2 disks on each side of the barrel. And a silicon strip tracker with 10 barrel detection layers extending up to $1.1\,\meter$ with 3 plus 9 disks on each side of the barrel. A schematic of the detector module distribution can be found at figure \ref{FIGURE:ExperimentalApparatus_CMS_Tracker_Layout}. This detector has an acceptance covering up to pseudorapidity of $|\eta|<2.5$ and has a total active area of about $200\,\meter^2$ making the largest silicon tracker ever built. 

\begin{figure}[!htb]
  \centering
  \includegraphics[width=1.0\textwidth]{Chapter02/CMS/Images/CMS_Tracker_Layout.png}
  \caption{Schematic cross section of the CMS tracker. Each line represent a detector module. Double lines represent dual surface back-to-back detector modules.}
  \label{FIGURE:ExperimentalApparatus_CMS_Tracker_Layout}
\end{figure}

%%%%%%%%%%%%%%%%%%%%%%%%%%%%%%%%%%%%%%%%%%%%%%%%%%%%%%%%%%%%%%%%%%%%%%%%%%%%%%%%%%%%%%%
%%% SUBSECTION
%%%%%%%%%%%%%%%%%%%%%%%%%%%%%%%%%%%%%%%%%%%%%%%%%%%%%%%%%%%%%%%%%%%%%%%%%%%%%%%%%%%%
\subsection{Electromagnetic Calorimeter}
\label{SUBSECTION:ExperimentalApparatus_CMS_ECAL}

% Status: DONE (just with MSc input, and needs review)

The \gls{ECAL} is the detector responsible for measuring the energy of electrons and photons. It is an hermetic energy measurement system comprised of 61200 lead tungstate ($PbWO_4$) crystals mounted in the barrel and 7324 crystals in each of the 2 endcaps and it has an acceptance up to $|\eta|<3.0$.

Lead tungstate has a fairly high density ($8.28\,\gram/\centi\meter^3$), has a short radiation length ($0.89\,\centi\meter$) and a small Moliere redius ($2.2\,\centi\meter$). The crystals also have a fast scintillation decay time emitting 80\% of the light yield in $25\,\nano\second$ (the minimal bunch crossing time at the \gls{LHC}). This characteristics make it a good choice for an electromagnetic calorimeter allowing a compact design with fine granularity. However, this crystals emit a fairly low light yield ($30\,\gamma/\MeV$) which requires the use of photo-detectors with intrinsic gain which will preform well inside a magnitic field. In the barrel region silicon \gls{APD} are used and \gls{VPT} are used in the endcaps. To guarantee good response from both crystals and \gls{APD} it is necessary to have system thermal stability, with the goal being temperature variation of less than $0.1 \celsius$.

The barrel section, the \gls{EB}, has an inner radius of $129\,\centi\meter$ and is composed of 36 identical ``supermodules``, each covers the barrel length and corresponding to a pseudo-rapidity interval of $0<|\eta|<1.479$. The crystals are quasi-projective (the axes are tilted at 3º with respect to the line from the nominal vertex position) and cover 0.0174 (i.e. 1º ) in $\Delta\phi$ and $\Delta\eta$. The crystals have a front face cross-section of $\approx 22 \times 22\,\milli\meter^2$ and a length of $230\,\milli\meter$, corresponding to 25.8 $X_0$.

The endcap section, the \gls{EE}, is at a distance of $314\,\centi\meter$ from the vertex and covering a pseudorapidity range of $1.479<|\eta|<3.0$, are each structured as 2 “Dees”, consisting of semi-circular aluminium plates from which are cantilevered structural units of $5\times 5$ crystals, known as “supercrystals”.

figure \ref{FIGURE:ExperimentalApparatus_CMS_ECAL_Layout}

% \begin{figure}[!htb]
%   \centering
%   \includegraphics[width=1.0\textwidth]{Chapter02/CMS/Images/CMS_ECAL_Layout.png}
%   \caption{}
%   \label{FIGURE:ExperimentalApparatus_CMS_ECAL_Layout}
% \end{figure}
% 
% The energy resolution of the ECAL can be parameterized as:
% \begin{equation}
% \frac{\sigma}{E} = \frac{S}{\sqrt{E}} \oplus \frac{N}{E} \oplus C
% \end{equation}
% where $E$ is the energy of the incident particle and $S$, $N$ and $C$ are known as the stochastic, noise and constant terms respectively. The stochastic term encapsulates fluctuations in the scintillation and lateral containment of the shower; the noise term originates from the electronics and digitisation; and the constant term from non-uniform longitudinal response and inter-calibration errors. These have been measured in an electron beam test as $S=0.028\,\GeV^{1/2}$, $N=0.12\,\GeV$ and $C=0.003$, although without the presence of a magnetic field or material in front of the \ac{ECAL}.

%%%%%%%%%%%%%%%%%%%%%%%%%%%%%%%%%%%%%%%%%%%%%%%%%%%%%%%%%%%%%%%%%%%%%%%%%%%%%%%%%%%%%%%
%%% SUBSECTION
%%%%%%%%%%%%%%%%%%%%%%%%%%%%%%%%%%%%%%%%%%%%%%%%%%%%%%%%%%%%%%%%%%%%%%%%%%%%%%%%%%%%
\subsection{Hadronic Calorimeter}
\label{SUBSECTION:ExperimentalApparatus_CMS_HCAL}

% STATUS: DONE

The \glsreset{HCAL} is a sampling calorimeter which is designed to measure the properties of hadron jets and indirectly neutrinos or other undiscovered particles that would result in apparent missing energy\cite{ARTICLE:CMSTechnicalProposal}. The design of the \gls{HCAL} was strongly influenced by the choice of the magnet parameters since most of the calorimetry is inside of the magnet. A diagram of the \gls{HCAL} subsystems and their location inside \gls{CMS} can be found in figure \ref{FIGURE:ExperimentalApparatus_CMS_HCAL_Layout}.

\begin{figure}[!htb]
  \centering
  \includegraphics[width=1.0\textwidth]{Chapter02/CMS/Images/CMS_HCAL_Layout.png}
  \caption{Longitudinal view of the CMS detector highlighting the location of the \gls{HCAL} components: \gls{HB}, \gls{HE} \gls{HO} and \gls{HF}.}
  \label{FIGURE:ExperimentalApparatus_CMS_HCAL_Layout}
\end{figure}

The \glsreset{HB} covers the region up to $|\eta|<1.3$ and is limited from the beam side by the \gls{ECAL} at radius $r=1.77\,\meter$ and outwards by the magnet at radius $r=2.95\,\meter$. This is a strict limitation on the amount of absorber material to be used. This detector is composed of 36 identical azimuthal wedges with for two half-barrels. They are constructed of brass absorber plates alternated with plastic scintillator. Brass has a short interaction length ($X_0=16.42\,\centi\meter$) and ins non-magnetic. The detector is composed of 2304 towers with a segmentation of $\Delta\eta \times \Delta\phi = 0.087 \times 0.087$ which corresponds to the same area of a $5 \times 5$ arrays of \gls{ECAL} crystals.

To improve the measurement capability, an outer calorimeter, the \glsreset{HO}, is placed outside of the magnet as a \textit{tail catcher}. It increases the effective thickness of the hadronic calorimeter by over 10 interaction lengths. This detector covers the range $|\eta|<1.26$, it is composed or iron absorber and scintillator and is subdivided into sectors that cover 30º azimuthal angle in each of the barrel wheels. 

The \glsreset{HE} covers the range of $1.3<|\eta|<3.0$. It is composed by 2034 towers with a 14 towers segmentation in $\eta$ and 5º segmentation in $\phi$. The 8 inner most towers the segmentation is 10º in $\phi$, whilst the $\eta$ segmentation increases in $\eta$ from 0.09 to 0.35.

Additionally, to extend acceptance to $|\eta|<5.2$ the \gls{HF} is installed at $11.2\,\meter$ from the interaction point providing excellent hermeticity for $E_{\perp}^{miss}$ measurement. Its steel absorber is $1.65\,\meter$ deep and had quartz fibers running through it parallel to the beam line. The energy measurement is made via Cerenkov light produced by the incoming particles inside the fibers. There are 13 tower in $\eta$ with segmentation of $\approx \Delta\eta=0.175$ except the lowest $\eta$ tower with $\approx \Delta\eta=0.1$ and highest $\eta$ tower with $\approx \Delta\eta=0.3$. The segmentation in $\phi$ is of $\Delta\phi=10º$ except in the highest $\eta$ towers which is $\Delta\phi=20º$. There are a total of tower 900 per \gls{HF} module. 

% The HCAL is designed to provide good energy resolution for the measurement of hadronic jets, with single charged pion resolution measured in a test beam \cite{Abdullin:2009zz} and found to be approximately
% \begin{equation}
% \frac{\sigma}{E} = \frac{94.3\%}{\sqrt{E}} \oplus 8.4\%.
% \end{equation}
% The hermetic design and shower containment of the HCAL is driven by the need for accurate measurement of the transverse energy balance in an event and to ensure unambiguous identification of muons by minimising hadronic punch-through into the muon chambers.


%%%%%%%%%%%%%%%%%%%%%%%%%%%%%%%%%%%%%%%%%%%%%%%%%%%%%%%%%%%%%%%%%%%%%%%%%%%%%%%%%%%%%%%
%%% SUBSECTION
%%%%%%%%%%%%%%%%%%%%%%%%%%%%%%%%%%%%%%%%%%%%%%%%%%%%%%%%%%%%%%%%%%%%%%%%%%%%%%%%%%%%
\subsection{Solenoid Magnet}
\label{SUBSECTION:ExperimentalApparatus_CMS_Magnet}

%Status: DONE

The design requirements for correct charge assignment and \pt determination for charge particles and specially muons imply drive the magnet parameters choice. For muons, unambiguously charge determination requires momentum resolution of $\Delta p/p \approx 10\%$ at $p = 1 \TeV$. This requirements are specially difficult to obtain in the forward regions but with the correct length/radius ratio can be obtained with a modestly sized solenoid magnet but with large field\cite{CMSTDR:CMSUpgradeTDR}.

The choice of the \gls{CMS} collaboration was to build a Niobium-Titanium (NbTi) superconducting solenoid magnet which has been design to operate at fields up to $4\,\tesla$ it has a diameter of $6\,\meter$ and a length of $12.5\,\meter$ at maximum field the stored energy reaches $2.7\,\giga\joule$. Typically, the magnet is only run at $3.8\,\tesla$ in order to maximize its lifetime. To contain such an enormous magnetic flux $10\,\kilo\ton$ return yoke envelopes the magnet with 5 wheels in the barrel region and 2 endcaps composed of 3 disks closing the sides\cite{ARTICLE:TheCMSExperiment}. A summary of the most important magnet parameters can be found at table \ref{TABLE:ExperimentalApparatus_CMSMagnetParameters}.

\begin{table}[!htb]
  \centering
  \begin{tabular}{|l|c|}
  \hline
  Parameter & Value \\
  \hline\hline
  Field           & 4 T \\
  Inner Bore      & 5.9 m \\
  Length          & 12.9 m \\
  Number of turns & 2168 \\
  Current         & 19.5 kA \\
  Stored Energy   & 2.7 GJ \\
  Hoop Stress     & 64 atm \\
  \hline
  \end{tabular}
  \caption[Parameters of the CMS superconducting solenoid]{Parameters of the CMS superconducting solenoid}
  \label{TABLE:ExperimentalApparatus_CMSMagnetParameters}
\end{table}


%%%%%%%%%%%%%%%%%%%%%%%%%%%%%%%%%%%%%%%%%%%%%%%%%%%%%%%%%%%%%%%%%%%%%%%%%%%%%%%%%%%%%%%
%%% SUBSECTION
%%%%%%%%%%%%%%%%%%%%%%%%%%%%%%%%%%%%%%%%%%%%%%%%%%%%%%%%%%%%%%%%%%%%%%%%%%%%%%%%%%%%
\subsection{Muon System}
\label{SUBSECTION:ExperimentalApparatus_CMS_Moun}

%Status: DONE (needs review)

The muon detection is an important part of the mission of \gls{CMS} as the middle name of the experiment indicates. Muons are fairly easy to detect when compared with other elementary particles and are only rarely produced in proton-proton collisions. Lets take the example of the \gls{SM} Higgs boson, while the decay mode involving a pair of Z bosons is fairly unlikely compared with other decays the Z bosons can decay into 4 mouns. This decay while rare does not have significant backgrounds making it a ''golden channel`` for discovery, which indeed was proven the case. Many other models, like SUSY, use muon final states in their searches exactly for the same reason. The \gls{CMS} muon system is composed of 3 types of gaseous detectors depending on they location and momentum reconstruction needs. A diagram of the disposition of this system inside \gls{CMS} can be found on figure \ref{FIGURE:ExperimentalApparatus_CMS_Muon_Layout}.

\begin{figure}[!htb]
  \centering
  \includegraphics{Chapter02/CMS/Images/CMS_Muon_Layout.png}
  \caption{Diagram of the \gls{CMS} muon systems. The location of each muon chamber for each subsystem is showed.}
  \label{FIGURE:ExperimentalApparatus_CMS_Muon_Layout}
\end{figure}

In the barrel and up to $|\eta|<1.2$, \gls{DT} are used. since the neutron background is small and the field is constant. This system is composed 250 chambers and is arranged in 4 concentric cylindrical layers which are installed inside of the return yoke. This chambers have a total of 172000 wires with a length of $2.4\,\meter$ which are housed inside of tubes filled with a mixture of argon and carbon-dioxide. Each of the wheels of the barrel is split into 12 sectors covering an angle of 30º azimuthal angle. The maximum gas ionization drift is of $2.0 \centi\meter$ and results in a single point resolution is $\approx 200\,\micro\meter$ per wire. For each station for each measured muon the $\phi$ resolution is better than $200\,\micro\meter$ and direction resolution is $\approx 1\,\milli\radian$.

In the endcaps the region between $2.4>|\eta|>0.9$, \gls{CSC} are used. In the region the muon and background rates are high and the magnetic field is not uniform. This systems have fast response and are radiation resistant and is composed by 468 chambers arranged in 4 stations per side. Each chamber is trapezoidal in shape and made of 6 gas gaps and covers either 10º or 20º in $\phi$. Each gap contains a plane of cathode strips and a plane of anode wires. For each chamber the spacial resolution is of the order of $200\,\micro\meter$ and the angular resolution is $\approx 10\,\milli\radian$ in $\phi$.

Finally the \gls{RPC} cover the $|\eta|<1.6$ range. This system overlaps with the 2 other muon systems. This system is very fast with an ionization event being much faster than the bunch crossing time. This fast response allows in conjunction with a dedicated trigger system to select the correct bunch crossing associated with the detection of a muon. In the barrel there 480 rectangular chambers arranged in 4 stations with 6 \gls{RPC} layers (2 layers are present in the 2 stations closeste to the beam pipe). In the endcaps there are 3 \gls{RPC} disk shaped stations on each side which are composed by trapezoidal shaped detectors.

The combined muon system offline momentum resolution is of the order of 9\% for small values of $\eta$ and $p$ for transverse momenta of up to $200\,\GeV$. At higher energies of around $1\,\TeV$ the standalone momentum resolution is in the range of 15-40\% depending on $|\eta|$. This values are limited by the muon multiple-scattering before arriving to the muon system. If we combine the tracker information into a global fit the resolution for lower \pt tracks improves an order of magnitude while higher momenta (around $1\,\TeV$) it is of about 5\% which is well inside the \gls{CMS} design requirements.

%%%%%%%%%%%%%%%%%%%%%%%%%%%%%%%%%%%%%%%%%%%%%%%%%%%%%%%%%%%%%%%%%%%%%%%%%%%%%%%%%%%%%%%
%%% SUBSECTION
%%%%%%%%%%%%%%%%%%%%%%%%%%%%%%%%%%%%%%%%%%%%%%%%%%%%%%%%%%%%%%%%%%%%%%%%%%%%%%%%%%%%
\subsection{Data Acquisition System}
\label{SUBSECTION:ExperimentalApparatus_CMS_DAQ}

%Status: DONE (needs review)

The \gls{CMS} \gls{DAQ} system is designed to process, analyses and ultimately store the information collected by the detector. The \gls{LHC} produces bunch crossings at a rate of $40\,\mega\hertz$ but we are only capable of storing between $10^2-10^3$ events per second. At design luminosity each collision will have an average of 20 simultaneous collisions. Each collision will produce a zero-suppressed data payload of around $1\,\mega\byte$. To reduce the initial incoming rate a first level of trigger was designed in order to reduce the amount of events to be processed to a maximum of $100 \kilo\hertz$. Even with this event suppression the \gls{DAQ} will have to handle a $\approx 100 \giga\byte\second^{-1}$ which come from approximately 650 data sources. The information is collected and passed to a computer farm where a software filter is installed which serves as a second level of trigger, and is known as the \gls{HLT}. In this system the event rate is further reduced by a factor of 1000 and the resulting events are saved into permanent storage. A diagram of this system can be found on figure TODO:XXX.

%TODO: Image of the DAQ arragement

%%%%%%%%%%%%%%%%%%%%%%%%%%%%%%%%%%%%%%%%%%%%%%%%%%%%%%%%%%%%%%%%%%%%%%%%%%%%%%%%%%%%%%%
%%% SUBSECTION
%%%%%%%%%%%%%%%%%%%%%%%%%%%%%%%%%%%%%%%%%%%%%%%%%%%%%%%%%%%%%%%%%%%%%%%%%%%%%%%%%%%%
\subsection{Trigger System}
\label{SUBSECTION:ExperimentalApparatus_CMS_Trigger}

The \gls{L1T} and \glsreset{HLT}

CMS Tridas TDR\cite{CMSTDR:CMSTridasTDR} 

\begin{figure}[!htb]
  \centering
  \includegraphics{Chapter02/CMS/Images/CMS_L1T_Layout.png}
  \caption{TODO}
  \label{FIGURE:ExperimentalApparatus_CMS_L1T_Layout}
\end{figure}

%%%%%%%%%%%%%%%%%%%%%%%%%%%%%%%%%%%%%%%%%%%%%%%%%%%%%%%%%%%%%%%%%%%%%%%%%%%%%%%%%%%%%%%
%%% SUBSECTION
%%%%%%%%%%%%%%%%%%%%%%%%%%%%%%%%%%%%%%%%%%%%%%%%%%%%%%%%%%%%%%%%%%%%%%%%%%%%%%%%%%%%
\subsection{Computing}
\label{SUBSECTION:ExperimentalApparatus_CMS_Computing}

%Status: DONE

The quantity of data produced by \gls{LHC} and the processing necessary are so big that it would be difficult to have all computing resources in a single place. For this reason a tiered system was developed all computing sites are connected and have specific roles and responsibilities in the data taking and processing.

The \gls{CERN} Data Centre is the tier 0 of this network, know as the Grid, All data produced by the \gls{LHC} passes through it. Only about 20\% of the total capacity of the Grid is hosted here, but \gls{CERN} had the very important mission of safe keeping all the raw data produced by the \gls{LHC} experiments. It also has the task of doing the first attempt at reconstructing this data into meaningful physics objects. 

There are 13 tier 1 computer centres around the world. They are responsible to store a proportional amount of raw and reconstructed data among them. If any reprocessing of the data is needed, this centres are responsible for this task and storing the resulting output as well. Tier 1 centres also host simulated data and distribute their data to affiliated tier 2 centres. 

Local research centres like universities of scientific laboratories are normally at the tier 2 level. They should have enough computing power and storage space to the analysis those centres are involved. This centres will have the responsibility of handling a proportional share of simulated data production and reconstruction. Currently there are over 150 tier 2 centres around the world

Individual computers or local clusters without any formal engagement with the Grid structure, are at the so called tier 3 level.

% The \gls{DQM} 

%%%%%%%%%%%%%%%%%%%%%%%%%%%%%%%%%%%%%%%%%%%%%%%%%%%%%%%%%%%%%%%%%%%%%%%%%%%%%%%%%%%%%%%
%%% SUBSECTION
%%%%%%%%%%%%%%%%%%%%%%%%%%%%%%%%%%%%%%%%%%%%%%%%%%%%%%%%%%%%%%%%%%%%%%%%%%%%%%%%%%%%
\subsection{Run II Upgrades}
\label{SUBSECTION:ExperimentalApparatus_CMS_RUNII}

%Status: Writing

An extensive upgrade program for the \gls{L1T} electronics was planed and is being executed in order to cope with the increase of luminosity and pile-up predicted for the period after \gls{LS1}\cite{CMSTDR:CMSL1Upgrade}. It is expected that center-of-mass energy will almost double from $8\,\TeV$ to $13\,\TeV$, instantaneous luminosity will also increase as will average pile-up. With the change from bunch separation from $50\,\nano\second$ to $25\,\nano\second$ out-of-time pile-up will also become a significant problem. 

To ensure physics performance during 2015 only a partial upgrade is planned for the 2015 run, which is known at the \textit{Stage-1} upgrade system. 

         %Status: DONE
   \chapter{Event Reconstruction and Physics Objects}
\label{CHAPTER:EventReconstructionPhysicsObjects}

\glsresetall % Resetting all acronyms

This chapter describes how the \gls{CMS} detector produces physics objects from the information collected at each event. The \glsreset{VBF} Higgs to invisible analysis uses almost all the physics objects produced by the detector and for this uses information from all the experiment sub-detectors. The following sections detail for each of these objects how they are reconstructed and what are the choices made to filter them.

%%%%%%%%%%%%%%%%%%%%%%%%%%%%%%%%%%%%%%%%%%%%%%%%%%%%%%%%%%%%%%%%%%%%%%%%%%%%%%%%%%%%%%%
%%% SECTION
%%%%%%%%%%%%%%%%%%%%%%%%%%%%%%%%%%%%%%%%%%%%%%%%%%%%%%%%%%%%%%%%%%%%%%%%%%%%%%%%%%%%%%%
\section{Tracks}

%Status: DONE

Reconstructing the trajectories of charged particles allows us to measure their momentum and determining their charge. This is possible by analysing the hit patterns in the inner tracking system. In \gls{CMS} this reconstruction is made with the \gls{CTF} algorithm \cite{ARTICLE:CMSTrackReconstruction}. The relevant steps for track generation are described below:

\begin{itemize}
  \item Seed generation is made with hits at the pixel detector. A track seeds can be made with two or three hits. In the first case a know vertex or the beam spot is used to constrain the seed momentum. The parameters of each seed are estimated using the assumption that the trajectory is a helix, but it takes into account hit errors and multiple scattering \cite{ARTICLE:CMSTrackReconstructionSeedGeneration}.
  \item The track seed is extrapolated through the tracker layers with on a combinatorial Kalman filter \cite{ARTICLE:KalmanFilteringTrackVertexFitting} . For each additional layer, the best matching hit if any is added and track parameters are recomputed. This procedure continues until the last layer is reached \cite{ARTICLE:CMSTrackReconstruction}.
  \item Ambiguity resolution may be necessary since since it is possible to have the same track being reconstructed from different seeds, or a seed may results in more than a single trajectory candidate. The resolve this possible double counting, when considering a pair of tracks with more than 50\% of shared hits, we discard the one with less hits. In case so equal number of hits the one with lowest $\chi^2$ is kept. 
  \item After the track building and cleaning stage is done final refitting is performed. This procedure is aimed at removing possible bias by constrains at the seed forming stage. A standard Kalman filter and smoother are used.
\end{itemize}

The process of track finding is repeated up to six times where the hits for each successfully reconstructed track are removed for the next iteration. Using early \gls{LHC} data and a dataset of pions and muons it was possible to estimate that the tracking efficiency is $>98\%$ for all track $\pt > 500\,\MeV$ and $>99\%$ for tracks with $\pt > 2,\GeV$ \cite{ARTICLE:CMSMeasurmentTrackEfficiency}.

%%%%%%%%%%%%%%%%%%%%%%%%%%%%%%%%%%%%%%%%%%%%%%%%%%%%%%%%%%%%%%%%%%%%%%%%%%%%%%%%%%%%%%%
%%% SECTION
%%%%%%%%%%%%%%%%%%%%%%%%%%%%%%%%%%%%%%%%%%%%%%%%%%%%%%%%%%%%%%%%%%%%%%%%%%%%%%%%%%%%%%%
\section{Vertex Reconstruction}

%Status: DONE

The \gls{LHC} can produce extreme collision intensities which are obtained partially by having multiple collisions happening at each bunch crossing. As it has been discussed in section \ref{SUBSECTION:ExperimentalApparatus_CMS_RunningAndPerformance} an average of 21 simultaneous collisions happened per bunch crossing in the \gls{CMS} experiment during 2012. In this environment, it is crucial to identify the \gls{PV} and the particles that come from it. This information can then be used to reject particles coming from other additional collisions and to identify displaced vertices which can be the signature of long lived particles like b-mesons.

The individual tracks are reconstructed making use of the inner tracker. Each vertex is initially seeded by two tracks with separation in \textit{z} less than $1\,\centi\meter$. Then remaining track are clustered to seed vertex with the \gls{DA} algorithm \cite{ARTICLE:DeterministicAnnealing}. After the clustering process is done, the position of each vertex is recomputed using the adaptive vertex fitter algorithm \cite{ARTICLE:AdaptiveVertexFitting}. In this algorithm weights, $w_{i}$ are assigned to each track according to how compatible they are with the fitted vertex position. Weight vary from 1 to 0, being that track assigned weights of close 1 are highly compatible with the vertex and close 0 would be given to low compatibility tracks. Then we can define the number of degrees of freedom of the new fit as:

\begin{equation}
n_{dof}(vertex)=2\sum\limits_{i}^{tracks} w_i - 3
\end{equation}

This variable can be used to distinguish real proton-proton interactions from misclustered vertices, since it is correlated with the number of tracks compatible with that specific vertex \cite{ARTICLE:CMSTrackingAndPrimaryVertex}. The vertex position and resolution have been measured with \gls{LHC} data and compared with simulation. The resulting plots can be found in figure \ref{FIGURE:EventReconstructionPhysicsObjects_Vertex} as a function of number of tracks.

\begin{figure}[htp]%
\centering
\subfloat[][]{\includegraphics[width=0.45\linewidth]{Chapter04/Vertex/Images/vtx-res.pdf}}\qquad
\subfloat[][]{\includegraphics[width=0.45\linewidth]{Chapter04/Vertex/Images/vtx-eff.pdf}}\\
\caption[Primary vertex resolution in the $z$ coordinate and vertex reconstruction efficiency as a function of the number of constituent tracks.]{(a) Primary vertex resolution in the $z$ coordinate a function of the number of associated tracks. Results are give for three ranges of average track \pt. (b) Primary vertex efficiency as a function of the number of associated track \cite{ARTICLE:CMSTrackingAndPrimaryVertex}}
\label{FIGURE:EventReconstructionPhysicsObjects_Vertex}
\end{figure}

The \gls{PV} is defined as the vertex with highest sum of associated tracks \pt squared. In situations were no vertex can be reconstructed, like if there is a tracking failure, the beam spot position is assumed. Knowing precisely the interaction point allows to determine particle candidate quantities relative to it which allow for better object identification and pile-up control. 

Most \gls{CMS} analysis, including the ones presented in this thesis, require explicitly that a good vertex is reconstructed with the following characteristics:

\begin{itemize}
  \item We require a real reconstructed vertex from tracks, not the beam spot.
  \item A minimum number of degrees of freedom: $n_{dof}>4$.
  \item Collision must be near the interaction point. We require longitudinal distance to be $|z|<=24\,\centi\meter$.
  \item We required the collision be be close to the beam line. Radial distance to beam line: $d_{xy}<2\,\centi\meter$. 
\end{itemize}

%%%%%%%%%%%%%%%%%%%%%%%%%%%%%%%%%%%%%%%%%%%%%%%%%%%%%%%%%%%%%%%%%%%%%%%%%%%%%%%%%%%%%%%
%%% SECTION
%%%%%%%%%%%%%%%%%%%%%%%%%%%%%%%%%%%%%%%%%%%%%%%%%%%%%%%%%%%%%%%%%%%%%%%%%%%%%%%%%%%%%%%
\section{Electron}

%Status: DONE

In the \gls{CMS} experiment electrons are reconstructed by matching energy clusters in the \gls{ECAL} with tracks coming from the inner tracking system. Unfortunately, electrons can loose and disperse significant amounts of energy until they reach the \gls{ECAL}. While they transverse the inner tracker they may emit photons through bremsstrahlung and in turn this photon can convert to $e^+e^-$ pairs. About 35\% of the electron radiate at least 70\% of their energy in this way \cite{ARTICLE:CMSElectronReconstruction}. This spread of energy is mostly in $\phi$ due to the applied magnetic field \cite{ARTICLE:CMSElectronReconstructionECAL}. Dedicated algorithms were developed to combine the the \gls{ECAL} energy deposits, into a so called \textit{super-clustering algorithm}, of the initial electron and its emissions.

Different algorithms are used in the barrel and endcaps regions. In the barrel region we explore the simple $\eta-\phi$ geometry with the ``hybrid clustering algorithm''. The procedure started by identifying \textit{seed crystals} with $E_\perp>1\,\GeV$. We form a domino around this seed in the $\eta$ direction of $3 \times 1$ or $5 \times 1$ crystals centred at the seed. Additional dominoes are added in both $\phi$ direction in an attempt to collect the bremsstrahlung emissions up to $\Delta\phi \approx 0.3\,\radian$. Any domino with energy below $100\,\MeV$ is disregarded. The resulting additional sub-clusters must have its own seed with $E_\perp>350\,\MeV$ and they are all combined to form the final \textit{supercluster}. 

In the encaps the ``$\text{Multi-}5 \times 5$'' is used. In the region of the detector the geometry is more complex and does not follow a simple $\eta-\phi$ symmetry. We start by selecting for seeds the crystals which are local maxima over their four direct neighbours and have a deposit of $E_\perp>0.18\,\GeV$. Then, and starting with the seeds with highest $E_\perp$, we collect the energy around them into clusters of $5 \times 5$ crystals. We then search for similar seeds and form clusters that can overlap within $\Delta\eta<0.07$ and $\Delta\phi<0.3\,\radian$ of the initial seed. Those clusters are then combined into a single \textit{supercluster} which needs to have at least $E_\perp>1\,\GeV$. The \textit{supercluster} is then extrapolated to the \gls{ECAL} preshower
by clustering the energy within $\Delta\eta<0.15$ and $\Delta\phi<0.45$ around the most energetic cluster and adding it to the \textit{supercluster} itself \cite{ARTICLE:CMSElectronReconstruction8TeV}.

In order to reconstruct the electron track we need to take into account the bremsstrahlung emissions. The \gls{CTF} algorithm is not appropriate for this purpose so a different track-finding algorithm had to be developed. For high \pt electrons we use the \gls{ECAL} supercluster energy deposit weighted mean impact point as a seed. If we combine this information with the determined $E_\perp$ we can define two $\eta-\phi$ search regions in the pixel detector depending on the charge hypothesis. If we find two compatible hits, the electron trajectory is updated. From this point normal track building is performed but instead of a Kalman filter algorithm we use a \gls{GSF} algorithm \cite{ARTICLE:CMSReconstructionElectronsGSF}. This method performs better in the presence of non-Gaussian losses like the one coming from the bremsstrahlung emissions.

The typical background to real electrons are collimated hadronic jets, like from $\pi^0$ and $\pi^{\pm}$ overlap or from $\pi^{\pm}$ showers \cite{ARTICLE:CMSElectronReconstruction}. There are many useful variables that may be used to reduce such background and are often used in \textit{electron identification} criteria:

\begin{itemize}
  \item $\Delta\eta_{in}$ and $\Delta\phi_{in}$, are the distance between the track direction at the vertex and extrapolated to the \gls{ECAL} and supercluster.
  \item $\sigma_{i \eta i \eta}$ is the energy-weighted $\eta$ width of the cluster. For real prompt electrons this is normally small since this quantity is not significantly affected by the magnetic field.
  \item $H/E$ is the ration of hadronic to electromagnetic energy in the region of the seed cluster. 
\end{itemize}

Distributions of the variables for simulated electrons and jets can be found in figure \ref{FIGURE:EventReconstructionPhysicsObjects_Electrons}. 

\begin{figure}[htp]%
\centering
\subfloat[]{\includegraphics[width=0.45\linewidth]{Chapter04/Electrons/Images/elecs_deta.pdf}}\qquad
\subfloat[]{\includegraphics[width=0.45\linewidth]{Chapter04/Electrons/Images/elecs_dphi.pdf}}\\
\subfloat[]{\includegraphics[width=0.45\linewidth]{Chapter04/Electrons/Images/elecs_hovere.pdf}}\qquad
\subfloat[]{\includegraphics[width=0.45\linewidth]{Chapter04/Electrons/Images/elecs_sigmaietaieta.pdf}}
\caption[Distributions for the variables $\Delta\eta_{\text in}$, $\Delta\phi_{\text in}$, $\sigma_{i\eta i\eta}$ and $H/E$ for simulated electrons and misidentified jets.]{Distributions for (a) $\Delta\eta_{in}$, (b) $\Delta\phi_{in}$, (c) $H/E$ and (d) $\sigma_{i \eta i \eta}$. Here \textit{golden electrons} are those who emit minimal bremmstrahlung photons, \textit{showering} are electrons that lose a large faction of their energy in emissions and \textit{jets} are the typical distributions for hadronic jets.}
\label{FIGURE:EventReconstructionPhysicsObjects_Electrons}
\end{figure}

%%%%%%%%%%%%%%%%%%%%%%%%%%%%%%%%%%%%%%%%%%%%%%%%%%%%%%%%%%%%%%%%%%%%%%%%%%%%%%%%%%%%%%%
%%% SECTION
%%%%%%%%%%%%%%%%%%%%%%%%%%%%%%%%%%%%%%%%%%%%%%%%%%%%%%%%%%%%%%%%%%%%%%%%%%%%%%%%%%%%%%%
\section{Muon}

%Status: Writing

Muon track reconstruction starts independently at the inner-tracker (\textit{tracker track} and in the muon systems (\textit{standalone-muon track}) \cite{ARTICLE:CMSMuonReconstruction7TeV}. Then this information can be combined into a single muon track in two possible ways.

\textit{Global Muon reconstruction} is an \textit{outside-in algorithm}. We starts by finding tracker track match for each stand-alone muon track. This is done by propagating the match candidate pair to a common surface and comparing track parameters. For each matched pair, a \textit{global-muon fit} is performed using all hits from the two tracks using a Kalman-filter algorithm \cite{ARTICLE:KalmanFilteringTrackVertexFitting}. For muons of $\pt\gtrsim 200\,\GeV/c$, it has been showed that a \textit{global-muon fit} improves the momentum resolution compared to a \textit{tracker-only fit} \cite{CMSTDR:CMSPhysicsVol1, ARTICLE:CMSPerformanceMuonReconstructionCosmicRay}.

\textit{Tracker Muon reconstruction} is an \textit{inside-out algorithm}. In this method we start by selecting all tracker tracks with $\pt>0.5\,\GeV$ and $p>2.5\,\GeV$. We extrapolate those tracks to the muon system while taking into account the magnetic field, energy loss and scattering. If we find a match with at least one muon segment in the muon system (track stub in the \gls{DT} or \gls{CSC}) this this tracker track now becomes a Tracker Muon. 

Tracker muon reconstructions is more efficient than the global muon reconstruction at low momenta at $p\lesssim 5\,\GeV$. This difference is due to tracker muons reconstruction only requiring one segment on the muon system. While global muon reconstruction is more efficient for higher energies where the muons are more likely to pass several muon stations.

%TODO: Needs more stuff here

%\cite{ARTICLE:CMSMuonReconstruction7TeV} %AG 72

%%%%%%%%%%%%%%%%%%%%%%%%%%%%%%%%%%%%%%%%%%%%%%%%%%%%%%%%%%%%%%%%%%%%%%%%%%%%%%%%%%%%%%%
%%% SECTION
%%%%%%%%%%%%%%%%%%%%%%%%%%%%%%%%%%%%%%%%%%%%%%%%%%%%%%%%%%%%%%%%%%%%%%%%%%%%%%%%%%%%%%%
\section{Particle Flow}

%Status: Writing

The \gls{PF} algorithm \cite{ARTICLE:CMSComissioningOfParticleFlow, ARTICLE:CMSParticleFlowEventRecontruction, ARTICLE:CMSComissioningOfParticleFlowWithMinBias} is used in the \gls{CMS} experiment as a way to combine all available information in an attempt to reconstruct every particle produced in the event. 



% \cite{ARTICLE:CMSComissioningOfParticleFlow}            AG 73
% \cite{ARTICLE:CMSParticleFlowEventRecontruction}        AG 74
% \cite{ARTICLE:CMSComissioningOfParticleFlowWithMinBias} AG 75


%%%%%%%%%%%%%%%%%%%%%%%%%%%%%%%%%%%%%%%%%%%%%%%%%%%%%%%%%%%%%%%%%%%%%%%%%%%%%%%%%%%%%%%
%%% SECTION
%%%%%%%%%%%%%%%%%%%%%%%%%%%%%%%%%%%%%%%%%%%%%%%%%%%%%%%%%%%%%%%%%%%%%%%%%%%%%%%%%%%%%%%
\section{Jets}

%Status: Writing

\subsection{Jet Clustering}

\cite{ARTICLE:TowardsJetography}

\cite{ARTICLE:AntiKtAlgorithm}

\cite{ARTICLE:FastJetUserManual}

\subsection{Jet Energy Corrections}

\cite{ARTICLE:CMSDeterminationJetEnergyCalibration}

\cite{ARTICLE:PileupSubtractionJetAreas}

%%%%%%%%%%%%%%%%%%%%%%%%%%%%%%%%%%%%%%%%%%%%%%%%%%%%%%%%%%%%%%%%%%%%%%%%%%%%%%%%%%%%%%%
%%% SECTION
%%%%%%%%%%%%%%%%%%%%%%%%%%%%%%%%%%%%%%%%%%%%%%%%%%%%%%%%%%%%%%%%%%%%%%%%%%%%%%%%%%%%%%%
\section{Taus}

%%%%%%%%%%%%%%%%%%%%%%%%%%%%%%%%%%%%%%%%%%%%%%%%%%%%%%%%%%%%%%%%%%%%%%%%%%%%%%%%%%%%%%%
%%% SECTION
%%%%%%%%%%%%%%%%%%%%%%%%%%%%%%%%%%%%%%%%%%%%%%%%%%%%%%%%%%%%%%%%%%%%%%%%%%%%%%%%%%%%%%%
\section{Missing Transverse Energy}

\cite{ARTICLE:CMSMissingTransverseEnergyPerformance}

\cite{ARTICLE:CMSMETPerformance8TeV}

                %Status: DONE
   \chapter{Technical work}

\section{Level 1 Trigger Data Quality Monitoring System}

\subsection{Online Monitoring}

\subsection{Offline Monitoring}

\subsection{Release Validation}

\section{Implemented Tests}

\subsection{Rates Monitoring}

\subsection{Synchronization Monitoring}

\subsection{Occupancy Monitoring}

\subsection{Status Summary Display}

\section{Certification}

\section{Proposed Future Upgrades}
                 %Status: DONE
   \chapter{Prompt Data Analysis}

\section{Data and Monte Carlo samples}

\section{Data and Monte Carlo correction factors}

\subsection{}
            %Status: DONE
   \chapter{Preparation for H(Inv) decays in the VBF channel with CMS parked data}
\label{CHAPTER:PreparationParkedDataAnalysis}

%Status: DONE (reviewed J.Pela x1)

The Run I \gls{CMS} \gls{VBF} Higgs to invisible analysis was performed over two overlapping datasets. The promptly reconstructed data became available almost immediately after recording and its analysis was already presented in chapter \ref{CHAPTER:PromptDataAnalysis}, and became known as the Run I \textit{prompt analysis}. Simultaneously, a second data stream was recorded with lower trigger thresholds which only became available for analysis after the full \gls{LHC} Run I was finished. This chapter describes the studies made to prepare the Run I \textit{parked analysis} over this additional dataset. 

A description of the studies performed to develop the trigger conditions used to record data for the \gls{VBF} Higgs to invisible analysis can be found in section \ref{SECTION:ParkedDataAnalysis_ParkedTriggerDevelopment}. This study was extend to create a condition to select \gls{VBF} Higgs events independently of the final state.

The lower trigger thresholds required a re-optimization of the analysis event selection. Due to the lack of enough simulated event statistics the \textit{prompt analysis} was tuned to suppress the selection of \gls{QCD} multi-jet processes to a negligible level. The ability to lower the selection thresholds to take advantage of the lower parked data trigger requirements was therefore  limited by our understanding of the \gls{QCD} multi-jet background. Having a \gls{MC} description of these type of events would allow easier threshold optimization and create the opportunity for the analysis to evolve to a shape based or \gls{MVA} based analysis.

The production and characterization of \gls{QCD} multi-jet samples with \gls{VBF} characteristics and real \gls{MET} is described in section \ref{SECTION:ParkedDataAnalysis_QCDVBFMET}. Further studies of possible approaches to suppress the \gls{QCD} multi-jet background are presented in sections \ref{SECTION:PreparationParkedDataAnalysis_DijetMETSystemVars} and \ref{SECTION:PreparationParkedDataAnalysis_TrackDistributionVariables}.

%%%%%%%%%%%%%%%%%%%%%%%%%%%%%%%%%%%%%%%%%%%%%%%%%%%%%%%%%%%%%%%%%%%%%%%%%%%%%%%%%%%%
%%% SECTION
%%%%%%%%%%%%%%%%%%%%%%%%%%%%%%%%%%%%%%%%%%%%%%%%%%%%%%%%%%%%%%%%%%%%%%%%%%%%%%%%%%%%
\section{L1T Parked Trigger Development}
\label{SECTION:ParkedDataAnalysis_ParkedTriggerDevelopment}

%Status: DONE (reviewed J.Pela x1)

The first step of any analysis is defining or selecting a trigger to collect data. This trigger should have a high signal efficiency while recording an acceptable rate.

At the beginning of 2012 the possibility of recording data without promptly reconstructing it was introduced. This additional data is now known as \textit{parked data}. The \gls{CMS} \gls{VBF} Higgs to invisible analysis saw this as an opportunity to develop a secondary set of triggers with lower thresholds to allow more signal to be collected when compared with the already developed prompt trigger. As this effort developed it became clear that an inclusive trigger that would record \gls{VBF} events regardless of final state could be implemented.

%%%%%%%%%%%%%%%%%%%%%%%%%%%%%%%%%%%%%%%%%%%%%%%%%%%%%%%%%%%%%%%%%%%%%%%%%%%%%%%%%%%%
%%% SUBSECTION
%%%%%%%%%%%%%%%%%%%%%%%%%%%%%%%%%%%%%%%%%%%%%%%%%%%%%%%%%%%%%%%%%%%%%%%%%%%%%%%%%%%%
\subsection{VBF Higgs to Invisible Higgs Level 1 Trigger development}
\label{SUBSECTION:ParkedDataAnalysis_ParkedTriggerDevelopment_VBFHiggsInvisibleTrigger}

%Status: DONE (reviewed J.Pela x1)

Data recorded during the special high \gls{PU} run in late 2011, was used to study \gls{L1T} trigger algorithms to be used during the 2012 proton run. During this \gls{LHC} fill the average \gls{PU} was of $\approx 30$ simultaneous interactions. 

The investigated algorithms select the typical topology of our signal. They look for events with \gls{MET} and and two jets located in opposite sides of the detector by requiring $\eta_{jet1}\times\eta_{jet2}<0$, large pseudo-rapidity separation of at least $\Delta\eta_{jj}>3$. The possibility of using $\Delta\phi_{jj}$ was also studied but was disfavoured since it could lead to lower signal efficiency in some \gls{BSM} models.

The conditions expected for early 2012 were of instantaneous luminosity of $5 \times 10^{33}\,\centi\meter^{-2}\second^{-1}$ and an average \gls{PU} of 28 interactions (scenario A). For late 2012 conditions were expected to increase to instantaneous luminosity of $7 \times 10^{33}\,\centi\meter^{-2}\second^{-1}$ and an average \gls{PU} of 32 interactions (scenario B).

Algorithm parameters were optimized for both this scenarios of \gls{LHC} running and considering several benchmark \gls{L1T} rates. The proposed target rate for the algorithm was assumed to be $2\,\kilo\Hz$, the additional working points were calculated with the intention to adjust the selection cuts according to higher or lower bandwidth available on the menu. The two key variables are $\pt^{\text{jets}}$ and the \gls{MET} and were optimized separately. Each of these variables was set in turn to the lowest reasonable value while the other was scanned until the necessary rate value was achieved.  Results for scenario A can be found in table \ref{TABLE:ParkedDataAnalysis_L1TParkedTriggerDevelopment_Rate5E33} and for scenario B in table \ref{TABLE:ParkedDataAnalysis_L1TParkedTriggerDevelopment_Rate7E33}.

\begin{table}[!htb]
\begin{minipage}{.5\linewidth}
  \centering
  \begin{tabular}{|c||c|c|c|c|}
  \hline
  \multicolumn{5}{|c|}{MET [GeV] ($\pt^{\text{jets}} > 20\,\GeV$)} \\
  \hline\hline
  $\Delta\phi$ & no cut & 2.5 & 2.1 & 1.8 \\
  \hline
   $10\,\kHz$  &     32 &  32 &  32 &  32 \\
    $5\,\kHz$  &     35 &  35 &  35 &  35 \\
  \hline\hline
    $2\,\kHz$  &     41 &  41 &  41 &  41 \\
  \hline\hline
    $1\,\kHz$  &     47 &  47 &  47 &  46 \\
  $0.5\,\kHz$  &     54 &  54 &  54 &  53 \\
  \hline
  \end{tabular}
\end{minipage}%
\begin{minipage}{.5\linewidth}
  \centering
  \begin{tabular}{|c||c|c|c|c|}
  \hline
  \multicolumn{5}{|c|}{$\pt^{\text{jets}}$ [GeV] ($\text{MET}>30\,\GeV$)} \\
  \hline
  $\Delta\phi$ & no cut & 2.5 & 2.1 & 1.8 \\
  \hline\hline
   $10\,\kHz$  &     28 &  28 &  24 &  24 \\
    $5\,\kHz$  &     32 &  32 &  32 &  32 \\
  \hline\hline
    $2\,\kHz$  &     52 &  48 &  44 &  44 \\
  \hline\hline
    $1\,\kHz$  &     68 &  68 &  64 &  64 \\
  $0.5\,\kHz$  &     92 &  92 &  88 &  88 \\
  \hline
  \end{tabular}
\end{minipage} 
\caption{Tables showing the \gls{L1T} rate for different selection criteria for $5 \times 10^{33}\,\centi\meter^{-2}\second^{-1}$ and an average \gls{PU} of 28 interactions (scenario A). In selected events the leading two jets is in opposite sides of the detector. On the left table the \gls{MET} cut is calculated while requiring the two leading jets to have $\pt^{\text{jets}} > 20\,\GeV$. Similarly, on the right table $\pt^{\text{jets}}$ cut is calculated while requiring $\text{MET}>30\,\GeV$.}
\label{TABLE:ParkedDataAnalysis_L1TParkedTriggerDevelopment_Rate5E33}
\end{table}

\begin{table}[!htb]
\begin{minipage}{.5\linewidth}
  \centering
  \begin{tabular}{|c||c|c|c|c|}
  \hline
  \multicolumn{5}{|c|}{MET [GeV] ($\pt^{\text{jets}}>20\,\GeV$)} \\
  \hline\hline
  $\Delta\phi$ & no cut & 2.5 & 2.1 & 1.8 \\
  \hline
   $10\,\kHz$  &     36 &  36 &  36 &  36 \\
    $5\,\kHz$  &     40 &  40 &  40 &  40 \\
  \hline\hline
    $2\,\kHz$  &     47 &  47 &  47 &  46 \\
  \hline\hline
    $1\,\kHz$  &     54 &  54 &  54 &  54 \\
  $0.5\,\kHz$  &     67 &  66 &  66 &  64 \\
  \hline
  \end{tabular}
\end{minipage}%
  \begin{minipage}{.5\linewidth}
    \centering
  \begin{tabular}{|c||c|c|c|c|}
  \hline
  \multicolumn{5}{|c|}{$\pt^{\text{jets}}$ [GeV] ($\text{MET}>30\,\GeV$)} \\
  \hline
  $\Delta\phi$ & no cut & 2.5 & 2.1 & 1.8 \\
  \hline\hline
   $10\,\kHz$  &     32 &  32 &  32 &  32 \\
    $5\,\kHz$  &     40 &  40 &  40 &  40 \\
  \hline\hline
    $2\,\kHz$  &     64 &  60 &  60 &  56 \\
  \hline\hline
    $1\,\kHz$  &     76 &  76 &  76 &  76 \\
  $0.5\,\kHz$  &    100 & 100 &  96 &  92 \\
  \hline
  \end{tabular}
\end{minipage} 
\caption{Tables showing the \gls{L1T} rate for different selection criteria for $7 \times 10^{33}\,\centi\meter^{-2}\second^{-1}$ and an average \gls{PU} of 32 interactions (scenario B). In selected events the leading two jets is in opposite sides of the detector. On the left table the \gls{MET} cut is calculated while requiring the two leading jets to have $\pt^{\text{jets}} > 20\,\GeV$. Similarly, on the right table $\pt^{\text{jets}}$ cut is calculated while requiring $\text{MET}>30\,\GeV$.}
\label{TABLE:ParkedDataAnalysis_L1TParkedTriggerDevelopment_Rate7E33}
\end{table}

These results were used to define working points for this trigger, which were proposed to the \glsreset{TSG}
to be included on \gls{L1T} Menus. Proposed trigger options were:
\begin{itemize}
\item Algorithm A: Lead dijet (opp. sides + $\pt^{\text{jets}}>20\,\GeV$ + $\Delta\eta_{jj}>3$) + $\text{MET}>40\,\GeV$
\item Algorithm B: Lead dijet (opp. sides + $\pt^{\text{jets}}>50\,\GeV$ + $\Delta\eta_{jj}>3$) + $\text{MET}>30\,\GeV$
\end{itemize}

It can be observed that with the predicted increase of instantaneous luminosity and \gls{PU} from scenario A to scenario B, the necessary rate for such proposed algorithms could escalate up to $\approx 5\,\kilo\Hz$. For the rate to be maintained at $2\,\kilo\Hz$ the value of the MET cut in algorithm A would have to be raised to $47\,\GeV$ and the value of $\pt^{\text{jets}}$ cut in algorithm B would have to increase to $64\,\GeV$.

%%%%%%%%%%%%%%%%%%%%%%%%%%%%%%%%%%%%%%%%%%%%%%%%%%%%%%%%%%%%%%%%%%%%%%%%%%%%%%%%%%%%
%%% SUBSECTION
%%%%%%%%%%%%%%%%%%%%%%%%%%%%%%%%%%%%%%%%%%%%%%%%%%%%%%%%%%%%%%%%%%%%%%%%%%%%%%%%%%%%
\subsection{VBF Higgs Inclusive Trigger}
\label{SUBSECTION:ParkedDataAnalysis_ParkedTriggerDevelopment_InclusiveHiggsTrigger}

%Status: DONE

It would be desirable to have a dedicated \gls{VBF} Higgs inclusive \gls{L1T} trigger that would be decay independent. Such an algorithm would allow analysts to have a single trigger for all \gls{VBF} produced Higgs decay signatures, which would imply less systematics in their comparison. Additionally, if an algorithm is used by more analyses it will become better understood.

When selecting events based only on the presence of a dijet with \gls{VBF} characteristics we can remove the dependency on the Higgs decay. This approach would be suitable since it ignores the Higgs decays themselves. Since we are not making any assumptions of the Higgs model, we could study all possible decays even those predicted by yet to be defined models with a single trigger. Thus, it would be a model-independent trigger.

This trigger can also be used for analysis of WW scattering, which in the case of the absence of the now discovered Higgs boson would allow to eventually exclude the standard model itself. Since without a Higgs boson this process probability would become larger than unity at the $\TeV$ scale.
%TODO: need reference here.

Such a trigger would have to select two forward jets with \gls{VBF} characteristics, and would be limited to what is possible to implement on the current \gls{L1T} hardware. The following variables were considered to suppress the algorithm rate by constrain the dijet system: dijet invariant mass, dijet transverse invariant mass ($M_{T}$), and event \gls{HT}. For this study we always require a \gls{L1T} dijet with both jets in opposite sides of the detector with $\Delta\eta>3$. The selected jets \pt is scanned and \gls{L1T} algorithm rate is calculated for a grid of points of each considered variable.

%%%%%%%%%%%%%%%%%%%%%%%%%%%%%%%%%%%%%%%%%%%%%%%%%%%%%%%%%%%%%%%%%%%%%%%%%%%%%%%%%%%%
%%% SUBSUBSECTION
%%%%%%%%%%%%%%%%%%%%%%%%%%%%%%%%%%%%%%%%%%%%%%%%%%%%%%%%%%%%%%%%%%%%%%%%%%%%%%%%%%%%
\subsubsection{Dijet invariant mass}

%Status: DONE

The for \gls{VBF} processes the outgoing dijet is expected to have high invariant mass making this quantity a possible handle at in a \gls{L1T} trigger algorithm. Calculation of dijet invariant mass was not implemented in the \gls{L1T} hardware but according to trigger experts it was in principle possible. Unfortunately, to obtain acceptable rates the required threshold for jet $p_T$ and $M_{jj}$ would reject almost all \gls{SM} \gls{VBF} Higgs signal.

%%%%%%%%%%%%%%%%%%%%%%%%%%%%%%%%%%%%%%%%%%%%%%%%%%%%%%%%%%%%%%%%%%%%%%%%%%%%%%%%%%%%
%%% SUBSUBSECTION
%%%%%%%%%%%%%%%%%%%%%%%%%%%%%%%%%%%%%%%%%%%%%%%%%%%%%%%%%%%%%%%%%%%%%%%%%%%%%%%%%%%%
\subsubsection{Dijet transverse invariant mass}

%Status: DONE

The dijet transverse mass was also considered and was proved to be more effective at suppressing \gls{QCD} multi-jet events. Again, this quantity is not implemented on the \gls{L1T} hardware but could be implemented. A possible working point for scenario A was obtained with a predicted \gls{L1T} rate of $5\kilo\hertz$. Events would be selected with at least one dijet with $\Delta\eta>3$ and $M_T>50\,\GeV$ where both jets have $p_T \sim 45\,\GeV$ and are in opposite sides of the detector. The expected signal efficiency was calculated to be $\approx 70\%$ for a \gls{SM} \gls{VBF} Higgs with $m_H=125\,\GeV$ decaying to $\tau\tau$.

%%%%%%%%%%%%%%%%%%%%%%%%%%%%%%%%%%%%%%%%%%%%%%%%%%%%%%%%%%%%%%%%%%%%%%%%%%%%%%%%%%%%
%%% SUBSUBSECTION
%%%%%%%%%%%%%%%%%%%%%%%%%%%%%%%%%%%%%%%%%%%%%%%%%%%%%%%%%%%%%%%%%%%%%%%%%%%%%%%%%%%%
\subsubsection{Event scaler sum of the transverse hadronic energy}

%Status: Writting

The event scaler sum of the transverse hadronic energy is the sum of the energy of all \gls{L1T} jets in the event. This variable has the advantage of being already implemented in the \gls{L1T} hardware. Again, a possible working point for scenario A was obtained with a predicted \gls{L1T} rate of $5\kilo\hertz$. Events would be selected with $HT>100\,\GeV$ and with at least one dijet with $\Delta\eta>3$ where both jets have $p_T \sim 40\,\GeV$ and are in opposite sides of the detector. The expected signal efficiency was calculated to be $\approx 98\%$ for a \gls{SM} \gls{VBF} Higgs with $m_H=125\,\GeV$ decaying to $\tau\tau$. A plot of the scan over \gls{L1T} $\pt^{jets}$ for $HT>100\,\GeV$ can be found in figure \ref{FIGURE:ParkedDataAnalysis_ParkedTriggerDevelopment_PU28_5e33_RateFBDijetDEtaDPhi00HT100}.

\begin{figure}[ht]
\centering
\includegraphics[width=0.60\textwidth]{Chapter07/ParkedDataTriggerDevelopment/Images/PU28_5e33_RateFBDijetDEtaDPhi00HT100.png}
\caption{Level 1 rate as a function of dijet $p_T^{jets}$ while selecting events with at least one dijet with $\Delta\eta>3$ where are in opposite sides of the detector and $HT>100\,\GeV$ for scenario A. Results based on data from the high pileup special run taken late 2011. The red lines indicate the selected working point of \gls{L1T} rate of $5\,\kilo\hertz$ resulting in a $p_T^{jets} \gtrsim 40\,\GeV$ threshold.}
\label{FIGURE:ParkedDataAnalysis_ParkedTriggerDevelopment_PU28_5e33_RateFBDijetDEtaDPhi00HT100}
\end{figure}

%%%%%%%%%%%%%%%%%%%%%%%%%%%%%%%%%%%%%%%%%%%%%%%%%%%%%%%%%%%%%%%%%%%%%%%%%%%%%%%%%%%%
%%% SUBSECTION
%%%%%%%%%%%%%%%%%%%%%%%%%%%%%%%%%%%%%%%%%%%%%%%%%%%%%%%%%%%%%%%%%%%%%%%%%%%%%%%%%%%%
\subsection{Final proposal}

%Status: DONE

The \gls{L1T} algorithm selecting \gls{MET} bigger than $40\,\GeV$ was unprescaled in the trigger menus used during Run I, it was decided to use this algorithm as seed for the invisible component of \gls{VBF} Higgs decays. Thus avoiding the additional complication of having to implement a dijet plus \gls{MET} \gls{L1T} algorithm for minimal threshold gain. For the \gls{VBF} channels with visible decays and considering the findings of this study, a similar simplified approach was taken. Depending on the delivered instantaneous luminosity and prescale events with \gls{HT} bigger than 150 or $170\,\GeV$ would be selected. The developed inclusive \gls{VBF} Higgs \gls{HLT} path was seeded by a logical or the these three \gls{L1T} seeds.

%%%%%%%%%%%%%%%%%%%%%%%%%%%%%%%%%%%%%%%%%%%%%%%%%%%%%%%%%%%%%%%%%%%%%%%%%%%%%%%%%%%%
%%% SECTION
%%%%%%%%%%%%%%%%%%%%%%%%%%%%%%%%%%%%%%%%%%%%%%%%%%%%%%%%%%%%%%%%%%%%%%%%%%%%%%%%%%%%
\section{QCD VBF+MET Monte Carlo simulation}
\label{SECTION:ParkedDataAnalysis_QCDVBFMET}

% %%%%%%%%%%%%%%%%%%%%%%%%%%%%%%%%%%%%%%%%%%%%%%%%%%%%%%%%%%
% QCD VFB + MET samples
% %%%%%%%%%%%%%%%%%%%%%%%%%%%%%%%%%%%%%%%%%%%%%%%%%%%%%%%%%%
% 20140204_ICVBFHiggs_JPela_VBFQCD.pdf (plot on the AN)
% 20140506_ICVBFHiggs_JPela.pdf (selection test)
% 
% Something odd here
% /home/hep/jca10/work/vbfinv/ws08/CMSSW_5_3_11/src/UserCode/ICHiggsTauTau/Analysis/HiggsNuNu


\begin{table}
\centering
\begin{tabular}{|c|r|c|r|c|c|}
  \hline
  Sample          &    Ev. Gen. & Filter Eff. &  Events & XS $[pb]$ & Eq. Lumi. $[fb^{-1}]$ \\
  \hline \hline
  QCD-Pt-80to120  & 39376000000 &    0.000049 & 1614416 &  1033680 &  38.09 \\
  QCD-Pt-120to170 &  7000000000 &    0.000283 & 2051000 & 156293.3 &  44.79 \\
  QCD-Pt-170to300 &  1375000000 &    0.000987 & 1391500 & 34138.15 &  40.28 \\
  QCD-Pt-300to470 &    80000000 &    0.002659 &  207840 & 1759.549 &  45.47 \\
  QCD-Pt-470to600 &    25000000 &    0.004127 &  104675 & 113.8791 & 219.53 \\
  \hline
\end{tabular}
\end{table}

\begin{figure}[!htb]
\centering
\includegraphics[width=0.55\textwidth]{Chapter05/ParkedDataPreparation/Images/Joao_140209_p11.png}
\caption{Reconstructed PFMET as a function of generator-level MET
      in the inclusive QCD samples $80 < \hat{p_T} < 600$ GeV before
      any selection.}
\label{fig:qcdRecovsGenMET}
\end{figure}

%%%%%%%%%%%%%%%%%%%%%%%%%%%%%%%%%%%%%%%%%%%%%%%%%%%%%%%%%%%%%%%%%%%%%%%%%%%%%%%%%%%%
%%% SECTION
%%%%%%%%%%%%%%%%%%%%%%%%%%%%%%%%%%%%%%%%%%%%%%%%%%%%%%%%%%%%%%%%%%%%%%%%%%%%%%%%%%%%
\section{Dijet-MET system topological variables}
\label{SECTION:PreparationParkedDataAnalysis_DijetMETSystemVars}
% 18 Months report was on the Sep 9 2013
% 
% /home/hep/jca10/work/vbfinv/ws04/CMSSW_5_3_11/src/UserCode/ICHiggsTauTau/Analysis/HiggsNuNu/PLOTS_mjj1200_dijetFrac/nunu/MET130/n_vtx_2012_DijetFraction_log.pdf
% Looks like plots are from Jun 25 2013
% 
% FOUND FILE: This is the file used in the plots for the 18 months report
% /home/hep/jca10/work/vbfinv/ws04/CMSSW_5_3_11/src/UserCode/ICHiggsTauTau/Analysis/HiggsNuNu/output_mjj1100_dijetFrac/nunu/MET130/MC_VBF_HToZZTo4Nu_M-120.root
% 
% /vols/cms02/jca10/work/ws01/CMSSW_5_3_11/src/UserCode/ICHiggsTauTau/Analysis/HiggsNuNu/plots/DPhiSIGNAL_CJVpass/dijetMet/dijetFrac_htMET.png
% Looks like plots are from Feb 20 2014




%%%%%%%%%%%%%%%%%%%%%%%%%%%%%%%%%%%%%%%%%%%%%%%%%%%%%%%%%%%%%%%%%%%%%%%%%%%%%%%%%%%%
%%% SECTION
%%%%%%%%%%%%%%%%%%%%%%%%%%%%%%%%%%%%%%%%%%%%%%%%%%%%%%%%%%%%%%%%%%%%%%%%%%%%%%%%%%%%
\section{Track distribution variables}
\label{SECTION:PreparationParkedDataAnalysis_TrackDistributionVariables}

% 20140603_ICVBFHiggs_JPela.pdf
% /afs/cern.ch/user/p/pela/work/cms/vbfinv/ws03/CMSSW_5_3_11/src/VBFHiggsToInvisible/VariableAnalyser/python/PowhegSignal_cfi.py
% Plots remade here:
% /afs/cern.ch/user/p/pela/work/cms/vbfinv/ws03/slc6/CMSSW_5_3_11/src/VBFHiggsToInvisible/VariableAnalyser/results/

One of the important features of \gls{VBF} processes is that there is no colour connection between the outgoing quark jets. Since \gls{QCD} multi-jet processes do have colour connection, it is expected that \gls{QCD} background events will have hadronic activity between the leading jets. Motivating the use of a central jet veto in the Run I prompt analysis. Events were vetoed if a jet identified as coming from \gls{PV} with enough \pt was found between the two leading jets. But colour connection can result simply in a spread of energy in the event which may be clustered into \gls{PU} jets or remain unclustered. Also by having a minimum \pt requirement on a single veto jet may result in accepting events where multiple jets coming from the \gls{PV} are present with low \pt, but if combined would pass the veto threshold.

For colour connected jets it is expected that a significant amount tracks coming from the primary vertex would be spread in the event specially between the selected jets. It forward physics analysis with similar unconnected dijet signal topologies it is common to use variables like the fraction of tracks inside the selected jets to isolate the signal from background \cite{ARTICLE:AnalysisDiffractiveJets}. Considering tracks has the advantages of not depending on jet clustering or veto jet selection and using directly the excellent resolution of the \gls{CMS} inner tracking system. 

Two variables were considered for inclusion in a cut or \gls{MVA} based signal selection criteria, the fraction of track coming from the \gls{PV} contained inside the selected jets and the fraction of summed momentum associated with tracks from the \gls{PV} contained inside the selected jets, $\phi$.

%TODO: Continue here

\begin{figure}[!htb]
\centering
\includegraphics[width=0.45\textwidth]{Chapter06/TrackVariables/Images/Tracks1_TracksNRatio.pdf} 
\includegraphics[width=0.45\textwidth]{Chapter06/TrackVariables/Images/Tracks1_TracksERatio.pdf} \\
\includegraphics[width=0.45\textwidth]{Chapter06/TrackVariables/Images/Tracks1_CJVPass_TracksERation.pdf}
\caption{TODO}
\label{FIGURE:PreparationParkedDataAnalysis_TrackDistributionVariables_Selection}
\end{figure}

\begin{figure}[!htb]
\centering
\includegraphics[width=0.45\textwidth]{Chapter06/TrackVariables/Images/BB_Tracks1_TracksERatio.pdf}
\includegraphics[width=0.45\textwidth]{Chapter06/TrackVariables/Images/BE_Tracks1_TracksERatio.pdf} \\
\includegraphics[width=0.45\textwidth]{Chapter06/TrackVariables/Images/EE_Tracks1_TracksERatio.pdf}
\caption{TODO}
\label{FIGURE:PreparationParkedDataAnalysis_TrackDistributionVariables_Acceptance}
\end{figure}

 %Status: DONE
   \chapter{Parked Data Analysis}
\label{CHAPTER:ParkedDataAnalysis}

\glsresetall % Resetting all acronyms

This chapter describes the analysis performed over the \gls{CMS} Run I parked data collected over 2012 and 2013. This data was collected and stored without reconstruction and only became fully available a few months after data taking was finished. The advantage of this dataset is the possibility to use lower threshold triggers which can collect more signal but also more backgrounds. To take full advantage of this data the analysis had to be redesigned and extended with new control regions.

% The search for an invisible decay of a vector boson produced Higgs boson was first made public with CMS Physics Analysis Summary (PAS) HIG-13-013 which was further improved and combined with other Higgs boson production channel in the CMS paper HIG-13-30. Additional support material can be found at the CMS Analysis Notes (AN) AN-2012/403 \cite{CMS_AN_2013-403} and AN-2013/205.
% 
% During the 2012 data taking run two main streams of data were recorded. The main stream with an event rate of the order of 300 Hz to be promptly reconstructed and made available for analysis in a few days after being recorded, this dataset is referred to as the prompt data. The secondary stream with lower trigger thresholds with an event rate of the order of 1kHz which would only be reconstructed when the computing resources would be available outside of the data taking period, dataset is referred to as parked data. Our previous results
% were produced using the prompt data only and this work now extends on previous work by using the now available parked data. Since this dataset has been recorded with lower trigger thresholds the analysis was re-optimised to take advantage of this new available phase space. The details of the newly developed analysis can be found in CMS AN-14-243\cite{CMS_AN_2014-243}.
% 
%TODO: references? previous chapter?

%%%%%%%%%%%%%%%%%%%%%%%%%%%%%%%%%%%%%%%%%%%%%%%%%%%%%%%%%%%%%%%%%%%%%%%%%%%%%%%%%%%%
%%% SECTION
%%%%%%%%%%%%%%%%%%%%%%%%%%%%%%%%%%%%%%%%%%%%%%%%%%%%%%%%%%%%%%%%%%%%%%%%%%%%%%%%%%%%
\section{The Cross Check Analysis}

It is normally a requirement for many CMS publications to have a cross check analysis implemented independently from the main result in order to be able to ensure accuracy of the final results due to possible errors with the software implementation. For this purpose the previous prompt data \gls{VBF} Higgs to Invisible results and publication were produced by two different and independent code frameworks and before publication a good level of synchronization was obtained. Due to lack of man power and time it was decide for the 2012-13 parked data analysis to only proceed with a single framework. At a later stage of the analysis it was thought that at least some level of cross check would be a good measure to limit the possibility of implementation errors and to allow extra confidence on the final results.
 
This cross check analysis starts from the same ntuples produced by the main analysis which were produced over all the relevant datasets. This ntuples are recorded with data formats also used by other analysis at Imperial College London, e.g. both the \gls{SM} and \gls{MSSM} Higgs to $\tau\bar{\tau}$, the Higgs to $\tau\bar{\tau}b\bar{b}$ and prompt Higgs to invisible analyses. No cuts are applied at ntuple production except the official \gls{CMS} selection for good usable data using the appropriate golden \gls{JSON} file.
 
To analyse those initial ntuple an independent code framework was developed in order to replicate all relevant numbers and plots produced by the main analysis.

%%%%%%%%%%%%%%%%%%%%%%%%%%%%%%%%%%%%%%%%%%%%%%%%%%%%%%%%%%%%%%%%%%%%%%%%%%%%%%%%%%%%
%%% SECTION
%%%%%%%%%%%%%%%%%%%%%%%%%%%%%%%%%%%%%%%%%%%%%%%%%%%%%%%%%%%%%%%%%%%%%%%%%%%%%%%%%%%%
\section{Data and MC samples}

%%%%%%%%%%%%%%%%%%%%%%%%%%%%%%%%%%%%%%%%%%%%%%%%%%%%%%%%%%%%%%%%%%%%%%%%%%%%%%%%%%%%
%%% SUBSECTION
%%%%%%%%%%%%%%%%%%%%%%%%%%%%%%%%%%%%%%%%%%%%%%%%%%%%%%%%%%%%%%%%%%%%%%%%%%%%%%%%%%%%
\subsection{Data}

In this analysis we used the full certified data with collisions at $\sqrt{s}=8\,\TeV$ from 2012-13 data acquisition (Run I), using golden JSON file \verb|Cert_190456-208686_8TeV_22Jan2013ReReco_Collisions12_JSON.txt| was used in this analysis. It amounts to an integrated luminosity of $19.2 \pm 0.5 \,\femto\barn^{-1}$. A summary of the dataset names and their integrated luminosity can be found in table \ref{TABLE:ParkedData_Data_RunI_IntegratedLuminosity}.

\begin{table}[!htb]
\centering
\begin{tabular}{|l|c|}
\hline
Dataset & $\int{Luminosity}$ $[pb^{-1}]$ \\
\hline \hline
/MET/Run2012A-22Jan2013-v1/AOD & 889 \\
/VBF1Parked/Run2012B-22Jan2013-v1/AOD & 3871 \\
/VBF1Parked/Run2012C-22Jan2013-v1/AOD & 7152 \\
/VBF1Parked/Run2012D-22Jan2013-v1/AOD & 7317 \\
\hline
Total analysed & 19229 \\
\hline \hline
Total certified luminosity & 19789 \\
\hline
\end{tabular}
\caption{Relevant parked datasets from Run I and their total analysed integrated luminosity. Total analysed and certified also showed.}
\label{TABLE:ParkedData_Data_RunI_IntegratedLuminosity}
\end{table}


The difference between certified and analysed datasets is due to out analysis trigger not being active for the first few runs of of Run 2012B. 

%%%%%%%%%%%%%%%%%%%%%%%%%%%%%%%%%%%%%%%%%%%%%%%%%%%%%%%%%%%%%%%%%%%%%%%%%%%%%%%%%%%%
%%% SUBSECTION
%%%%%%%%%%%%%%%%%%%%%%%%%%%%%%%%%%%%%%%%%%%%%%%%%%%%%%%%%%%%%%%%%%%%%%%%%%%%%%%%%%%%
\subsection{Monte Carlo Samples}


%%%%%%%%%%%%%%%%%%%%%%%%%%%%%%%%%%%%%%%%%%%%%%%%%%%%%%%%%%%%%%%%%%%%%%%%%%%%%%%%%%%%
%%% SECTION
%%%%%%%%%%%%%%%%%%%%%%%%%%%%%%%%%%%%%%%%%%%%%%%%%%%%%%%%%%%%%%%%%%%%%%%%%%%%%%%%%%%%
\section{Monte Carlo to Data correction factors}

%%%%%%%%%%%%%%%%%%%%%%%%%%%%%%%%%%%%%%%%%%%%%%%%%%%%%%%%%%%%%%%%%%%%%%%%%%%%%%%%%%%%
%%% SUBSECTION
%%%%%%%%%%%%%%%%%%%%%%%%%%%%%%%%%%%%%%%%%%%%%%%%%%%%%%%%%%%%%%%%%%%%%%%%%%%%%%%%%%%%
\subsection{Pile-up}

%%%%%%%%%%%%%%%%%%%%%%%%%%%%%%%%%%%%%%%%%%%%%%%%%%%%%%%%%%%%%%%%%%%%%%%%%%%%%%%%%%%%
%%% SUBSECTION
%%%%%%%%%%%%%%%%%%%%%%%%%%%%%%%%%%%%%%%%%%%%%%%%%%%%%%%%%%%%%%%%%%%%%%%%%%%%%%%%%%%%
\subsection{Trigger efficiency}
\label{SUBSECTION:ParkedDataAnalysis_CorrectionFactors_TriggerEfficiency}

%Status: DONE

The initial event selection for this analysis starts with a dedicated set of triggers which were recorded to the \gls{VBF} parked dataset. We used the following \gls{HLT} trigger paths for each Run I era:

\begin{itemize}
  \item Run A:       \verb!HLT_DiPFJet40_PFMETnoMu65_MJJ800VBF_AllJets!
  \item Run B and C: \verb!HLT_DiJet35_MJJ700_AllJets_DEta3p5_VBF!
  \item Run D:       \verb!HLT_DiJet30_MJJ700_AllJets_DEta3p5!
\end{itemize}

All the used paths are seeded by \gls{L1T} trigger condition \verb!L1_ETM40!. 


To maximize the usage of the event statistics of the selected \gls{MC} samples, we do not veto events that fail the trigger conditions. Instead an event by event weight is calculated which which taking into account how much luminosity was recorded by each of the three trigger and depending on the offline quantities which correspond to the ones used in the trigger conditions: PFMETnoMu, leading dijet $m_{jj}$ and sub-leading jet \pt. 

To define the weights, turn on curves were determined according to these offline variables as a function of PFMETnoMu in bins of dijet $m_{jj}$ and sub-leading jet \pt. This approach allows the determination of the weights which include the correlations between these variables. The turn on curves are obtained by fitting equation \ref{EQUATION:ParkedDataAnalysis_TriggerEfficiency_Efficiency} to each bin.

\begin{equation}
\frac{\varepsilon_{max}}{2}\text{Erf}\left(\frac{x-x_{0}}{\sqrt{\Gamma}}\right)+1,
\label{EQUATION:ParkedDataAnalysis_TriggerEfficiency_Efficiency} 
\end{equation}

Where $\varepsilon_{max}$ os the maximum efficiency of the trigger in the bin, $x_{0}$ is the mid-value of the turn on and $\Gamma$ is the width of the turn on.

%%%%%%%%%%%%%%%%%%%%%%%%%%%%%%%%%%%%%%%%%%%%%%%%%%%%%%%%%%%%%%%%%%%%%%%%%%%%%%%%%%%%
%%% SUBSECTION
%%%%%%%%%%%%%%%%%%%%%%%%%%%%%%%%%%%%%%%%%%%%%%%%%%%%%%%%%%%%%%%%%%%%%%%%%%%%%%%%%%%%
\subsection{Lepton Identification}

%%%%%%%%%%%%%%%%%%%%%%%%%%%%%%%%%%%%%%%%%%%%%%%%%%%%%%%%%%%%%%%%%%%%%%%%%%%%%%%%%%%%
%%% SUBSECTION
%%%%%%%%%%%%%%%%%%%%%%%%%%%%%%%%%%%%%%%%%%%%%%%%%%%%%%%%%%%%%%%%%%%%%%%%%%%%%%%%%%%%
\subsection{Top reweighting}

%%%%%%%%%%%%%%%%%%%%%%%%%%%%%%%%%%%%%%%%%%%%%%%%%%%%%%%%%%%%%%%%%%%%%%%%%%%%%%%%%%%%
%%% SECTION
%%%%%%%%%%%%%%%%%%%%%%%%%%%%%%%%%%%%%%%%%%%%%%%%%%%%%%%%%%%%%%%%%%%%%%%%%%%%%%%%%%%%
\section{Signal event selection}

%%%%%%%%%%%%%%%%%%%%%%%%%%%%%%%%%%%%%%%%%%%%%%%%%%%%%%%%%%%%%%%%%%%%%%%%%%%%%%%%%%%%
%%% SECTION
%%%%%%%%%%%%%%%%%%%%%%%%%%%%%%%%%%%%%%%%%%%%%%%%%%%%%%%%%%%%%%%%%%%%%%%%%%%%%%%%%%%%
\section{Control Regions}

%%%%%%%%%%%%%%%%%%%%%%%%%%%%%%%%%%%%%%%%%%%%%%%%%%%%%%%%%%%%%%%%%%%%%%%%%%%%%%%%%%%%
%%% SUBSECTION
%%%%%%%%%%%%%%%%%%%%%%%%%%%%%%%%%%%%%%%%%%%%%%%%%%%%%%%%%%%%%%%%%%%%%%%%%%%%%%%%%%%%
\subsection{Top background estimation}

%%%%%%%%%%%%%%%%%%%%%%%%%%%%%%%%%%%%%%%%%%%%%%%%%%%%%%%%%%%%%%%%%%%%%%%%%%%%%%%%%%%%
%%% SUBSECTION
%%%%%%%%%%%%%%%%%%%%%%%%%%%%%%%%%%%%%%%%%%%%%%%%%%%%%%%%%%%%%%%%%%%%%%%%%%%%%%%%%%%%
\subsection{W background estimation}

%%%%%%%%%%%%%%%%%%%%%%%%%%%%%%%%%%%%%%%%%%%%%%%%%%%%%%%%%%%%%%%%%%%%%%%%%%%%%%%%%%%%
%%% SUBSECTION
%%%%%%%%%%%%%%%%%%%%%%%%%%%%%%%%%%%%%%%%%%%%%%%%%%%%%%%%%%%%%%%%%%%%%%%%%%%%%%%%%%%%
\subsection{W background estimation}

%%%%%%%%%%%%%%%%%%%%%%%%%%%%%%%%%%%%%%%%%%%%%%%%%%%%%%%%%%%%%%%%%%%%%%%%%%%%%%%%%%%%
%%% SUBSUBSECTION
%%%%%%%%%%%%%%%%%%%%%%%%%%%%%%%%%%%%%%%%%%%%%%%%%%%%%%%%%%%%%%%%%%%%%%%%%%%%%%%%%%%%
\subsubsection{W to electron+neutrino}

%%%%%%%%%%%%%%%%%%%%%%%%%%%%%%%%%%%%%%%%%%%%%%%%%%%%%%%%%%%%%%%%%%%%%%%%%%%%%%%%%%%%
%%% SUBSUBSECTION
%%%%%%%%%%%%%%%%%%%%%%%%%%%%%%%%%%%%%%%%%%%%%%%%%%%%%%%%%%%%%%%%%%%%%%%%%%%%%%%%%%%%
\subsubsection{W to muon+neutrino}

%%%%%%%%%%%%%%%%%%%%%%%%%%%%%%%%%%%%%%%%%%%%%%%%%%%%%%%%%%%%%%%%%%%%%%%%%%%%%%%%%%%%
%%% SUBSUBSECTION
%%%%%%%%%%%%%%%%%%%%%%%%%%%%%%%%%%%%%%%%%%%%%%%%%%%%%%%%%%%%%%%%%%%%%%%%%%%%%%%%%%%%
\subsubsection{W to tau+neutrino}

%%%%%%%%%%%%%%%%%%%%%%%%%%%%%%%%%%%%%%%%%%%%%%%%%%%%%%%%%%%%%%%%%%%%%%%%%%%%%%%%%%%%
%%% SUBSECTION
%%%%%%%%%%%%%%%%%%%%%%%%%%%%%%%%%%%%%%%%%%%%%%%%%%%%%%%%%%%%%%%%%%%%%%%%%%%%%%%%%%%%
\subsection{Z background estimation}

%%%%%%%%%%%%%%%%%%%%%%%%%%%%%%%%%%%%%%%%%%%%%%%%%%%%%%%%%%%%%%%%%%%%%%%%%%%%%%%%%%%%
%%% SUBSECTION
%%%%%%%%%%%%%%%%%%%%%%%%%%%%%%%%%%%%%%%%%%%%%%%%%%%%%%%%%%%%%%%%%%%%%%%%%%%%%%%%%%%%
\subsection{QCD background estimation}

%%%%%%%%%%%%%%%%%%%%%%%%%%%%%%%%%%%%%%%%%%%%%%%%%%%%%%%%%%%%%%%%%%%%%%%%%%%%%%%%%%%%
%%% SUBSUBSECTION
%%%%%%%%%%%%%%%%%%%%%%%%%%%%%%%%%%%%%%%%%%%%%%%%%%%%%%%%%%%%%%%%%%%%%%%%%%%%%%%%%%%%
\subsubsection{QDC VBF-like + MET Monte Carlo Sample}

%%%%%%%%%%%%%%%%%%%%%%%%%%%%%%%%%%%%%%%%%%%%%%%%%%%%%%%%%%%%%%%%%%%%%%%%%%%%%%%%%%%%
%%% SUBSUBSECTION
%%%%%%%%%%%%%%%%%%%%%%%%%%%%%%%%%%%%%%%%%%%%%%%%%%%%%%%%%%%%%%%%%%%%%%%%%%%%%%%%%%%%
\subsubsection{Data driven QCD estimation}

%%%%%%%%%%%%%%%%%%%%%%%%%%%%%%%%%%%%%%%%%%%%%%%%%%%%%%%%%%%%%%%%%%%%%%%%%%%%%%%%%%%%
%%% SECTION
%%%%%%%%%%%%%%%%%%%%%%%%%%%%%%%%%%%%%%%%%%%%%%%%%%%%%%%%%%%%%%%%%%%%%%%%%%%%%%%%%%%%
\section{Systematics}

%%%%%%%%%%%%%%%%%%%%%%%%%%%%%%%%%%%%%%%%%%%%%%%%%%%%%%%%%%%%%%%%%%%%%%%%%%%%%%%%%%%%
%%% SECTION
%%%%%%%%%%%%%%%%%%%%%%%%%%%%%%%%%%%%%%%%%%%%%%%%%%%%%%%%%%%%%%%%%%%%%%%%%%%%%%%%%%%%
\section{Results}

%%%%%%%%%%%%%%%%%%%%%%%%%%%%%%%%%%%%%%%%%%%%%%%%%%%%%%%%%%%%%%%%%%%%%%%%%%%%%%%%%%%%
%%% SECTION
%%%%%%%%%%%%%%%%%%%%%%%%%%%%%%%%%%%%%%%%%%%%%%%%%%%%%%%%%%%%%%%%%%%%%%%%%%%%%%%%%%%%
\section{Extraction of limits}

% %%%%%%%%%%%%%%%%%%%%%%%%%%%%%%%%%%%%%%%%%%%%%%%%%%%%%%%%%%%%%%%%%%%%%%%%%%%%%%%%%%%%
% %%% SECTION
% %%%%%%%%%%%%%%%%%%%%%%%%%%%%%%%%%%%%%%%%%%%%%%%%%%%%%%%%%%%%%%%%%%%%%%%%%%%%%%%%%%%%
% \section{Event quality filters}
% 
% During data recording issues may happen with the detector or data acquisition which may render some of the events unusable. The groups responsible for each part of the detector and physics object check the data after it was taken and if they find such problems occurred. This groups produce software event filters analysts to be able to remove this problematic events. This issues cover from know detector problems to miss firing of calibration sequences or even failure to reconstruct physics objects.
% 
% The Jet-MET Particle Object Group (POG) recommends the usage of the following filters which are used in this analysis.
% % \cite{CMS:JetMETPOG:MissingETOptionalFilters}
% 
% \begin{itemize}
%   \item CSCTightHaloFilter
%   \item HBHENoiseFilter
%   \item EcalDeadCellTriggerPrimitiveFilter
%   \item trackingFailureFilter
%   \item eeBadScFilter
%   \item ECAL Laser filter
%   \item HCAL Laser filter
% \end{itemize}
% 
% In turn the JetMET group recommend the usage of the following Tracking POG Filter:
% %\cite{CMS:TrackingPOG:TrackingPOGFilters}
% 
% 
% \begin{itemize}
%   \item logErrorTooManyClusters
%   \item manystripclus53X
%   \item toomanystripclus53X
% \end{itemize}
            %Status: DONE
   \chapter{Run II Preparation}

\section{Run II trigger studies}

\section{Run II QCD Monte Carlo samples}
              %Status: DONE
   \chapter{Conclusions}

This thesis has presented the work developed to implement effective monitoring tools for the \gls{CMS} \gls{L1T} system, two physics analysis performed using the \gls{LHC} Run I data using the \gls{CMS} detector and the preparation of the future Run II analysis. The presented physics analysis are focused on the search for the \gls{VBF} produced \gls{SM} Higgs boson decaying invisibly using initially \textit{prompt data} and later \textit{parked data}. This search main objective was to attempt to observe excesses which could point to the direct production of dark matter via Higgs decay.

The developed tools for the \gls{CMS} \gls{L1T} system, were using during Run I and provide the ability to monitoring benchmark trigger rates and synchronization, monitor the correct operation of the \gls{BPTX} system, detect problematic regions of the detector and allow easy trouble shooting for the shift crew. These tools also played a crucial role the data certification for physics usage allowing identification of problematic periods of data taking.

                   
\end{mainmatter}

\begin{backmatter}
  

\bibliographystyle{h-physrev}
\bibliography{PhDThesis_JoaoPela}
  
\glsaddall
\printglossaries
\end{backmatter}

\end{document}
