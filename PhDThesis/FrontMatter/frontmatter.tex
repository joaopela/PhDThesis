%% Title
%!!UPDATE THIS
\titlepage[of Imperial College London]%
{A dissertation submitted to Imperial College London\\
  for the degree of Doctor of Philosophy}
  
\input{cpright}

%% Abstract
\begin{abstract}%[\smaller \thetitle\\ \vspace*{1cm} \smaller {\theauthor}]
The Compact Muon Solenoid (CMS) is a general-purpose particle detector at the CERN Large Hadron Collider (LHC). The goal of this experiment is to search for the Higgs boson and evidence of new physics and to test the prediction of the Standard Model (SM) at the TeV scale. This thesis describes the analysis of proton-proton collision data recorded by CMS during 2012 and support work for data taking during the same period.

A search for the Vector Boson Fusion (VBF) produced Higgs boson invisible decays, using $19.5\,\femto\barn^-1$ of data recorded with prompt reconstruction triggers at a center of mass energy of $8\,\TeV$, is presented. Events are selected with two forward jets and large Missing Transverse Momentum. Assuming the SM VBF production cross section and acceptance, the observed (expected) upper limit at the 95\% confidence level on the \BRinv\, is determined to be of 65\% (49\%) for $m_H=125\,\GeV$.

A second search for the VBF Higgs boson invisible decays, using $19.2\,\femto\barn^-1$ of data recorded with delayed reconstruction (parked) triggers at a center of mass energy of $8\,\TeV$, is also presented. A new event selection criteria was developed taking advantage of the lower trigger requirements. Assuming the SM VBF Higgs production cross section and acceptance, the observed (expected) upper limit at the 95\% confidence level on the \BRinv\, is determined to be of 57\% (40\%) for $m_H=125\,\GeV$.

Monitoring for the CMS Level 1 Trigger system has been developed and used during the 2012 and subsequent LHC data acquisition periods. Contribution to the high reliability of this system during data taking and providing crucial information for validation of the data quality.
\end{abstract}

%DONE

%% Declaration
\begin{declaration}
I declare that the work contained in this thesis is my own, and all results and figures taken from other sources are indicated in the text and referenced appropriately. The analyses presented in this thesis were developed in close collaboration with other members of the CMS experiment.

For the Vector Boson Fusion (VBF) Higgs to invisible prompt analysis I have contributed with the literature review and final background cross section input values for background normalization. QCD multi-jet background studies were also preformed with the target of improving the final selection or prepare the parked analysis, this analysis is presented to give context for the more important work developed later. 

For the VBF Higgs to invisible parked analysis I have participated in the development of the parked trigger, which was used to record the majority of the analysed data. I have continued the QCD multi-jet background studies which have lead to the production of dedicated Monte Carlo simulations and novel approaches to reject this type of events. I have been the responsible for the cross check analysis of the main result which has successfully validated the implementation of the main analysis \cite{ARTICLE:CMSVBFHiggsInvisibleParkedAnalysisPAS}. It is a normal a requirement for many \gls{CMS} publications to have a cross check analysis implemented independently from the main result in order to be able to ensure the accuracy of the final results. I have also participated in the preparation of the Run II analysis where I have lead the development of both the triggers used for data recording during 2015. Additionally, I have developed a method to create the first QCD multi-jet Monte Carlo sample with no MET requirements with signal like properties.

As part of the CMS Level 1 Trigger (L1T) Detector Performance Group (DPG) I have developed monitoring tools for this system, which were both used for real-time monitoring and posterior data certification for physics analysis usage. My work in this group has lead to my appointment for two years to the position of coordinator of the \gls{CMS} \gls{L1T} Data quality Monitoring software development team. Field work was also performed by doing shifts as Trigger and Shift Leader in the experiment control room and on call shifts as the Trigger Detector on Call (DOC) expert.
 
\vspace*{1cm}
\begin{flushright}
João Pela
\end{flushright}
\end{declaration}

%% Acknowledgements
\begin{acknowledgements}
I would like to thank the Imperial College HEP Group and the FCT for supporting and giving me the opportunity to pursue research including my time spent at CERN. Thanks to my supervisor, David Colling, for his guidance during the years of my PhD, the many useful discussions and his enthusiasm for particle physics research. Thanks must also go to all my colleagues of the CMS VBF Higgs to invisible group with whom I have worked directly, Patrick Dunne, Anne-Marie Magnan, Alexandre Nikitenko, Jim Brooke and Chayanit Asawatangtrakuldee. Thanks to everyone in CMS L1T DPG group, specially to Carlo Battilana and Arno Heister for their support. Thanks to all my friends that supported me on my way to finish this degree, it is difficult to mention all of you, but here is a small list, Antonio Pacheco, André David, Alice Mitchell, Chris Holdsworth-Swan, Francisco Sousa, Glenn Spiers, Guillermo Marrero, João Filipe, Miguel Cunhal, Miguel Machado, Nadia Silva, Raquel Monteiro, and many, many more. Finally, and most importantly, I thank my parents and grandparents for their constant support, love and encouragement.
\end{acknowledgements}

\clearpage

\vspace*{\fill}
The work presented in this thesis was supported by the Portuguese Government through \gls{FCT} in the form of my PhD grant with the reference SFRH/BD/77979/2011. I am thankful for their support which allowed me to attain higher education.
\vspace*{\fill}

\begin{center}
\resizebox{1.0\linewidth}{!}{
\includegraphics{FrontMatter/Images/FundingBand.png}
}
\end{center}


% \begin{preface}
% Thesis structure and so on...
% \end{preface}

\dedication{To my grandmother.}

%% Preface
%\begin{preface}
%\end{preface}

%% ToC
\tableofcontents

\newpage
\listoffigures

\newpage
\listoftables

%% Strictly optional!
%\frontquote%
%{Writing in English is the most ingenious torture\\
%   ever devised for sins committed in previous lives.}%
%  {James Joyce}
 
% \titlepage[of the High Energy Physics Group]{%
%   A dissertation submitted to the Imperial College London\\ for the degree of Doctor of Philosophy}
